\chapter{Formale
Angaben\label{/mi-2017/selbstbericht/0000-formale-angaben/0000-formale-angaben}}\label{formale-angabenpathlabelmi-2017selbstbericht0000-formale-angaben0000-formale-angaben}

Es handelt sich um konsekutive Studiengänge mit den Abschlüssen Bachelor
und Master. Die formalen Angaben werden daher für die verschiedenen
Studiengänge getrennt gemacht.

\section{Bachelor
Medieninformatik\label{/mi-2017/selbstbericht/0000-formale-angaben/0000-formale-angaben}}\label{bachelor-medieninformatikpathlabelmi-2017selbstbericht0000-formale-angaben0000-formale-angaben}

\begin{description}
\tightlist
\item[Bezeichnung des Studiengangs in deutsch]
Bachelor Medieninformatik
\item[Bezeichnung des Studiengangs in englisch]
Bachelor in media informatics
\item[Unterrichtssprache]
Deutsch
\item[Kontaktperson]
Prof.~Christian Noss, christian.noss@th-koeln.de, +49 171 79 19 249, +49
2261 8196 6412
\item[Web Adresse]
http://www.medieninformatik.th-koeln.de/bachelor
\item[Zuordnung zu einem Profil]
n/a
\item[Einordnung konsekutiv/ nicht konsekutiv]
Konsekutiv
\item[Zu verleihender Hochschulgrad]
Bachelor of Science
\item[Regelstudienzeit]
Sechs Semester
\item[Studienbeginn]
Jeweils zum Wintersemester
\end{description}

\section{Master
Medieninformatik\label{/mi-2017/selbstbericht/0000-formale-angaben/0000-formale-angaben}}\label{master-medieninformatikpathlabelmi-2017selbstbericht0000-formale-angaben0000-formale-angaben}

\begin{description}
\tightlist
\item[Bezeichnung des Studiengangs in deutsch]
Master Medieninformatik
\item[Bezeichnung des Studiengangs in englisch]
Master in media informatics
\item[Unterrichtssprache]
Deutsch
\item[Kontaktperson]
Prof.~Dr.~Mario Winter, mario.winter@th-koeln.de, +49 2261 8196 6285
\item[Web Adresse]
http://www.medieninformatik.th-koeln.de/master
\item[Zuordnung zu einem Profil]
anwendungsorientierter Studiengang
\item[Einordnung konsekutiv/ nicht konsekutiv]
Konsekutiv, vertiefend zum Studiengang Medieninformatik (Bachelor)
\item[Zu verleihender Hochschulgrad]
Master of Science
\item[Regelstudienzeit]
Vier Semester
\item[Studienbeginn]
Zum Winter- und Sommersemester
\end{description}

\chapter{Einbettung der Studiengänge in die
Hochschule\label{/mi-2017/selbstbericht/0040-einbettung-in-die-hochschule/0000-einbettung-in-die-hochschule}}\label{einbettung-der-studienguxe4nge-in-die-hochschulepathlabelmi-2017selbstbericht0040-einbettung-in-die-hochschule0000-einbettung-in-die-hochschule}

\section{Kurzüberblick über die Struktur der
Hochschule\label{/mi-2017/selbstbericht/0040-einbettung-in-die-hochschule/0000-einbettung-in-die-hochschule}}\label{kurzuxfcberblick-uxfcber-die-struktur-der-hochschulepathlabelmi-2017selbstbericht0040-einbettung-in-die-hochschule0000-einbettung-in-die-hochschule}

\subsection{Profil der
Hochschule\label{/mi-2017/selbstbericht/0040-einbettung-in-die-hochschule/0000-einbettung-in-die-hochschule}}\label{profil-der-hochschulepathlabelmi-2017selbstbericht0040-einbettung-in-die-hochschule0000-einbettung-in-die-hochschule}

Die TH Köln ist die größte Hochschule für angewandte Wissenschaften in
Deutschland. Sie betreibt mehrere Standorte in Köln und unterhält
jeweils einen eigenen Campus in Leverkusen und Gummersbach. Aufgrund
ihrer Größe, der Angebotsvielfalt, ihres Forschungsvolumens und ihrer
internationalen Ausrichtung, versteht sie sich als Hochschule neuen Typs
mit ausgeprägtem Praxisbezug und anwendungsorientierter Forschung.

Die TH Köln gehört der UAS7 an, dem Verbund von sieben leistungsfähigen
Fachhochschulen in Deutschland. Sie ist zudem Vollmitglied in der
European University Association (EUA). Auch Corporate Social
Responsibility ist für die Hochschule kein Fremdwort: sie ist als
familiengerechte Hochschule zertifiziert und eine nach den europäischen
öko-Managementrichtlinien EMAS und ISO 14001 geprüfte umweltorientierte
Einrichtung.

Die TH Köln pflegt eine Lehr- und Lernkultur, welche die zunehmende
Vielfalt der Studierenden in den Blick nimmt und dazu beiträgt, die
Potenziale aller Hochschulangehörigen in den Lernprozess zu integrieren
und dabei zu erschließen. Unter dem Begriff „Gute Lehre`` hat die TH
Köln einen Perspektivwechsel vom Lehrenden zum Lernenden vollzogen. Das
ganze Studium hindurch werden Studierende über Mentoring-, Tutoring- und
Blended Learning-Programme begleitet. Flexiblere Studiengangsmodelle und
hochschuldidaktische Coaching-Angebote gehören ebenso zum Portfolio wie
die Förderung von leistungsstarken und sozial engagierten Studierenden
-- vor allem durch die Beteiligung am Deutschlandstipendium.

Ihre Programme zur hochschuldidaktischen Differenzierung, ihre
Diversity-Konzepte\footnote{\href{https://www.th-koeln.de/hochschule/educational-diversity_5710.php}{Educational
  Diversity Konzept}} und ihr Programm ProfiL2\footnote{\href{https://www.th-koeln.de/mam/downloads/deutsch/hochschule/profil/lehre/profil2_antrag_ministerium.pdf}{ProfiL2
  Antrag der TH Köln}} für projektorientiertes Lehren und Lernen zählen
zu den herausragenden Lehr- und Lernkonzepten in Deutschland. Mithilfe
eines systematischen Qualitätsmanagements entwickelt die TH Köln die
Kompetenzen in den Bereichen Studium und Lehre, Struktur- und
Curriculumentwicklung sowie Hochschuldidaktik permanent weiter.

Die hohe Studierendenzufriedenheit und die breite Anerkennung der
Qualität eines an der TH Köln erworbenen Abschlusses, sind das Fundament
auf dem das Weiterbildungsportfolio der Hochschule aufbaut. Mit
unterschiedlichen Programmen vom Tagesseminar bis hin zum
Weiterbildungsstudium ermöglicht sie Wissenserwerb als
lebensbegleitendes Lernen. Die TH Köln versteht sich als
forschungsorientierte Hochschule für angewandte Wissenschaften. Die
Hochschule achtet bei der Auswahl des wissenschaftlichen Personals
besonders auf die berufliche Reputation und das ausgeprägte
Forschungsinteresse ihrer Lehrenden; sie fördert gezielt
Forschungsaktivitäten mit inter- bzw. transdisziplinärem Charakter. Mit
diesem innovativen Ansatz möchte sie wichtige und zukunftsweisende
Impulse zur gesellschaftlichen Entwicklung setzen. Die TH Köln arbeitet
in der Forschung deshalb intensiv mit der Wirtschaft,
Non-Profit-Organisationen, öffentlichen Einrichtungen und Verbänden,
sowie mit anderen nationalen und internationalen Hochschulen und
Wissenschaftseinrichtungen zusammen.

Die Forschungsaktivitäten beschränken sich nicht alleine auf die
Kompetenzen der Professorinnen, Professoren, wissenschaftlichen
Mitarbeiterinnen und Mitarbeitern. Vor allem über die Masterstudiengänge
bringen auch die Studierenden ihre Kompetenzen und Kreativität in die
Forschungsprojekte ein. Um dem akademischen Nachwuchs eine weitere
wissenschaftliche Karriere zu ermöglichen, bietet die TH Köln verstärkt
kooperative Promotionen mit Universitäten an. Als aktives Mitglied der
InnovationsAllianz der nordrhein-westfälischen Hochschulen sowie der
Patentverwertungsgesellschaft PROvendis engagiert sich die Hochschule
beim Wissenstransfer zwischen Hochschulen, Wirtschaft und Gesellschaft.

Auch international pflegt die TH Köln enge Beziehungen zu anderen
Hochschulen. Sie ist derzeit Partnerin von rund 290 Hochschulen im
Ausland und unterstützt, über ein breites Angebot von
Auslandsaufenthalten und Fördermöglichkeiten, die Mobilität der
Studierenden. So werden mehrere Masterstudiengänge komplett in
englischer Sprache angeboten. Ein Drittel der Studierenden aus dem
Ausland kommt aus Übersee: aus Afrika, Amerika, Asien oder Australien.

\subsection{Lehr- und
Forschungsschwerpunkte\label{/mi-2017/selbstbericht/0040-einbettung-in-die-hochschule/0000-einbettung-in-die-hochschule}}\label{lehr--und-forschungsschwerpunktepathlabelmi-2017selbstbericht0040-einbettung-in-die-hochschule0000-einbettung-in-die-hochschule}

Die TH Köln ist eine forschungsaktive und forschungsstarke Hochschule.
Sie kooperiert national und international mit Universitäten und anderen
Forschungseinrichtungen, da hochwertige Forschung vom fachlichen
Austausch lebt -- über institutionelle und geographische Grenzen hinweg.

Klimawandel, knappe Ressourcen, Sicherheit und demographischer Wandel
sind einige der großen Herausforderungen der nächsten Jahrzehnte. Die
erfahrenen Wissenschaftlerinnen und Wissenschaftler der TH Köln forschen
im Rahmen ihrer anwendungsorientierten und interdisziplinären Projekte
an Lösungen für diese „Great Challenges`` und leisten einen aktiven
Beitrag zur Weiterentwicklung von Wissenschaft, Wirtschaft und
Gesellschaft.

Die vielfältigen Forschungsaktivitäten spiegeln sich im Forschungsprofil
der TH Köln, bestehend aus 10 thematischen Clustern\footnote{\href{https://www.th-koeln.de/forschung/cluster_2734.php}{Forschungscluster}}
wider. Die Cluster dienen als thematische Klammer für die
Forschungsaktivitäten in den unterschiedlichen Forschungsstrukturen der
Hochschule, wie Forschungsinstituten, Kompetenzplattformen,
Forschungsschwerpunkten und Forschungsstellen.

\section{Einbettung der Studiengänge in die
Fakultät\label{/mi-2017/selbstbericht/0040-einbettung-in-die-hochschule/0000-einbettung-in-die-hochschule}}\label{einbettung-der-studienguxe4nge-in-die-fakultuxe4tpathlabelmi-2017selbstbericht0040-einbettung-in-die-hochschule0000-einbettung-in-die-hochschule}

Die Fakultät für Informatik und Ingenieurwissenschaften ist am Standort
Gummersbach angesiedelt (Campus Gummersbach) und ist mit derzeit 5200
Studierenden\footnote{\href{https://th-koeln.github.io/mi-2017/anhaenge/stat-Studentenzahlen_WS-2016_(01.12.2016).pdf}{Statistik
  Studierendenzahlen (01.12.2016)}} die größte Fakultät der TH Köln. An
der Fakultät sind 8 Institute angesiedelt; zum Studienangebot der
Fakultät gehören 8 Bachelor- und 6 Masterstudiengänge. Die
Medieninformatik Studiengänge werden von der Fakultät für Informatik und
Ingenieurwissenschaften ausgerichtet und sind im Institut für Informatik
organisatorisch verankert.

Das Institut für Informatik betreibt Labore für:

\begin{itemize}
\tightlist
\item
  Allgemeine Datenverarbeitung (ADV)
\item
  Systemgestaltung (SG)
\item
  Mathematik \& ihre Anwendungen
\item
  Medieninformatik (MI)
\item
  Mobile und verteilte Informationstechnologie (moxd)
\item
  Kommunikationstechnik \& Datensicherheit (KTDS)
\item
  Wirtschaftsinformatik (WI)
\end{itemize}

\chapter{Analyse der
Studiengänge\label{/mi-2017/selbstbericht/0050-analyse-der-studiengaenge/0000-analyse-der-studiengaenge}}\label{analyse-der-studienguxe4ngepathlabelmi-2017selbstbericht0050-analyse-der-studiengaenge0000-analyse-der-studiengaenge}

Die Studiengänge wurden auf Basis verschiedener quantitativer und
qualitativer Erhebungen analysiert und in einem iterativen Prozess
optimiert. An diesem Prozess waren folgende Personengruppen beteiligt:

\begin{longtable}[c]{@{}ll@{}}
\toprule
\begin{minipage}[b]{0.5\columnwidth}\raggedright\strut
Beteiligte Personengruppe
\strut\end{minipage} &
\begin{minipage}[b]{0.5\columnwidth}\raggedright\strut
Art der Beteiligung
\strut\end{minipage}\tabularnewline
\midrule
\endhead
\begin{minipage}[t]{0.5\columnwidth}\raggedright\strut
Professoren der Medieninformatik-spezifischen Module
\strut\end{minipage} &
\begin{minipage}[t]{0.5\columnwidth}\raggedright\strut
regelmäßige Akkreditierungstreffen
\strut\end{minipage}\tabularnewline
\begin{minipage}[t]{0.5\columnwidth}\raggedright\strut
Professoren der Medieninformatik-übergreifenen Module
\strut\end{minipage} &
\begin{minipage}[t]{0.5\columnwidth}\raggedright\strut
themenspezifische Abstimmungsmeetings, Einzelgespräche
\strut\end{minipage}\tabularnewline
\begin{minipage}[t]{0.5\columnwidth}\raggedright\strut
Studierende
\strut\end{minipage} &
\begin{minipage}[t]{0.5\columnwidth}\raggedright\strut
Evaluationen, Einzelgespräche, Feedbackrunden
\strut\end{minipage}\tabularnewline
\begin{minipage}[t]{0.5\columnwidth}\raggedright\strut
wissenschaftliche Mitarbeiter der Medieninformatik
\strut\end{minipage} &
\begin{minipage}[t]{0.5\columnwidth}\raggedright\strut
Einzelgespräche, Feedbackrunden
\strut\end{minipage}\tabularnewline
\begin{minipage}[t]{0.5\columnwidth}\raggedright\strut
Prüfungsausschuss
\strut\end{minipage} &
\begin{minipage}[t]{0.5\columnwidth}\raggedright\strut
themenspezifische Abstimmungsmeetings, Einzelgespräche
\strut\end{minipage}\tabularnewline
\begin{minipage}[t]{0.5\columnwidth}\raggedright\strut
Prüfungsamt
\strut\end{minipage} &
\begin{minipage}[t]{0.5\columnwidth}\raggedright\strut
themenspezifischen Abstimmungsmeetings, Einzelgespräche
\strut\end{minipage}\tabularnewline
\begin{minipage}[t]{0.5\columnwidth}\raggedright\strut
Qualitätsmanagement-Team
\strut\end{minipage} &
\begin{minipage}[t]{0.5\columnwidth}\raggedright\strut
themenspezifische Abstimmungsmeetings, Einzelgespräche
\strut\end{minipage}\tabularnewline
\begin{minipage}[t]{0.5\columnwidth}\raggedright\strut
Alumni und Wirtschaftsvertreter
\strut\end{minipage} &
\begin{minipage}[t]{0.5\columnwidth}\raggedright\strut
Evaluationen, Einzelgespräche
\strut\end{minipage}\tabularnewline
\bottomrule
\end{longtable}

\chapter{Ist-Zustand\label{/mi-2017/selbstbericht/0100-ist-zustand/0100-ist-zustand}}\label{ist-zustandpathlabelmi-2017selbstbericht0100-ist-zustand0100-ist-zustand}

Mit den Studiengängen der Medieninformatik bietet die Fakultät für
Informatik und Ingenieurwissenschaften der TH Köln seit dem Jahr 2000
ein wissenschaftlich fundiertes und praxisorientiertes
Informatik-Studienprogramm mit dem Schwerpunkt Medien an. Beide
Studiengänge wurden bereits 2004 akkreditiert und 2010 reakkreditiert
und gehören damit zu den ersten erfolgreich akkreditierten Studiengängen
der TH Köln.

Fachlich und strukturell sind sowohl der Bachelorstudiengang als auch
der konsekutive Master-Studiengang auf die Analyse, Konzeption,
Realisierung und Adaption von oft web-basierten Prozessen und Systemen
zur Produktion, Bearbeitung und Distribution medienbasierter
Informationen sowie entsprechender interaktiver Systeme ausgerichtet.
Den Kern bildet ein Informatikstudium. Hinzu kommt die Vermittlung
umfassender, vielschichtiger analytischer wie konstruktiver
Medienkompetenzen sowie ökonomischer, technischer und gesellschaftlicher
Grundkenntnisse. Darüber hinaus werden zeitgemäße Werkzeuge und
Werkzeugketten, Kollaborations- und Entwicklungsmethoden vermittelt und
überwiegend projektorientiert eingeübt.

Die Studiengänge, vor allem der Bachelorstudiengang, erfreuen sich
großer Nachfrage, sowohl von Studierenden als auch von Unternehmen. Beim
unabhängigen Bewertungsportal ``Studycheck.de'' \footnote{\url{http://studycheck.de}}
wird die Medieninformatik am Campus Gummersbach unter den TOP 5
Studiengängen in diesem Segment gelistet \footnote{\href{https://th-koeln.github.io/mi-2017/anhaenge/eva-snapshot_2017_02_17_bewertungen_studiengang_medieninformatik_auf_studycheck_._de.pdf}{Snapshot
  Bewertungen von studycheck.de}}.

Mittlerweile haben etwa 15 Absolventen des Masterstudienganges
Medieninformatik ein Promotionstudium abgeschlossen bzw. sind gerade im
Begriff, diese abzuschliessen. Die Promotionsverfahren fanden bzw.
finden an deutschen (Münster, Paderborn, Tübingen) aber auch an
europäischen (Schweden, Norwegen, Niederlande, Spanien, UK)
Universitäten statt und decken fachlich ein breites Spektrum ab.

\section{Erfüllung der Auflagen der Reakkreditierung
2010\label{/mi-2017/selbstbericht/0100-ist-zustand/0100-ist-zustand}}\label{erfuxfcllung-der-auflagen-der-reakkreditierung-2010pathlabelmi-2017selbstbericht0100-ist-zustand0100-ist-zustand}

Der Technischen Hochschule Köln wurden im Rahmen der Reakkreditierung im
März 2010 folgende Auflagen der Akkreditierungskommission mitgeteilt.

\subsection{Auflagen Medieninformatik
Bachelor\label{/mi-2017/selbstbericht/0100-ist-zustand/0100-ist-zustand}}\label{auflagen-medieninformatik-bachelorpathlabelmi-2017selbstbericht0100-ist-zustand0100-ist-zustand}

\begin{siderules}
\begin{enumerate}
\def\labelenumi{\arabic{enumi}.}
\tightlist
\item
  Die Prüfungsorganisation muss gewährleisten, dass
  studienzeitverlängernde Effekte beim Übergang vom Grund- zum
  Hauptstudium vermieden werden.
\item
  Eine Beschreibung des Moduls Abschlussarbeit muss erstellt werden.
\end{enumerate}
\end{siderules}

Die Auflagen für den Bachelorstudiengang Medieninformatik wurden von der
TH Köln folgendermaßen erfüllt:

\begin{itemize}
\tightlist
\item
  zu 1: In §17 (3) der Bachelorprüfungsordnung wurde der folgende Passus
  ersatzlos gestrichen: Zu den Modulprüfungen des Hauptstudiums (Teil
  1), mit Ausnahme des Moduls ``Netzbasierte Anwendungen'', wird
  zugelassen, wer die Zwischenprüfung mit einer beliebigen Ausnahme
  bestanden hat. Zu den Modulprüfungen des Hauptstudiums (Teil 2) wird
  zugelassen, wer die Zwischenprüfung ohne Ausnahme bestanden hat. Somit
  gibt es keine der Prüfungsorganisation anzulastenden
  studienzeitverlängernden Effekte beim Übergang vom Grund- zum
  Hauptstudium mehr.
\item
  zu 2: Die Beschreibung des Moduls Abschlussarbeit wurde vorgelegt.
\end{itemize}

\subsection{Auflagen Medieninformatik
Master\label{/mi-2017/selbstbericht/0100-ist-zustand/0100-ist-zustand}}\label{auflagen-medieninformatik-masterpathlabelmi-2017selbstbericht0100-ist-zustand0100-ist-zustand}

\begin{siderules}
\begin{enumerate}
\def\labelenumi{\arabic{enumi}.}
\tightlist
\item
  Es muss sichergestellt werden, dass den Studierenden zu Beginn der
  Veranstaltungen die Form der Prüfungsleistungen bekannt gegeben wird
  und diese auf die Ausbildungsziele abgestimmt ist.
\item
  Vorlage der gemäß den Auflagen geänderten und in Kraft gesetzten
  Ordnungen.
\end{enumerate}
\end{siderules}

Diese Auflagen wurden von der Technischen Hochschule Köln folgendermaßen
erfüllt:

\begin{itemize}
\tightlist
\item
  zu 1: Nach unserer Auffassung entspricht die vorgelegte Klausel der
  von der Agentur gewünschten Regelung. Insbesondere ist in §16(4) der
  Prüfungsordnung festgelegt: ``Der Prüfungsausschuss legt in der Regel
  zu Beginn eines Semesters im Benehmen mit den Prüferinnen und Prüfern
  für jedes Modul die Prüfungsform und die Prüfungsmodalitäten \ldots{}
  fest.'' Die von der Agentur angemerkte Zweimonatsfrist bezieht sich
  nur auf die Festlegung des Prüfungszeitraums - nicht auf die Form. Zu
  Semesteranfang bedeutet nach unserer Auffassung 1. April oder 1.
  September des Jahres, also ca. 1 Monat vor Veranstaltungsbeginn. Die
  Flexibilität durch den Passus ``in der Regel'' sollte erhalten
  bleiben, um bspw. auf Erkrankungen oder Ausfall von Dozenten, bzw.
  Lehrbeauftragten oder andere, von außen einwirkende Ereignisse
  reagieren zu können. Die seitens der Studierenden geäußerte Kritik
  hinsichtlich der betreffenden Fristen interpretieren wir so, dass die
  Regelung vermutlich in Ausnahmefällen von einzelnen Dozenten nicht
  vollständig umgesetzt wurde. Von daher erscheint es uns angeraten,
  eine eigentlich in sich konsistente Prüfungsordnung an dieser Stelle
  nicht zu ändern, sondern die Umsetzung zu verbessern. Dazu wird die
  folgende explizite und zentralisierte Verfahrensweise zur
  verbindlichen Bekanntgabe der Prüfungsform zu Beginn des Semesters
  festgelegt:
\item
  Falls die Prüfungsform dem im Internet oder beim
  Studiengangsbeauftragten einsehbaren Modulhandbuch entspricht, gilt
  diese damit als bekannt gemacht. Diese Teilregelung wird dauerhaft im
  Prüfungsamt ausgehängt.
\item
  Sollte die Prüfungsform, bspw. wegen aktueller didaktischer
  Erwägungen, von der im Modulhandbuch bekannt gemachten Form abweichen,
  so ist diese durch den Dozenten dem Prüfungsamt rechtzeitig
  mitzuteilen, welches über die Änderung dann per Aushang fristgerecht
  informiert. § 14 der Prüfungsordnung für den Masterstudiengang
  Medieninformatik wurde dementsprechend neu gefasst.
\item
  zu 2: Die Prüfungsordnungen vom 7. Januar 2011, in denen alle Auflagen
  erfüllt sind, wurden vom Fakultätsrat der Fakultät für Informatik und
  Ingenieurwissenschaften am 10.11.2010 bzw. 6.1.2011 beschlossen und
  vom Präsidenten der TH Köln am 7.1.2011 genehmigt. Sie liegen der
  ASIIN vor.
\end{itemize}

\subsection{Begleitende Betreuung während des
Studiums\label{/mi-2017/selbstbericht/0100-ist-zustand/0100-ist-zustand}}\label{begleitende-betreuung-wuxe4hrend-des-studiumspathlabelmi-2017selbstbericht0100-ist-zustand0100-ist-zustand}

\paragraph{MentoRing Programm des Campus
Gummersbach\label{/mi-2017/selbstbericht/0100-ist-zustand/0100-ist-zustand}}\label{mentoring-programm-des-campus-gummersbachpathlabelmi-2017selbstbericht0100-ist-zustand0100-ist-zustand}

Mit Beginn des Wintersemesters 11/12 wurde das mehrstufige
MentoRing4Excellence© an der Fakultät für Informatik und
Ingenieurwissenschaften der TH Köln eingeführt. Das Mentoringprogramm
spannt ein Netzwerk zwischen Studienanfängern, erfahrenen Studierenden
und externen Führungskräften. Es beinhaltet die Unterprogramme
MentoRing4Beginners© und MentoRing4LeadershipDevelopment©.
MentoRing4Beginners© bietet den Erstsemestern am Campus Gummersbach
Orientierung und Hilfestellung bei ihrem Studienstart. Erfahrene
Studierende stehen den „Neuen`` als Mentoren/innen zur Seite, damit
diese besser in den Studienalltag hineinfinden und schneller Kontakte
knüpfen können. Sie sind während des ersten Semesters Begleitung und
Ansprechpartner in allen Fragen rund ums Studium, z. B. in Bezug auf die
Studienorganisation, die Studienlaufbahnplanung, Lerntechniken,
Prüfungsvorbereitung und Projekte. In einem mehrstufigen Verfahren
werden 15 erfolgreiche Studierende der Informatik und
Ingenieurwissenschaften mit ausgeprägter Sozialkompetenz als
Mentoren/innen ausgewählt und in einem mehrtägigem Intensivtraining auf
ihre Aufgabe vorbereitet. Während des Semesters erhalten sie laufende
Supervision.

Für Studienanfänger und Mentoren bedeutet das Mentoring ein Gewinn. Die
Studienanfänger finden leichter ins Studium und schneller Kontakte. Die
Mentoren erhalten neue Impulse für den eigenen Studienkontext und
entwickeln/stärken die eigene Kommunikations- und Beratungskompetenz.
Das Mentoringprogramm wird evaluiert.

Besonders engagierte, leistungs- und kommunikationsbereite Studierende,
die sich als Mentoren bewährt haben, können sich für das „Leadership
Development Program`` bewerben. Im Rahmen dieses studienbegleitenden
Qualifizierungsprogramms werden den Studierenden externe Führungskräfte
als Mentoren zur Seite gestellt, so dass sie in Gesprächen, in
Projektmitarbeit und durch Einbindung in berufsrelevante Netzwerke von
deren langjähriger Berufs- und Lebenserfahrung profitieren können.

\paragraph{Medieninformatik
Mentor\label{/mi-2017/selbstbericht/0100-ist-zustand/0100-ist-zustand}}\label{medieninformatik-mentorpathlabelmi-2017selbstbericht0100-ist-zustand0100-ist-zustand}

Ergänzend zum MentoRing Programm, dass sich an alle Studienanfänger des
Campus Gummersbach richtet, wurde in der Medieninformatik mit der
letzten Reakkreditierung die Stelle des Medieninformatik Mentor
geschaffen. Diese wird mit erfahrenen wissenschaftlichen
Mitarbeitern/innen besetzt, die selbst zumindest den Bachelorstudiengang
Medieninformatik am Campus Gummersbach absolviert haben. Der
Medieninformatik Mentor fungiert als institutionelles Bindeglied
zwischen Studierenden und Lehrenden und ist bei Problemen und Fragen
rund um das Studium Ansprechpartner für die Studierenden, aber auch für
Lehrende. Mit Hilfe dieser Stelle werden auch wiederkehrende Probleme
sichtbar und können seitens der Studiengangsbetreiber behoben,
verbessert oder zumindest thematisiert werden.

\subsection{Außercurriculare
Maßnahmen\label{/mi-2017/selbstbericht/0100-ist-zustand/0100-ist-zustand}}\label{auuxdfercurriculare-mauxdfnahmenpathlabelmi-2017selbstbericht0100-ist-zustand0100-ist-zustand}

Mehrere gebündelte und ständig weiter entwickelte außercurriculare
Maßnahmen tragen, insbesondere vor dem Hintergrund der stark
ansteigenden Studierendenzahlen, zur weiteren Verbesserung der
Studienqualität bei.

\paragraph{Showcase
\label{/mi-2017/selbstbericht/0100-ist-zustand/0100-ist-zustand}}\label{showcase-pathlabelmi-2017selbstbericht0100-ist-zustand0100-ist-zustand}

Das jährlich durchgeführte Medieninformatik-Showcase dient zur Stärkung
der Identität der Medieninformatik, zur besseren Vernetzung von
Studierenden sowohl zwischen Master- und Bachelorstudierenden als auch
über die Studiensemester, zum Ausblick auf die Praxis durch externe
Sprecher (oft Alumni), sowie als strukturierte Feedbackmöglichkeit. Das
Event verbessert außerdem die Sichtbarkeit der Medieninformatik am
Campus und in der Region.

\paragraph{Social Media
Angebote\label{/mi-2017/selbstbericht/0100-ist-zustand/0100-ist-zustand}}\label{social-media-angebotepathlabelmi-2017selbstbericht0100-ist-zustand0100-ist-zustand}

Die von der Medieninformatik eingerichteten und administrierten Social
Media Angebote in YouTube, Facebook- und Twitter erreichen regelmäßig
etwa 1000 Abonnenten, sprich: Studierende, Interessierte und Alumni. Sie
bieten eine gute Gelegenheit um im Gespräch zu bleiben, Themen und
Arbeitsergebnisse zu platzieren, sowie Studienanfänger, Jobs und
Projekte zu akquirieren oder anzubieten.

\paragraph{\texorpdfstring{Wettbewerb ``Die besten
Projekte''\label{/mi-2017/selbstbericht/0100-ist-zustand/0100-ist-zustand}}{Wettbewerb Die besten Projekte\label{/mi-2017/selbstbericht/0100-ist-zustand/0100-ist-zustand}}}\label{wettbewerb-die-besten-projektepathlabelmi-2017selbstbericht0100-ist-zustand0100-ist-zustand}

Der jährlich vom Labor für Medieninformatik durchgeführte Wettbewerb
``Die besten Projekte'', welcher einerseits gute und sehr gute
Ergebnisse aus dem Bachelor- und dem Masterstudiengang herausstellt,
andererseits in der gemeinsamen Abschlusspräsentation zur Vernetzung
zwischen den Studierenden des Bachelor- und des Masterstudiengangs
beiträgt, und letztendlich den projektorientierten Ansatz in der
Medieninformatik nachhaltig sichtbar macht.

\paragraph{Medieninformatik
Kontaktbörse\label{/mi-2017/selbstbericht/0100-ist-zustand/0100-ist-zustand}}\label{medieninformatik-kontaktbuxf6rsepathlabelmi-2017selbstbericht0100-ist-zustand0100-ist-zustand}

Die bereits beschriebene, einmal im Semester durchgeführte,
Medieninformatik Kontaktbörse dient zur Erleichterung des Übergangs in
das Abschlusssemester, zur Herstellung von Kontakten zu potentiellen
Kooperationspartnern, und zum Geben von Ideen und Inspiration zu Themen
für die Abschlussarbeit.

\paragraph{Medieninformatik-Filmfest\label{/mi-2017/selbstbericht/0100-ist-zustand/0100-ist-zustand}}\label{medieninformatik-filmfestpathlabelmi-2017selbstbericht0100-ist-zustand0100-ist-zustand}

Das jährlich durchgeführte Medieninformatik-Filmfest dient zur Stärkung
der Identität der Medieninformatik, zur besseren Vernetzung der
Studierenden, insbesondere der Studienanfänger und zur Präsentation
ausgewählter Arbeitsergebnisse. Das Event verbessert außerdem die
Sichtbarkeit der Medieninformatik am Campus und in der Region.

\section{Stärken und Schwächen
Analyse\label{/mi-2017/selbstbericht/0100-ist-zustand/0100-ist-zustand}}\label{stuxe4rken-und-schwuxe4chen-analysepathlabelmi-2017selbstbericht0100-ist-zustand0100-ist-zustand}

\subsection{Beurteilung des Studienerfolgs auf der Basis von
Absolventenbefragungen und
Verbleibstudien\label{/mi-2017/selbstbericht/0100-ist-zustand/0100-ist-zustand}}\label{beurteilung-des-studienerfolgs-auf-der-basis-von-absolventenbefragungen-und-verbleibstudienpathlabelmi-2017selbstbericht0100-ist-zustand0100-ist-zustand}

Die folgenden Ausführungen beruhen auf den Erhebungen der
Studiernendenzahlen am Campus Gummersbach innerhalb des letzten
Akkreditierungszeitraums und der Datenerhebung\footnote{\href{https://th-koeln.github.io/mi-2017/anhaenge/stat-verbleib-und-studienabbruch.pdf}{Statistik
  zum Verbleib- und Studienabbruch}} zum 01.12.2015 für den Zeitraum
2011 bis 2015 und fokussieren die derzeit eingeschriebenen Studierenden,
erfolgreiche Abschlüsse und Studienfachabbrecher im
Medieninformatik-Bachelor.

\begin{figure}[htbp]
\centering
\includegraphics[width=\columnwidth]{../anhaenge/bilder/ba-anfaengerzahlen.png}
\caption{Studienanfänger im Bachelorstudiengang Medieninformatik}
\end{figure}

Die Zahlen zeigen einen stetig wachsenden Zulauf für den
Bachelorstudiengang Medieninformatik, der ursprünglich für 63
Studierende ausgelegt wurde. Erfreulicherweise ist aus den letzten
vorliegenden Zahlen von 2014, trotz der im Rahmen der
Fakultätsentwicklung und des Hochschulentwicklungsplans 2020\footnote{\href{https://www.verwaltung.th-koeln.de/imperia/md/content/verwaltung/broschueren_leitfaeden/hochschulentwicklungsplan2020.pdf}{Hochschulentwicklungsplan
  2020}} steigenden Anfängerzahlen, eine gleichbleibende Abbrecherquote
um die 30\%. Die Zahlen zeigen leider auch eine niedrige Quote an
Absolventen in Regelstudienzeit, die jedoch im Mittel aller Studiengänge
der Fakultät 10 liegt. Nach den vorliegenden Prüfungsstatistiken (vgl.
Prüfungsstatistiken \footnote{\href{https://th-koeln.github.io/mi-2017/anhaenge/ba-pruefungsstatistiken.pdf}{Bachelorstudiengang
  Medieninformatik, Prüfungsstatistik 2016}}) ist mit einem
proportionalen Anstieg der Absolventen zu rechnen.

Die im Rahmen der letzten Reakkreditierung eingebrachten Änderungen
können hinsichtlich der Quote der Studienabbrecher bereits als recht
erfolgreich bewertet werden. Vor allem die Auflösung der strikten, durch
Zulassungsvoraussetzungen in der Prüfungsordnung verankerte Trennung von
Grund- und Hauptstudium hat die Dauer des Fachstudiums definitiv
verkürzt. Auch die Einführung des Moduls „Einführung in die
Medieninformatik`` (EMI) erweist sich als sinnvoll und notwendig, um den
Studierenden früh die Perspektiven und fachlichen Aspekte der
Medieninformatik näher zu bringen.

Aus der INCHER-Studie von 2014\footnote{\href{https://th-koeln.github.io/mi-2017/anhaenge/studie-INCHER-Studie.pdf}{INCHER-Studie
  2014}} geht für alle Studiengänge in NRW hervor: Wer während des
Studiums ein Firmenpraktikum absolviert, schließt das Studium etwas
seltener in der Regelstudienzeit ab (54 Prozent vs.~60 Prozent). Ähnlich
ist eine Tendenz zwischen denjenigen, die ihr Studium hauptsächlich
durch Erwerbsarbeit finanzierten und den übrigen Absolventinnen und
Absolventen zu erkennen: Wenn das Studium durch eigene Erwerbsarbeit
finanziert wurde, wird es ebenfalls seltener in der Regelstudienzeit
abgeschlossen (50 Prozent vs.~57 Prozent).

Schlussfolgerungen über die Studienqualität sind auf Grundlage der
verfügbaren Daten nur bedingt möglich. Als Ausgangspunkt für die, im
Rahmen der Reakkreditierung anzustrebenden Änderungen, wurden daher
zusätzlich folgende Quellen mit einbezogen:

\begin{itemize}
\tightlist
\item
  Studentische Rückmeldungen aus den, im Rahmen des Medieninformatik
  Showcase stattfindenden Feedbackrunden
\item
  Persönliche Gespräche mit Studierenden, Alumni und
  Kooperationspartnern
\item
  Probleme und Fragen, die an die Medieninformatik Mentorin und die
  Studiengangsmanager gerichtet wurden
\item
  Befragung der beteiligten Professoren, wissenschaftlichen Mitarbeitern
  und Tutoren
\item
  Rückschlüsse aus Veranstaltungsevaluationen
\item
  Gespräche mit Unternehmensvertretern
\end{itemize}

Auf dieser Basis konnten, bezogen auf die bereits beschrieben
Erkenntnisse der INCHER-Studie, zwei Gründe für verlängerte Studiendauer
ermittelt werden:

\begin{itemize}
\tightlist
\item
  Viele Studierende finanzieren ihr Studium, vor allem in höheren
  Semestern und im Master-Studium, durch Erwerbsarbeit.
\item
  Das große Modul ``Entwicklungsprojekt interaktive Systeme'' (10
  Creditpoints) überfordert viele Studierende.
\item
  Das Praxisprojekt im sechsten Semester wird in der Regel in
  Kooperation mit Unternehmen absolviert.
\end{itemize}

Bei den Vorbereitungen zum Praxisprojekt, das in der Regel im selben
Themenfeld wie die Bachelorarbeit absolviert wird, durchlaufen die
Studierenden in der Regel einen dreistufigen Prozess:

\begin{enumerate}
\def\labelenumi{\arabic{enumi}.}
\tightlist
\item
  Identifikation eines geeigneten Themenfeldes für das Praxisprojext, in
  der Regel in Absprache mit einem oder mehreren Dozenten.
\item
  Bewerbung bei passenden Kooperationspartnern in der Wirtschaft.
\item
  Einarbeitung beim Unternehmen und Einigung auf das finale Thema zum
  Praxisprojekt mit dem Unternehmen und dem Dozenten.
\end{enumerate}

Dieser Prozess ist zeitaufwändig und wird von den meisten Studierenden
unterschätzt und daher häufig zu spät begonnen. Um diesem Problem
entgegen zu wirken wird in der Medieninformatik seit drei Jahren am Ende
des fünften Semesters eine Kontaktbörse durchgeführt. Auf dieser
Veranstaltung werden den künftigen Absolventen die Regularien, Abläufe
und Herausforderungen des Abschlusssemesters erläutert. Darüber hinaus
stellen ausgewählte Unternehmen und Organisationen potenzielle Themen
und Problemfelder für Praxisprojekt und Bachelorarbeit vor. Auch die
Professoren der Informatik haben hier die Möglichkeit ihre Themen und
Forschungsfelder als Ansatzpunkt für mögliche, forschungsnahe
Praxisprojekte und Abschlussarbeiten vorzustellen.

\subsection{Bewertung von Ergebnissen aus
Evaluationen\label{/mi-2017/selbstbericht/0100-ist-zustand/0100-ist-zustand}}\label{bewertung-von-ergebnissen-aus-evaluationenpathlabelmi-2017selbstbericht0100-ist-zustand0100-ist-zustand}

Hier kann auf die Befragungen zur allgemeinen Zufriedenheit\footnote{\href{https://th-koeln.github.io/mi-2017/anhaenge/eva-studierendenbefragung-2010-2015.pdf}{Studierendenbefragung
  2010-2015}} und regelmäßig semesterweise durchgeführte Evaluationen
der Lehrveranstaltungen\footnote{\href{https://th-koeln.github.io/mi-2017/anhaenge/eva-lv-bewertung-mi-module-ba-kern.pdf}{Evaluation
  der studiengangsübergreifenden Module im Bachelor}}\footnote{\href{https://th-koeln.github.io/mi-2017/anhaenge/eva-lv-bewertung-mi-module-ba.pdf}{Evaluation
  der studiengangsspezifischen Module im Bachelor}}\footnote{\href{https://th-koeln.github.io/mi-2017/anhaenge/eva-lv-bewertung-mi-module-ma.pdf}{Evaluation
  der studiengangsspezifischen Module im Master}} verwiesen werden. Die
Auswertung der Evaluationen erfolgt zentral durch das Hochschulreferat 4
\emph{Qualitätsmanagement}. Darüber hinaus ist an der Fakultät 10 ein
integriertes Qualitätsmanagement nach DIN/ISO 9001 etabliert. In den
Ergebnissen\footnote{\href{https://th-koeln.github.io/mi-2017/anhaenge/eva-evaluationen-f10.pdf}{Studentische
  Evaluationen Medieninformatik}} zeigt sich grundsätzlich bei den
Bachelorstudierenden ein etwas geringeres Zufriedenheitsmaß als bei den
Masterstudierenden. Dies lässt sich mit Verweis auf die allgemein hohen
Abbruchquoten in grundständigen Informatikstudiengängen ggf. so
interpretieren, dass die Unzufriedenheit nicht allein durch die
Studienangebotsseite verursacht ist. Dennoch lassen sich deutliche
Verbesserungspotentiale identifizieren, etwa bzgl. der Einführung neuer
Lehr- und Lernformate, Koordination der Praktika, Bereitstellung von
studentischen Arbeitsräumen, Gastvorträgen, Exkursionen und Workshops.

Der 2013 zu verzeichnende Rückgang der Zufriedenheit bzgl. des
Lehrangebotes im Master lässt sich nach unseren Analysen und Gesprächen
mit Studierenden u.A. als Auswirkung des ersten, im Informatik-Master
durchgeführten Projekt-Semesters interpretieren. Die dort durchgeführten
„Guided Projects`` zeigen einen starken Praxisbezug und eine klare, mit
den Methoden des (oft agilen) Projektmanagements gestaltete
Ablaufstruktur. Diese auf den Arbeitsmarkt ausgerichtete
Herangehensweise wird auch von vielen Studierenden im Medieninformatik
Master gewünscht.

Die Unzufriedenheit bei der Bewertung der Studien- und
Prüfungsorganisation in der Fakultät lässt sich auf punktuelle Ausfälle
des ``Prüfungs- und Studierendenservice Online'' (PSSO) und teilweise
nicht optimal im Internet kommunizierte Prüfungsinformationen
zurückführen. Erfreulich ist die weiterhin große Gesamtzufriedenheit der
Studierenden im Medieninformatik Master.

\subsection{Bewertung der statistischen Daten bezüglich der
Auslastung, der Prüfungserfolge, der Abbrecherquoten und der
Studienanfänger- und
Bewerberzahlen.\label{/mi-2017/selbstbericht/0100-ist-zustand/0100-ist-zustand}}\label{bewertung-der-statistischen-daten-bezuxfcglich-der-auslastung-der-pruxfcfungserfolge-der-abbrecherquoten-und-der-studienanfuxe4nger--und-bewerberzahlen.pathlabelmi-2017selbstbericht0100-ist-zustand0100-ist-zustand}

Die folgenden Ausführungen beruhen auf der Datenerhebung zum 22.09.2016
für den Zeitraum Wintersemester 2011 bis Wintersemester 2015\footnote{\href{https://th-koeln.github.io/mi-2017/anhaenge/ba-pruefungsstatistiken.pdf}{Bachelorstudiengang
  Medieninformatik, Prüfungsstatistik 2016}}\footnote{\href{https://th-koeln.github.io/mi-2017/anhaenge/fak-Fakultaetsstruktur-Studienangebot-Personal-Haushaltsmittel-Kennzahlen-2014.pdf}{Fakultät
  für Informatik und Ingenieurwissenschaften, Statistik 2013/14}} und
fokussieren die derzeit eingeschriebenen Studenten, erfolgreiche
Abschlüsse und Studienfachabbrecher.

Zur Bewertung der Auslastung kann wie folgt Stellung genommen werden:
Gemessen an den planmäßigen 63 Studierendenplätzen (WS13/14) werden seit
drei Studienjahren im Rahmen der strategischen Fakultätsplanung und des
Hochschulentwicklungsplans 2020 mehr als 200\% Überlast aufgenommen. Mit
den Abbrecherquoten im Bachelorstudiengang bewegt sich die
Medieninformatik im breiten Mittelfeld von Informatikstudiengängen im
Allgemeinen\footnote{\href{http://www.dzhw.eu/pdf/pub_fh/fh-201503.pdf}{Ulrich
  Heublein et al.: Studienbereichsspezifische Qualitätssicherung im
  Bachelorstudium - Befragung der Fakultäts- und Fachbereichsleitungen
  zum Thema Studienerfolg und Studienabbruch. Forum Hochschule, 3/2015}};
sehr erfreulich ist für den Masterstudiengang Medieninformatik die
geringe Abbrecherquote. In Verbindung mit der bedauerlich hohen, für
ingenieur- und naturwissenschaftliche Studiengänge, insbesondere im
Bachelor-Bereich jedoch leider inhärenten Abbrecherquote
(durchschnittlich geschätzte Schwundquote in der Informatik an
Fachhochschulen ist 39\%\footnote{\href{http://www.dzhw.eu/pdf/pub_fh/fh-201503.pdf}{Ulrich
  Heublein et al.: Studienbereichsspezifische Qualitätssicherung im
  Bachelorstudium - Befragung der Fakultäts- und Fachbereichsleitungen
  zum Thema Studienerfolg und Studienabbruch. Forum Hochschule, 3/2015}}),
zeigt sich hier ein deutliches noch zu hebendes Optimierungspotential.
Erfreulich ist hier die mit 27\% recht hohe Frauenquote im
Bachelorstudiengang Medieninformatik. Die durchschnittliche Frauenquote
in der Lehreinheit Informatik liegt bei 22\%. In der Fakultät 10 liegt
sie bei 20\%.

Die Prüfungserfolge sind bzgl. des Bachelor- und Masterstudiengangs zu
differenzieren.

Im Bachelorstudiengang Medieninformatik zeigt sich bei den
Prüfungserfolgen des „neuen`` im Vergleich zum „alten`` Studiengang
(BPO2 vs.~BPO3, s. Anhang Pruefungsstatistiken \footnote{\href{https://th-koeln.github.io/mi-2017/anhaenge/ba-pruefungsstatistiken.pdf}{Bachelorstudiengang
  Medieninformatik, Prüfungsstatistik 2016}}) ein früherer
Prüfungserfolg. Auch in höheren Semestern werden die Prüfungen früher
absolviert und mit weniger Fehlversuchen bestanden. In erster Näherung
findet man in den ersten beiden Semestern eine Gleichverteilung der
Noten innerhalb des Notenspektrums, die sich in den höheren Semestern zu
einer deutlichen Verbesserung hin verschiebt. Hier mögen zwei Faktoren
von Bedeutung sein: Zum einen der deutlich höhere Anteil an
medien(informatik)spezifischen Modulen und zum anderen kann gemutmaßt
werden, dass sich hier die Abbrecherzahlen positiv auswirken. Die
Abschluss- und die Endnoten setzen diesen Trend der Verbesserung des
Notendurchschnitts fort.

Im Masterstudium wirkt sich die im Rahmen der Reakkreditierung weg
gefallene Zulassungsvoraussetzung eines Mindest-Notenschnittes nicht
wesentlich auf die Verteilung der Prüfungsergebnisse aus. Auch hier ist
weiterhin das gesamte Notenspektrum abgedeckt, ebenso wie bei den
Ergebnissen der Master Thesen.

\subsection{Rückschlüsse aus informellen Gesprächen und
Kommentaren\label{/mi-2017/selbstbericht/0100-ist-zustand/0100-ist-zustand}}\label{ruxfcckschluxfcsse-aus-informellen-gespruxe4chen-und-kommentarenpathlabelmi-2017selbstbericht0100-ist-zustand0100-ist-zustand}

\emph{\ldots{} anstrengend und fordernd, aber macht viel Spaß \ldots{}}

Aus verschiedenen Einzel- und Gruppengesprächen im Team der
Studiengangsbetreiber, Gesprächen mit Studierenden und Alumni,
Kommentaren von Feedbackrunden sowie Online-Foren lassen sich eine Reihe
von Stärken und Schwächen ableiten.

Der Studienaufbau des Bachelorstudiengangs wird überwiegend als positiv
und gut durchdacht bewertet. Die Lehrveranstaltungen werden in Summe als
gut organisiert und vorbereitet, interessant, aber auch als sehr sehr
anspruchsvoll beschrieben. Der Umfang des Studiums wird zuweilen als
``vom Umfang überwältigend'' bezeichnet. Diese Einschätzung wird von
Alumni jedoch dahingehend ergänzt, dass nach dem Einstieg ins
Berufsleben die Wichtigkeit und Relevanz der einzelnen Module offenbar
wurde, sie sich mit dem Studium sehr gut im Beruf platzieren konnten und
in bestimmten Berufszweigen sehr flexibel einsetzbar sind. Das
Verhältnis der allgemeinen Informatik Anteile und der
medieninformatik-spezifischen Module wird gut bewertet.

Durchweg sehr positiv wird die gute und intensive Betreuung durch das
Lehrpersonal beschrieben: ``Die Dozenten sind super hilfreich
\ldots{}''. Dies gilt auch für die vielen praktischen Projekte und
Gruppenarbeiten. Auch die Offenheit für eigene Ideen und die
Gruppengröße bei den Praxisanteilen wird sehr positiv bewertet. Das
Mentoring-Programm wird ebenfalls als sehr hilfreich wahrgenommen.

Auch sehr positiv wird die gute und moderne Ausstattung der
Medieninformatik und der Bibliothek, als auch das recht ausgewogene
quantitative Verhältnis von Frauen und Männern bewertet.

Als problematisch wird, bezogen auf den Bachelorstudiengang, vor allem
die starke Fragmentierung der Module sowie der zugehörigen Praxisanteile
gesehen, sodass die Situation, vor allem im dritten Fachsemester, als
``zu voll'' oder mit ``zu viele Baustellen'' beschrieben wird. Dieses
Problem wurde auch im Rahmen der Analysen zum ProfiL2 Antrag der
Hochschule identifiziert \footnote{\href{https://www.th-koeln.de/mam/downloads/deutsch/hochschule/profil/lehre/profil2_antrag_ministerium.pdf}{ProfiL2
  Antrag der TH Köln}}. Derzeit wird dieser Problematik bereits mit der
sequentiellen Anordnung einiger Module begegnet. Dabei werden zwei
parallel laufende Module nacheinander, dafür aber mit halber Laufzeit
und doppelter SWS Anzahl angeboten, so dass sich die Studierenden auf
weniger Module zur gleichen Zeit konzentrieren können. Diese
Herangehensweise wurde ebenfalls im Rahmen von ProfiL2 als Maßnahme
vorgeschlagen\footnote{\href{https://www.th-koeln.de/mam/downloads/deutsch/hochschule/profil/lehre/profil2_antrag_ministerium.pdf}{ProfiL2
  Antrag der TH Köln}}. Viele Studierende wünschen sich die Möglichkeit
der Fachvertiefung. Das Problem wird häufig mit ``man kratzt alles nur
an und dann kommt schon das nächste Thema'' beschrieben. Gerade bei den
Implementierungs-affinen Studierenden, aber auch bei den Lehrenden wird
häufig der Wunsch nach mehr Unterstützung im Bereich Programmierung
genannt. Dies gilt vor allem für komplexere und größere Projekte.
Derzeit fehlt im Bachelorprogramm ein Modul, dass die Studierenden auf
die rechtlichen Fragestellung in der (Medien-)Informatik vorbereitet.
Dieses Defizit wurde in verschiedenen Feedbackrunden adressiert.

Bezogen auf den Master wird immer wieder die fehlende oder unzureichende
Praxisorientierung als Problem genannt. Auch hier fehlt den Studierenden
die Möglichkeit zur Fachvertiefung entsprechend der persönlichen
Neigung.

\subsection{Ableitungen aus den Bewertungen der zur Verfügung
stehenden Daten und
Evaluationen\label{/mi-2017/selbstbericht/0100-ist-zustand/0100-ist-zustand}}\label{ableitungen-aus-den-bewertungen-der-zur-verfuxfcgung-stehenden-daten-und-evaluationenpathlabelmi-2017selbstbericht0100-ist-zustand0100-ist-zustand}

Aus den Bewertungen der Daten, Evaluationen und Feedbacks lassen sich
folgende Probleme und Schwächen ableiten.

\paragraph{Medieninformatik
Bachelor\label{/mi-2017/selbstbericht/0100-ist-zustand/0100-ist-zustand}}\label{medieninformatik-bachelorpathlabelmi-2017selbstbericht0100-ist-zustand0100-ist-zustand}

Als Indikator für eine gute Studierbarkeit, kann die Anzahl der
abgelegten Prüfungen im vorgesehenen Fachsemester des Moduls angesehen
werden. Ziel ist es, dass die Studierenden Prüfungen möglichst im selben
Semester ablegen, in dem das Modul im Studienverlaufsplan verortet ist.
Gelingt dies nicht, so kann ein Studienabschluss innerhalb der
Regelstudienzeit fast nicht mehr realisiert werden. Ab dem dritten
Studiensemester werden Prüfungen zunehmend verspätet abgelegt (vgl.
Pruefungsstatistiken\footnote{\href{https://th-koeln.github.io/mi-2017/anhaenge/ba-pruefungsstatistiken.pdf}{Bachelorstudiengang
  Medieninformatik, Prüfungsstatistik 2016}}). Feedbacks, Befragungen
und Curriculumsanalyse\footnote{\href{https://th-koeln.github.io/mi-2017/anhaenge/ba-pruefungsstatistiken.pdf}{Curriculumsanalye}}
zeigen, dass in diesem Semester die Anzahl der unterschiedlichen Module
am höchsten ist und viele Module Praxisanteile in Projektform haben,
sodass sich die Studierenden in verschiedene Fachdisziplinen,
Modulregularien und Projektkontexte eindenken und vielen
Teamkonstellationen organisieren müssen. Darüberhinaus sind im dritten
Semester bei vielen Modulen Prüfungsvorleistungen (Teilnahmeschein)
notwendig.

Ein weiteres Problem bildet offenbar das große Projekt im fünften
Semester (Entwicklungsprojekt interaktive Systeme). Nachdem die
Studierenden in den vorangegangen Semestern nur mit Projektgrößen von
maximal 2,5 Creditpoints konfrontiert wurden, stehen sie im fünften
Semester einem Projekt der vierfachen Größe gegenüber. Dies scheint
viele zu überfordern, so dass sie entweder erst dann das Projekt
beginnen, wenn sie keine parallelen Veranstaltungen haben, oder das
Projekt vorzeitig abbrechen.

Die Probleme beim Übergang ins Abschlusssemester wurden bereits
beschrieben. Vor allem in Feedbacks und persönlichen Gesprächen wird ein
weiteres Defizit häufig genannt: Die fehlende Möglichkeit sich in einem
thematischen Bereich zu vertiefen.

Somit lassen sich die folgenden Defizite im aktuellen Medieninformatik
Bachelor Studiengang zusammenfassen:

\begin{itemize}
\tightlist
\item
  Überladenes drittes Fachsemester
\item
  Zu viele Projektkontexte
\item
  Zu großer Sprung der Projektgrößen
\item
  Zu viele verschiedene Module mit unterschiedlichen Regularien
\item
  Zeitproblem beim Einstieg ins Praxisprojekt
\item
  Fehlende Möglichkeit zur strukturierten Vertiefung
\item
  Übergang in den Spezialisierungsteil (vom vierten ins fünfte Semester)
  übergang ins Abschlussprojekt
\item
  zu starke Fragmentierung, zu wenige Zusammenhänge
\item
  zu viele „Baustellen``
\item
  keine Spezialisierung, zu allgemein
\item
  zu wenige Übergänge in den Master
\item
  kein Medienrecht
\item
  zu wenig Kenntnisse über verschiedene Programmierkonzepte
\item
  missverständliches Abschlusssemester
\end{itemize}

\paragraph{Medieninformatik
Master\label{/mi-2017/selbstbericht/0100-ist-zustand/0100-ist-zustand}}\label{medieninformatik-masterpathlabelmi-2017selbstbericht0100-ist-zustand0100-ist-zustand}

Beim Medieninformatik Master leiten sich die erkannten Defizite im
Wesentlichen aus Feedbacks und persönlichen Gesprächen mit Studierenden,
Bachelor Absolventen und Dozenten ab. Die fehlende Möglichkeit zur
fachlichen Vertiefung und der geringe Anteil an praxisnahen Projekten
werden als wesentliche Defizite wahrgenommen und führen schlussendlich
auch dazu, dass viele potenzielle Studieninteressierte an andere
Studiengänge, zumeist außerhalb der TH Köln, mit stärkerer Profilierung
und Praxisbezug verloren gehen.

Der Medieninformatik Master sieht derzeit zwar verschiedene Wahlmodule
vor, diese sind aber stark fragmentiert und reglementiert, so dass hier
häufig keine echte Wahl durch die Studierenden getroffen werden kann.
Hinzu kommt, dass sich die Informatik Masterstudiengänge der Fakultät 10
sehr stark auseinander entwickelt haben, sodass Synergien, auch bei den
angebotenen Wahlpflichtfächern und Projekten, nur schwer genutzt werden
können.

Somit lassen sich die folgenden Defizite im aktuellen Medieninformatik
Master Studiengang zusammenfassen:

\begin{itemize}
\tightlist
\item
  zu wenig Übergänge von Bachelorabsolventen des Campus Gummersbach
\item
  fehlende Profilschärfung und zu wenig sichbarer Praxisbezug
\item
  geringer Anteil an praxisnahen Projekten
\item
  geringe internationale Ausrichtung
\item
  fehlende Möglichkeiten zur fachlichen Vertiefung
\item
  zu wenig Wahlmöglichkeiten
\end{itemize}

\chapter{Soll-Zustand/ geplante
Veränderungen\label{/mi-2017/selbstbericht/0150-soll-zustand-geplante-veraenderungen/0000-geplante-veraenderungen-bachelor}}\label{soll-zustand-geplante-veruxe4nderungenpathlabelmi-2017selbstbericht0150-soll-zustand-geplante-veraenderungen0000-geplante-veraenderungen-bachelor}

Seit der Reakkreditierung der Medieninformatik Studiengänge vor sieben
Jahren haben sich sowohl die Berufsbilder der Absolventinnen und
Absolventen als auch der Diskurs über Curricula der Medieninformatik
weiterentwickelt. Wir sehen vor allem drei Felder, in denen die
Perspektive der Medieninformatik erhebliche Relevanz erlangt hat: - die
Modelle und Methoden der Mensch-Computer-Interaktion (MCI) haben sich
nicht zuletzt in der entsprechenden Fachgruppe der Gesellschaft für
Informatik (GI) als ein konstituierendes Element des Gebiets
``Medieninformatik'' herauskristallisiert. - die Entwicklung von
Anwendungen im und für das Web hat sich als ein zunehmend eigenständiges
Feld der Systementwicklung etabliert. Während die Wirtschaftsinformatik
und die allgemeine Informatik das Web als Plattform für
Geschäftsprozesse bzw. als technische Plattform in den Vordergrund
stellen, steht in der Medieninformatik das Web selbst als
gesellschaftliches und ökonomisches Phänomen im Vordergrund, das durch
die Vernetzung von Personen, Diensten, Daten und Dingen (``things'')
neue technische, soziale und ökonomische Qualitäten hervorruft. -
digitale Medien wie audiovisuelle Medien, Visualisierungen oder
virtuelle oder angereicherte Welten (``virtual and augmented
realities'') haben als eines der konstituierenden Felder der
Medieninformatik in den letzten Jahren stark an Bedeutung gewonnen. In
diesen drei Feldern beobachten wir auch, dass ein signifikanter Anteil
unserer Absolventinnen und Absolventen ihre berufliche Zukunft sucht.

Darüber hinaus hat das Institut für Informatik beschlossen, als
Zukunftsfeld den Bereich ``Social Computing'' zu gestalten. Hierunter
werden Methoden, Techniken und Modelle für den Einsatz von in der Regel
Web basierten IT Systemen im gesellschaftlicen Kontext subsummiert, also
etwa Lernsysteme, Assistenzsysteme oder auch Systeme für die
gesellschaftliche und politische Teilhabe. Dieses Feld wird zunächst als
Teil der Studiengänge der Medieninformatik aufgebaut.

\section{Weiterentwicklung des Lehrportfolios des Institut für
Informatik\label{/mi-2017/selbstbericht/0150-soll-zustand-geplante-veraenderungen/0000-geplante-veraenderungen-bachelor}}\label{weiterentwicklung-des-lehrportfolios-des-institut-fuxfcr-informatikpathlabelmi-2017selbstbericht0150-soll-zustand-geplante-veraenderungen0000-geplante-veraenderungen-bachelor}

Im Antrag für die ``Anstehende Wiederzuweisung von fünf Professuren im
Institut für Informatik''\footnote{\href{https://th-koeln.github.io/mi-2017/anhaenge/inst-AntragWiederzuweisung_Motivation_2013.pdf}{Wiederzuweisungsantrag
  des Instituts für Informatik (2013)}} wird der Lehr- und
Forschungsbereich ``Soziotechnische Systeme'' wie folgt argumentiert:

\begin{siderules}
Die Informatik als wissenschaftliche Disziplin allgemein und auch das
Institut für Informatik im Besonderen muss sich der Tatsache stellen,
dass der technische und wissenschaftliche Fortschritt auf diesem
Fachgebiet nach wie vor rasant ist. Dem versucht das Institut nicht nur
durch stetige inhaltliche Weiterentwicklung seiner Studiengänge und
Module gerecht zu werden, sondern hier werden auch neue, auf
Nachhaltigkeit ausgerichtete Entwicklungen wahrgenommen und bei
ausreichender Relevanz in die Überlegungen zur Weiterentwicklung des
Angebots einbezogen. Eine solche -- aus Sicht des Instituts für
Informatik -- besonders interessante, spannende, wichtige und
zukunftsträchtige Entwicklung zeigt sich aktuell im Themenfeld
Soziotechnischer Systeme. Folgende Bereiche sind hier exemplarisch zu
nennen: - Assistenzsysteme für Tätigkeiten: Sportliches Training,
Navigation, Robotik (Staubsaugen, Lebensrettung, schwierige Umgebungen),
Autofahren, Büro, Produktionsumgebungen - Assistenzsysteme für
Bevölkerungsgruppen: alte Menschen, behinderte Menschen, Kinder,
Menschen im Alltag - Kommunikationshilfen: Suchmaschinen, soziale
Medien, Sprach-Ein- und Ausgabe, Blindenschrift-Displays und andere
spezielle Formen der MCI

In all diesen Bereichen zeichnet sich bereits heute eine große, künftig
noch stark zunehmende Bedeutung von IT-Systemen ab, zu deren Funktionen
nicht nur technisches Wissen, sondern auch Fach- und Methodenwissen aus
unterschiedlichen Spezialgebieten innerhalb der Informatik benötigt
wird. Um diesen neuen Entwicklungen gerecht zu werden, soll zunächst ein
geringer Anteil der Kapazität für dieses neue Themenfeld zur Verfügung
gestellt werden. Entsprechende Lehrveranstaltungen können und sollen
dann zunächst in Form von Wahlpflichtfächern angeboten werden. Die
entsprechenden Voraussetzung im Modulkatalog werden, sofern
erforderlich, kurzfristig geschaffen. Pflichtfächer werden dadurch nicht
beeinträchtigt.

Das Themenfeld „IT für Menschen`` wird auf absehbare Zeit als attraktiv
angesehen, sowohl für Forscher als auch für Unternehmen und nicht
zuletzt vor allem auch für Studieninteressierte. Wegen der nicht nur
technischen Ausrichtung ist aufgrund bisheriger Erfahrungen auch ein
signifikanter Anteil weiblicher Studierender zu erwarten. Das Themenfeld
bietet darüber hinaus viele Anknüpfungsmöglichkeiten an bereits
vorhandene Kompetenzen: Datenbanken, Medieninformatik, Softwaretechnik,
Ergonomie, MCI, Kommunikationstechnik, technische Spezialthemen, mobile
Systeme und Anwendungen u. v. m.

Die starke Durchdringung der Gesellschaft mit leistungsfähigen,
zunehmend mobilen, mit umfangreicher Sensorik ausgestatteten Endgeräten,
eröffnet auch hier teilweise völlig neue Fragestellungen und
Möglichkeiten. In Kombinationen mit den bestehenden alten und anderen
neuen Schwerpunkten eröffnet der Studienbereich „IT für Menschen`` auch
ein neues Forschungsfeld. Mittelfristiges Ziel ist es, hier ein neues
Studienangebot zu realisieren, dass auch aus den vom Präsidium für
solche Zwecke in Aussicht gestellten neuen Professuren gespeist wird und
nicht zu Lasten vorhandener Ressourcen -- weder in der Lehreinheit
Informatik noch in der Lehreinheit Ingenieurwissenschaften -- geht. Das
benötigte Know-how ist zum großen Teil bereits vorhanden und soll durch
Wahlpflichtangebote in diesem Bereich ergänzt werden.
\end{siderules}

Der Lehr- und Forschungsbereich ``Soziotechnische Systeme'' findet sich
in den zu akkreditierenden Curricula unter dem Begriff ``Social
Computing''. Dieser Themenkomplex soll im Bachelor Studienprogramm als
Vertiefungsmodul und im Master als Schwerpunkt verankert werden.

\section{Geplante Veränderungen des Bachelorstudiengangs gegenüber
dem aktuellen
Akkreditierungszeitraum\label{/mi-2017/selbstbericht/0150-soll-zustand-geplante-veraenderungen/0000-geplante-veraenderungen-bachelor}}\label{geplante-veruxe4nderungen-des-bachelorstudiengangs-gegenuxfcber-dem-aktuellen-akkreditierungszeitraumpathlabelmi-2017selbstbericht0150-soll-zustand-geplante-veraenderungen0000-geplante-veraenderungen-bachelor}

Die im Folgenden dargestellten geplanten Veränderungen des
Bachelorstudienprogramms dienen zur Beseitigung erkannter Schwächen
(vgl. Defizite Medieninformatik Bachelor).

\subsection{Verbesserungen des
Studienaufbaus\label{/mi-2017/selbstbericht/0150-soll-zustand-geplante-veraenderungen/0000-geplante-veraenderungen-bachelor}}\label{verbesserungen-des-studienaufbauspathlabelmi-2017selbstbericht0150-soll-zustand-geplante-veraenderungen0000-geplante-veraenderungen-bachelor}

\begin{figure}[htbp]
\centering
\includegraphics[width=\columnwidth]{../anhaenge/bilder/ba-veraenderungen-studienverlaufsplan.png}
\caption{Geplante Veränderungen des Bachelorstudiengangs
Medieninformatik. Links das aktuelle und rechts das zu akkreditierende
Curriculum. Die lila hinterlegten Module werden gestrichen, die grün
hinterlegten in Vertiefungsmodulen zusammengefasst, die orange
hinterlegten Module wurden neu angeordnet und die gelben Module wurden
neu integriert.}
\end{figure}

Mit einer Verbesserung des Studienaufbaus sollen folgende bekannte
Defizite ausgeglichen werden:

\begin{itemize}
\tightlist
\item
  Überladenes drittes Fachsemester
\item
  zu viele Projektkontexte
\item
  zu starke Fragmentierung von Modulen und der projektorientierten
  Praxisanteile
\item
  zu viele „Baustellen``
\end{itemize}

Die starke Projektorientierung wird und wurde insgesamt als positiv
bewertet. Jedoch ist die Verteilung der Module mit Projektanteil derzeit
nicht optimal. So sind z.B. im dritten Fachsemester momentan 7 Module
angesiedelt, von denen vier projektorientiert durchgeführt werden.
Hingegen wird im vierten Semester kein projektorientiertes Modul
angeboten. Um hier die Aufwände gleichmäßiger zu verteilen, wurde die
Reihenfolge der Module verändert und Module wurden zusammengelegt.

Beim Studienaufbau wurde versucht die Modulreihenfolge auf einen groben
Workflow der Softwareentwicklung auszurichten. Die prägnanteste
Auswirkung dieser Maßnahme ist die Verschiebung des Moduls
``Mensch-Computer Interaktion'' vom vierten ins zweite Semester, um die
Studierenden hier frühzeitig mit konzeptionellen Problemen und
Fragestellungen wie Tätigkeitsmodellierung oder die Spezifikation von
Anforderungen zu konfrontieren und ihnen hierzu entsprechende
Grundkenntnisse und Fertigkeiten zu vermitteln, die in den späteren
Projekten angewendet und eingeübt werden können.

Im vierten Semester wurde ein Vertiefungsmodul mit 20 Creditpoints
installiert auf das später noch weiter eingegangen wird. Bezogen auf den
Studienaufbau wird hierdurch die Fragmentierung und die vielen
Projektkontexte, die schlechtestenfalls mit vielen kleinen Modulen
einhergeht, deutlich reduziert.

\subsection{Verbesserter Aufbau der projektorientierten Module und
der
Projektgrößen\label{/mi-2017/selbstbericht/0150-soll-zustand-geplante-veraenderungen/0000-geplante-veraenderungen-bachelor}}\label{verbesserter-aufbau-der-projektorientierten-module-und-der-projektgruxf6uxdfenpathlabelmi-2017selbstbericht0150-soll-zustand-geplante-veraenderungen0000-geplante-veraenderungen-bachelor}

\begin{figure}[htbp]
\centering
\includegraphics[width=\columnwidth]{../anhaenge/bilder/ba-projektanteile.png}
\caption{Veränderter Aufbau der Projektanteile des Bachelorstudiengangs
Medieninformatik. Links das aktuelle und rechts das zu akkreditierende
Curriculum.}
\end{figure}

Hiermit sollen folgende bekannte Defizite ausgeglichen werden:

\begin{itemize}
\tightlist
\item
  zu viele Projektkontexte
\item
  zu großer Sprung der Projektgrößen
\item
  zu viele „Baustellen``
\end{itemize}

Wie bereits beschrieben, wurden die projektorientierten Module
gleichmäßiger über den Studienverlauf verteilt und projektorientierte
Module teilweise zusammen gelegt. Um die Projektgrößen sinnvoll
aufzubauen, werden jetzt in den ersten drei Semestern Projekte mit einem
Gewicht von max. 2,5 Creditpoints absolviert. Im vierten Semester folgt
dann, als Teil des Vertiefungsmoduls, ein Projekt mit einem Gewicht von
etwa 5 Creditpoints. Im fünften Semester folgt dann das
Entwicklungsprojekt mit einem Gewicht von 10 Creditpoints. Im sechsten
Semester liegt dann das Praxisprojekt mit ebenfalls 10 Creditpoints und
die Bachelorarbeit mit 12 Creditpoints. Für diejenigen, die dann in den
Masterstudiengang wechseln wollen, bleibt die Projektgröße bei 12
Creditpoints.

\subsection{Strukturierte Möglichkeit zur individuellen
Fachvertiefung\label{/mi-2017/selbstbericht/0150-soll-zustand-geplante-veraenderungen/0000-geplante-veraenderungen-bachelor}}\label{strukturierte-muxf6glichkeit-zur-individuellen-fachvertiefungpathlabelmi-2017selbstbericht0150-soll-zustand-geplante-veraenderungen0000-geplante-veraenderungen-bachelor}

\begin{figure}[htbp]
\centering
\includegraphics[width=\columnwidth]{../anhaenge/bilder/ba-vertiefungen.png}
\caption{Zusammenfassung von Modulen aus einem Themenfeld zu
Vertiefungsmodulen im Medieninformatik Bachelor.}
\end{figure}

Mit dieser Änderungen sollen folgende bekannte Defizite ausgeglichen
werden:

\begin{itemize}
\tightlist
\item
  keine Spezialisierung, zu allgemein
\item
  zu viele verschiedene Module mit unterschiedlichen Regularien
\item
  fehlende Möglichkeit zur strukturierten Vertiefung
\item
  zu viele Projektkontexte
\item
  zu großer Sprung der Projektgrößen
\item
  zu viele „Baustellen``
\item
  zu starke Fragmentierung von Modulen und der projektorientierten
  Praxisanteile
\item
  Übergang in den Spezialisierungsteil vom vierten ins fünfte
  Fachsemester
\item
  Übergang ins Abschlussprojekt
\item
  Zeitproblem beim Einstieg ins Praxisprojekt
\item
  zu wenige Outgoings
\end{itemize}

Im vierten Semester wird ein Vertiefungsmodul mit einem Gewicht von 20
Creditpoints installiert. Hier stehen drei Vertiefungsrichtungen zur
Verfügung: Visual Computing, Social Computing und Web-Development. Im
Modul ist ein Projektanteil von etwa fünf Creditpoints vorgesehen. Mit
Hilfe des Vertiefungsmoduls werden eine Reihe von Schwächen
ausgeglichen. Die Studierenden haben hier die Möglichkeit tief in ein
Themenfeld einzudringen und darüber ggf. eine Spezialisierungsrichtung
einzuschlagen. Durch das Zusammenfassen mehrerer Module werden im
Vertiefungsmodul konsistente Regularien, sowie inhaltliche und
organisatorische Zusammenhänge geschaffen. Die Projektkontexte und die
inhaltlichen Perspektiven der Projekte werden reduziert. Der Übergang in
den Spezialisierungsabschnitt des Studiums im fünften Semester wird
idealerweise erleichtert, da durch die Wahl des Vertiefungsmoduls in
vielen Fällen schon eine Spezialisierungsrichtung vorgegeben ist.

Das Entwicklungsprojekt im fünften Semester wird inhaltlich geöffnet. Im
aktuellen Akkreditierungszeitraum war dieses Projekt fest an die
inhaltlichen Perspektiven ``Mensch-Computer Interaktion'' und
``Verteilte Anwendungen'' gebunden. Durch die inhaltliche Öffnungen
können die Studierenden jetzt ihre fachliche Vertiefung entsprechend
ihren Neigungen wählen. Damit geht die freie Wahl der betreuenden
Professoren einher. Idealerweise hat sich durch dieses Projekt und das
vorangegangene Vertiefungsmodul bereits eine inhaltliche und
organisatorische Zusammenarbeit gefestigt, die den Übergang ins
Abschlussprojekt deutlich erleichtert.

Das vierte Semester eignet sich jetzt aus verschiedenen Gründen gut für
ein Auslandssemester. Die Studierenden verfügen über ausreichende
Qualifikationen und Projekterfahrungen, um in verschiedenen Kontexten
handlungsfähig zu sein. Sie stehen aber noch vor dem
Spezialisierungsteil des Studiums und verfügen damit idealerweise über
die fachliche, mentale und organisatorische Offenheit für ein
Austauschsemester. Ob der wenigen Module im vierten Semester und vor
allem wegen des großen Vertiefungsmoduls können im Ausland erworbene
Qualifikationen sehr flexibel anerkannt werden. Die Anerkennung erfolgt
auf Basis des ``Übereinkommen über die Anerkennung von Qualifikationen
im Hochschulbereich in der europäischen Region'' \footnote{\href{https://de.wikipedia.org/wiki/\%C3\%9Cbereinkommen_\%C3\%BCber_die_Anerkennung_von_Qualifikationen_im_Hochschulbereich_in_der_europ\%C3\%A4ischen_Region}{Erläuterung
  zum Übereinkommen über die Anerkennung von Qualifikationen im
  Hochschulbereich in der europäischen Region}}.

\subsection{Weitere
Änderungen\label{/mi-2017/selbstbericht/0150-soll-zustand-geplante-veraenderungen/0000-geplante-veraenderungen-bachelor}}\label{weitere-uxe4nderungenpathlabelmi-2017selbstbericht0150-soll-zustand-geplante-veraenderungen0000-geplante-veraenderungen-bachelor}

Darüber hinaus wurden weitere Änderungen durchgeführt, um die folgenden
Defizite zu verbessern:

\begin{itemize}
\tightlist
\item
  kein Medienrecht
\item
  missverständliches Abschlusssemester
\item
  zu wenig Kenntnisse über verschiedene Programmierkonzepte
\end{itemize}

Das Abschlusssemester wurde in seiner grundsätzlichen Struktur
beibehalten, jedoch wurde der Seminarteil des Moduls ``Praxisprojekt''
ausgelagert und als eigenes Modul installiert. Hiermit wird die
Studierbarkeit verbessert, da die zeitliche Kopplung des Praxis- und
Seminarteils reduziert wird. Darüber hinaus ist nun das Praxisprojekt
mit einem Gewicht von 10 Creditpoints ausgestattet und damit weniger
gewichtig, als die Bachelorarbeit, die ein Gewicht von 12 Creditpoints
hat.

Das Modul ``Paradigmen der Programmierung'', das bislang nur im
Informatik Bachelor als Pflichtmodul im Curriculum verankert war, wird
jetzt ein Pflichtmodul im dritten Fachsemester in der Medieninformatik
um die Studierenden bessere Kenntnisse im Bereich verschiedener
Programmierkonzepte und deren Anwendung zu vermitteln.

Da web-basierte Architekturen elementarer Bestandteil der
Medieninformatik und dediziert in den Studiengangszielen verankert sind,
wurde hierzu ein Pflichtmodul installiert, das Inhalte aus den Modulen
``Web-basierte Anwendungen 1 \& 2'' enthält, die alle Studierenden
kennen sollten, auch wenn sie sich später in einem anderen Bereich
vertiefen möchten.

Im fünften Semester wurde das Modul ``Medieninformatik und
Gesellschaft'' umgewidmet in ``Medienrecht, Medien und Gesellschaft'' um
hier einen dedizierten Platz für rechtliche Themen innerhalb der Domäne
im Curriculum zu verankern.

\section{Auswirkungen auf die
Lehrkapazität\label{/mi-2017/selbstbericht/0150-soll-zustand-geplante-veraenderungen/0000-geplante-veraenderungen-bachelor}}\label{auswirkungen-auf-die-lehrkapazituxe4tpathlabelmi-2017selbstbericht0150-soll-zustand-geplante-veraenderungen0000-geplante-veraenderungen-bachelor}

Die Änderungen im Bachelorstudienprogramm sind weitgehend
kapazitätsneutral. Das von allen Informatikstudiengängen geteilte Modul
``Betriebswirtschaftslehre 2'' wurde durch das ebenfalls geteilte Modul
``Paradigmen der Programmierung'' ersetzt. Das Modul ``Mensch-Computer
Interaktion'' wurde zwar von fünf auf zehn Creditpoints vergrößert,
jedoch enthielt bislang das Modul ``Entwicklungsprojekt Interaktive
Systeme'' fünf Creditpoints Praxisanteil ``Mensch-Computer Interaktion''
die nun direkt dem Modul zugeschlagen werden.

Die Studierenden verteilen sich über die Vertiefungsmodule, in die auch
bisherige Wahlpflichtmodule integriert wurden. Somit reduziert sich hier
in Summe die Lehrbelastung. Durch die resultierende Reduktion ist es
möglich das Modul ``Web-Architekturen'' kapazitätsneutral anzubieten.
Das ``Entwicklungsprojekt'' im fünften Semester ist zukünftig nicht mehr
an nur zwei Lehrende gebunden, sondern kann von allen Lehrenden der
Informatik betreut werden. Dadurch verteilt sich die Lehrbelastung.

Lediglich das Vertiefungsmodul ``Social Computing'' ist nicht
kapazitätsneutral, hier wurde jedoch zusätzliche Kapazität aufgebaut.

Das Personalcontrolling der TH-Köln bestätigt die ausreichende
Lehrkapazität\footnote{\href{https://th-koeln.github.io/mi-2017/anhaenge/th-verwaltung-kapa-nachweis.pdf}{Nachweis
  über ausreichende Lehrkapazität durch das Team
  7.3(Personalcontrolling) der TH Köln}} für das vorliegende Konzept des
Medieninformatik Master Studiengangs.

\section{Geplante Veränderungen des Master-Studiengangs gegenüber dem
aktuellen
Akkreditierungszeitraum\label{/mi-2017/selbstbericht/0150-soll-zustand-geplante-veraenderungen/0000-geplante-veraenderungen-bachelor}}\label{geplante-veruxe4nderungen-des-master-studiengangs-gegenuxfcber-dem-aktuellen-akkreditierungszeitraumpathlabelmi-2017selbstbericht0150-soll-zustand-geplante-veraenderungen0000-geplante-veraenderungen-bachelor}

Die im Folgenden dargestellten geplanten Veränderungen des
Masterstudienprogramms dienen zur Beseitigung erkannter Schwächen (vgl.
Defizite Medieninformatik Master).

\subsection{Einheitliches Modulraster 6
CP\label{/mi-2017/selbstbericht/0150-soll-zustand-geplante-veraenderungen/0000-geplante-veraenderungen-bachelor}}\label{einheitliches-modulraster-6-cppathlabelmi-2017selbstbericht0150-soll-zustand-geplante-veraenderungen0000-geplante-veraenderungen-bachelor}

Grundsätzlich wurde die Basisgröße der Module von fünf auf sechs
Creditpoints erhöht. Module haben also stets ein Gewicht von sechs
Creditpoints oder einem Vielfachen davon. Dies spiegelt hauptsächlich
den für viele Module auf Master-Niveau erhöhten Selbststudienanteil
wider. Darüber hinaus werden durch diese Maßnahme auch strukturelle bzw.
organisatorische Schwächen beseitigt. Zum einen werden auch im Master
die einzelnen Fachsemester weniger stark fragmentiert, zum anderen
werden Synergien zu dem Masterstudiengang \emph{Informatik} ermöglicht,
der ebenfalls am Campus Gummersbach angeboten wird, indem Module und
Projekte studiengangsübergreifend angeboten werden können.

\subsection{Schärfung des
Profils\label{/mi-2017/selbstbericht/0150-soll-zustand-geplante-veraenderungen/0000-geplante-veraenderungen-bachelor}}\label{schuxe4rfung-des-profilspathlabelmi-2017selbstbericht0150-soll-zustand-geplante-veraenderungen0000-geplante-veraenderungen-bachelor}

Mit der Profilschärfung sollen folgende bekannte Defizite ausgeglichen
werden:

\begin{itemize}
\tightlist
\item
  zu wenig Übergänge von Bachelorabsolventen der eigenen Fakultät
\item
  Fehlende Profilschärfung und sichtbarer Praxisbezug
\item
  Fehlende Möglichkeiten zur fachlichen Vertiefung
\end{itemize}

Der Medieninformatik Masterstudiengang erschien bislang eher
generalistisch geprägt. Im Zuge der Reakkreditierung soll den
Studierenden die Möglichkeit gegeben werden, sich in bestimmten
Bereichen zu spezialisieren. Dafür werden bestimmte Module in
Studienschwerpunkte geclustert.

Den gemeinsamen Rahmen des zukünftigen Curriculums bilden drei
Kern-Module und drei Projekte. Die Kern-Module sind

\begin{itemize}
\tightlist
\item
  Spezielle Gebiete der Mathematik,
\item
  Research Methods,
\item
  Computerethik.
\end{itemize}

Die drei Projekte widmen sich jeweils einem für Projekte der
Medieninformatik relevanten übergeordneten Fragestellung:

\begin{itemize}
\tightlist
\item
  Projekt 1: Vision und Konzeption
\item
  Projekt 2: Entwicklung
\item
  Projekt 3: Forschung, Evaluation/Assessment, Verwertung
\end{itemize}

Aufbauend auf den thematischen Gebieten des Bachelorstudiengangs, die
sich dort unter anderem in den Vertiefungsmodulen manifestieren, und
ausgerichtet auf die oben skizzierte Fortentwicklung des Fachgebietes
\emph{Medieninformatik} sowie die Strategie des Instituts für
Informatik, werden folgende Schwerpunkte angeboten: ``Social
Computing'', ``Visual Computing'', ``Weaving the Web'' und
``Human-Computer Interaction''. Einen wie bislang eher generalistischen
Studienverlauf ermöglicht der Pseudo-Schwerpunkt ``Multiperspective
Product Development'', der sich aus ausgewählten Modulen der anderen
Schwerpunkte und des Wahlpflichtkatalogs speist. Die Module der
Schwerpunkte sind sich zum großen Teil eine Fortführung bestehender
Module des Pflicht- und Wahlbereichs des aktuellen Curriculums. Für die
Schwerpunkte ``Visual Computing'' und ``Social Computing'' wurden einige
neue Module erarbeitet, da hier, in Einklang mit der inhaltlichen
Strategie des Instituts, neue Themengebiete erschlossen oder verbreitert
werden sollen.

Die Schwerpunkte setzen sich aus den folgenden Modulen zusammen:

\paragraph{Schwerpunkt Social
Computing\label{/mi-2017/selbstbericht/0150-soll-zustand-geplante-veraenderungen/0000-geplante-veraenderungen-bachelor}}\label{schwerpunkt-social-computingpathlabelmi-2017selbstbericht0150-soll-zustand-geplante-veraenderungen0000-geplante-veraenderungen-bachelor}

\begin{itemize}
\tightlist
\item
  Sicherheit, Privatsphäre und Vertrauen im Netz
\item
  Soziotechnische Entwurfsmuster
\item
  Netzwerk-und Graphentheorie
\end{itemize}

\paragraph{Schwerpunkt Visual
Computing\label{/mi-2017/selbstbericht/0150-soll-zustand-geplante-veraenderungen/0000-geplante-veraenderungen-bachelor}}\label{schwerpunkt-visual-computingpathlabelmi-2017selbstbericht0150-soll-zustand-geplante-veraenderungen0000-geplante-veraenderungen-bachelor}

\begin{itemize}
\tightlist
\item
  Storytelling und Narrative Strukturen
\item
  Bildbasierte Computergrafik
\item
  Visualisierung
\end{itemize}

\paragraph{Schwerpunkt Human-Computer
Interaction\label{/mi-2017/selbstbericht/0150-soll-zustand-geplante-veraenderungen/0000-geplante-veraenderungen-bachelor}}\label{schwerpunkt-human-computer-interactionpathlabelmi-2017selbstbericht0150-soll-zustand-geplante-veraenderungen0000-geplante-veraenderungen-bachelor}

\begin{itemize}
\tightlist
\item
  Interaction Design
\item
  Design Methodologies
\item
  Angewandte Statistik für die Mensch-Computer Interaktion
\end{itemize}

\paragraph{Schwerpunkt Weaving the
Web\label{/mi-2017/selbstbericht/0150-soll-zustand-geplante-veraenderungen/0000-geplante-veraenderungen-bachelor}}\label{schwerpunkt-weaving-the-webpathlabelmi-2017selbstbericht0150-soll-zustand-geplante-veraenderungen0000-geplante-veraenderungen-bachelor}

\begin{itemize}
\tightlist
\item
  Sicherheit, Privatsphäre und Vertrauen im Netz
\item
  Web Architekturen
\item
  Web Technologien
\end{itemize}

\paragraph{Generalistischer Studienverlauf: Multiperspective Product
Development\label{/mi-2017/selbstbericht/0150-soll-zustand-geplante-veraenderungen/0000-geplante-veraenderungen-bachelor}}\label{generalistischer-studienverlauf-multiperspective-product-developmentpathlabelmi-2017selbstbericht0150-soll-zustand-geplante-veraenderungen0000-geplante-veraenderungen-bachelor}

\begin{itemize}
\tightlist
\item
  Sicherheit, Privatsphäre und Vertrauen im Netz
\item
  Interaction Design
\item
  Qualitätssicherung für Web-Anwendungen
\end{itemize}

\subsection{Erhöhung des Anteils an praxisnahen
Projekten\label{/mi-2017/selbstbericht/0150-soll-zustand-geplante-veraenderungen/0000-geplante-veraenderungen-bachelor}}\label{erhuxf6hung-des-anteils-an-praxisnahen-projektenpathlabelmi-2017selbstbericht0150-soll-zustand-geplante-veraenderungen0000-geplante-veraenderungen-bachelor}

Mit dieser Veränderung soll folgende bekannte Schwäche ausgeglichen
werden:

\begin{itemize}
\tightlist
\item
  Geringer Anteil an praxisnahen Projekten
\end{itemize}

Um dieses Defizit auszugleichen wird zukünftig in jedem der ersten drei
Fachsemester ein Projekt mit einem Gewicht von 12 Creditpoints
angeboten. Der Projektanteil wird damit von 10 Creditpoints auf 36
Creditpoints erhöht und auf alle Studiensemester verteilt, sodass der
Übergang ins Berufsleben und die Kooperation mit Unternehmen verbessert
werden können. Über jedem der Fachsemester steht eine übergeordnete
Fragestellung und die Projekte zahlen auf diese Fragestellung ein. In
den Projekten werden Fragestellungen und Probleme aus Sicht der
jeweiligen Schwerpunkte bearbeitet. Je nach Projektgegenstand können und
sollen Projektteams aus verschiedenen Schwerpunkten zusammenarbeiten und
die Perspektive ihres jeweiligen Studienschwerpunkts vertreten. Im
Rahmen des jeweiligen Projekts werden auch Workshops und
Lehrveranstaltungen zu verschiedenen Themen angeboten, z.B.
Projektmanagement.

Die Mitarbeit der Studierenden in Projekten trägt überdies zum Ausbau
der Forschungsaktivitäten der Fakultät bei. Die Projekte im Master
setzen so die Projektorientierung aus dem Bachelorstudiengang konsequent
fort. Über die Möglichkeit der Schwerpunkt-gemischten Teams werden
multiperspektivische Lösungsansätze und Fachdiskurse forciert. Somit
bilden die Projekte einen wesentlichen Bestandteil bei der Erreichung
der angestrebten Kompetenzziele des Studiengangs.

\subsection{Flexibilisierung des dritten
Fachsemesters\label{/mi-2017/selbstbericht/0150-soll-zustand-geplante-veraenderungen/0000-geplante-veraenderungen-bachelor}}\label{flexibilisierung-des-dritten-fachsemesterspathlabelmi-2017selbstbericht0150-soll-zustand-geplante-veraenderungen0000-geplante-veraenderungen-bachelor}

\begin{itemize}
\tightlist
\item
  Geringe internationale Ausrichtung
\item
  zu wenig Wahlmöglichkeiten
\end{itemize}

Im dritten Fachsemester sind neben dem Projekt drei Wahlmodule
vorgesehen, die - im Gegensatz zum bisherigen Curriculum - an keinerlei
weitere Regularien gebunden sind. Im aktuellen Curriculum werden die
Wahlpflichtmodule in vier Kategorien eingeteilt und pro Kategorie muss
ein Modul belegt werden. Die Kategorisierung der Wahlmodule führt jedoch
nicht selten dazu, dass den Studierenden keine oder nur sehr wenige
Optionen offen stehen. Im neuen Curriculum können als Wahlmodule alle
Module des Medieninformatik- und des Informatik-Masterstudiengangs
gewählt werden.

Durch die offene Gestaltung des dritten Fachsemesters eignet sich
selbiges gut für ein Auslandssemester, da hier die Anerkennung von
Modulen sehr leicht fallen sollte. Die Anerkennung der im Ausland
erworbenen Qualifikationen erfolgt auf Basis des ``Übereinkommen über
die Anerkennung von Qualifikationen im Hochschulbereich in der
europäischen Region'' \footnote{\href{https://de.wikipedia.org/wiki/\%C3\%9Cbereinkommen_\%C3\%BCber_die_Anerkennung_von_Qualifikationen_im_Hochschulbereich_in_der_europ\%C3\%A4ischen_Region}{Erläuterung
  zum Übereinkommen über die Anerkennung von Qualifikationen im
  Hochschulbereich in der europäischen Region}}.

\subsection{Auswirkungen auf die
Lehrkapazität\label{/mi-2017/selbstbericht/0150-soll-zustand-geplante-veraenderungen/0000-geplante-veraenderungen-bachelor}}\label{auswirkungen-auf-die-lehrkapazituxe4tpathlabelmi-2017selbstbericht0150-soll-zustand-geplante-veraenderungen0000-geplante-veraenderungen-bachelor-1}

Wie oben dargelegt dient die Weiterentwicklung im Wesentlichen einer
Profilschärfung des Masterstudiengangs Medieninformatik. Die
eingeführten Studienschwerpunkte basieren dabei auf den bereits
vorhandenen Modulen des Studiengangs. Die neu hinzugekommenen Module
werden in der Regel von neuberufenen Professoren durchgeführt und wurden
z.T. bereits als Wahlpflichtfächer im bisherigen Curriculum angeboten.
Die Projekte werden Schwerpunkt- und Studiengangsübergreifend angeboten
und können von allen Professorinnen und Professoren des Instituts für
Informatik betreut werden. Die Auswirkungen auf die Lehrkapazität sind
auch hier in einem vertretbaren Rahmen von zusätzlich ca. 25 SWS und
werden größtenteils durch den bereits erfolgten Kapazitätsaufbau des
Instituts für Informatik der Fakultät 10 sowie zum kleinen Teil durch
Lehraufträge aufgefangen.

In der folgenden Abbildung sind im linken Teil semesterweise die
Pflichtmodule des aktuellen Curriculums dargestellt. Der rechte Teil
zeigt die Module des zukünftigen Curriculums, wobei aus Gründen der
Übersichtlichkeit keine Aufteilung nach Semestern vorgenommen wurde.
Blau gefärbt sind die Kern-Module, Magenta die mindestens einem
Schwerpunkt zugeordneten Module, Grün die Projektmodule und grau die
Wahlpflich-Module. Die Pfeile zeigen die Weiterführung der Module des
aktuellen Curriculums an.

\begin{figure}[htbp]
\centering
\includegraphics[width=\columnwidth]{../anhaenge/bilder/ma-mpo3-mpo4.png}
\caption{Module des Masterstudiengangs Medieninformatik. Links das
aktuelle und rechts das zukünftige Curriculum.}
\end{figure}

Das Personalcontrolling der TH-Köln bestätigt die ausreichende
Lehrkapazität\footnote{\href{https://th-koeln.github.io/mi-2017/anhaenge/th-verwaltung-kapa-nachweis.pdf}{Nachweis
  über ausreichende Lehrkapazität durch das Team
  7.3(Personalcontrolling) der TH Köln}} für das vorliegende Konzept des
Medieninformatik Master Studiengangs.

\chapter{Qualifikationsziele der
Studiengangskonzepte\label{/mi-2017/selbstbericht/0200-qualifikationsziele/0000-qualifikationsziele}}\label{qualifikationsziele-der-studiengangskonzeptepathlabelmi-2017selbstbericht0200-qualifikationsziele0000-qualifikationsziele}

Die Fachgruppe Medieninformatik innerhalb des Fachbereichs
Mensch-Computer-Interaktion der Gesellschaft für Informatik definiert
das Lehr-, Forschungs- und Arbeitsgebiet Medieninformatik in ihrem
aktuellen Positionspapier\footnote{Martin Christof Kindsmüller,
  Christian Wolters, Andreas M. Heinecke:
  \href{http://dl.mensch-und-computer.de/bitstream/handle/123456789/5131/Kindsm\%C3\%BCller_Wolters_Heinecke_2016.pdf}{Medieninformatik
  2016: Was war, was ist, was soll sein?}} wie folgt:

\begin{siderules}
Medieninformatik ist ein Teilgebiet der Informatik. Sie beschäftigt sich
mit: - der Analyse, Konzeption, Realisierung und Evaluation von
interaktiven und multimedialen Mensch-Computer-Systemen sowie Systemen
zur computer-mediierten multimedialen Mensch-Mensch-Kommunikation,

\begin{itemize}
\tightlist
\item
  Methoden und Werkzeugen zur Konzeption, Gestaltung, Produktion,
  Speicherung und Verteilung digitaler Medien sowie
\item
  Zielen, Anforderungen und Wirkungen digitaler Medien für Mensch,
  Umwelt und Gesellschaft.
\end{itemize}
\end{siderules}

Darüber hinaus sieht sie fünf charakteristische Merkmale: \textgreater{}
- Medieninformatik ist ein Teil der Informatik. \textgreater{} -
Medieninformatik produziert, distribuiert und präsentiert digitale
Medien. \textgreater{} - Medieninformatik entwickelt multimediale
Benutzungsschnittstellen interaktiver Medien. \textgreater{} -
Medieninformatik arbeitet interdisziplinär. \textgreater{} -
Medieninformatik arbeitet forschungs- und anwendungsorientiert.

Aus diesen Definitionen und den Erfahrungen der beteiligten
Studiengangsverantwortlichen lässt sich ableiten, dass Medieninformatik
ein anspruchsvolles, facettenreiches Betätigungsfeld mit ausgeprägter
Interdisziplinarität ist. Das breite Spektrum an erforderlichen
kognitiven, sozialen und fachlichen Kompetenzen, Fertigkeiten und
Kenntnissen lässt sich kaum mit der nötigen Tiefe in einem einzigen
Ausbildungsprofil zusammenführen. Mit zunehmender Komplexität der zu
entwickelnden Systeme und zunehmenden Anforderungen an die Qualität
dieser Systeme, aber auch aufgrund der wachsenden Bedeutung von Software
für innovative Produkte und Dienstleistungen in unserer Gesellschaft,
zeigt sich daher immer mehr die Notwendigkeit einer professionellen
Differenzierung. Um ein möglichst beständiges, von aktuellen
technologischen Trends weitgehend unabhängiges
Medieninformatik-Curriculum bieten zu können, orientieren sich die
Inhalte der Medieninformatik Studiengänge weitgehend an Grundlagen, ohne
jedoch den Praxisbezug in Form von Fallstudien und Projekten zu
vernachlässigen.

Die wesentliche Basis für die Entwicklung und Ausgestaltung relevanter
Kompetenzen bildeten einerseits, soweit sie noch Bestand haben, die
bestehenden Bereiche und Kompetenzziele das aktuellen Curriculums und
andererseits die aktuellen ``Empfehlungen für Bachelor- und
Masterprogramme im Studienfach Informatik an Hochschulen'' \footnote{Gesellschaft
  für Informatik e.V. (GI):
  \href{https://www.gi.de/fileadmin/redaktion/empfehlungen/GI-Empfehlungen_Bachelor-Master-Informatik2016.pdf}{Empfehlungen
  für Bachelor- und Masterprogramme im Studienfach Informatik an
  Hochschulen}} der Gesellschaft für Informatik. Hier werden folgende
Kompetenzbereiche vorgeschlagen und mit exemplarischen Inhalten
verknüpft:

\textbf{Formale, algorithmische, mathematische Kompetenzen:} Diskrete
Strukturen, Logik und Algebra, Analysis und Numerik,
Wahrscheinlichkeitstheorie und Statistik, Formale Sprachen und
Automaten, Modellierung, Algorithmen und Datenstrukturen

\textbf{Analyse-, Entwurfs-, Realisierungs- und
Projektmanagement-Kompetenzen:} Programmiersprachen und -methodik,
Software-Engineering, Mensch-Computer-Interaktion, Projekt- und
Teamkompetenz

\textbf{Technologische Kompetenzen:} Digitaltechnik und
Rechnerorganisation, Betriebssysteme, Datenbanken und
Informationssysteme, Rechnernetze und verteilte Systeme, IT-Sicherheit

\textbf{Fachübergreifende Kompetenzen:} Methodische und
wissenschaftliche Aspekte, Gesellschaftliche und berufsethische Aspekte
von Informatiksystemen im Anwendungskontext, ökonomische und ökologische
Aspekte von Informatiksystemen im Anwendungskontext, rechtliche Aspekte
von Informatiksystemen im Anwendungskontext

\textbf{Soziale Kompetenzen und Selbstkompetenzen:}
Kooperationsmanagement, Diversity- und Konfliktmanagement,
Organisationsentwicklung

\textbf{Methoden- und Transferkompetenz:} Strategien des Wissenserwerbs
und der wissenschaftlichen Weiterbildung, Analyse von Informatiksystemen
in ihrem Anwendungskontext, Implementierungs- und Evaluationsstrategien

Diese wurden für die Medieninformatik um den Kompetenzbereich
\textbf{Medienkompetenz} mit den Gebieten Medienrezeption,
Medienkonzeption, Medientechnik und Mediengestaltung ergänzt.

Medieninformatiker analysieren, konzipieren, realisieren und adaptieren
in interdisziplinären Teams oft web-basierte Prozesse und Systeme zur
Produktion, Bearbeitung und Distribution medienbasierter Informationen
aus informatischen, ökonomischen und sozialen Perspektiven; ggf.
betreiben sie diese Systeme auch.

\section{Kompetenzbereiche, Ziele und
Lernergebnisse\label{/mi-2017/selbstbericht/0200-qualifikationsziele/0000-qualifikationsziele}}\label{kompetenzbereiche-ziele-und-lernergebnissepathlabelmi-2017selbstbericht0200-qualifikationsziele0000-qualifikationsziele}

Zugehörig zu den Kompetenzbereichen wurden die folgenden Ziele und
Lernergebnisse abgeleitet, die als Basis für die Ausrichtung und
Einteilung der einzelnen Module in beiden Studiengängen verwendet
werden:

\subsection{Formale, algorithmische, mathematische
Kompetenzen\label{/mi-2017/selbstbericht/0200-qualifikationsziele/0000-qualifikationsziele}}\label{formale-algorithmische-mathematische-kompetenzenpathlabelmi-2017selbstbericht0200-qualifikationsziele0000-qualifikationsziele}

Die Studierenden \ldots{}

\begin{itemize}
\tightlist
\item
  können Probleme und Anforderungen exakt beschreiben, um diese in
  geeigneten Datenstrukturen und effizienten Algorithmen umzusetzen.
\item
  kennen Vorgehensweisen und Werkzeuge, um Probleme und Sachverhalte zu
  abstrahieren und zu modellieren (logische und algebraische Kalküle,
  graphentheoretische Notationen, formale Sprachen und Automaten sowie
  spezielle Kalküle wie Petri-Netze oder die Prozessalgebra CSP)
\item
  kennen Verfahrensweisen um den algorithmischen Kern eines Problems zu
  identifizieren und können Algorithmen entwerfen, verifizieren und
  bzgl. ihres Ressourcenbedarfs bewerten.
\item
  können Lösungen angemessen fachlich kommunizieren, bewerten und im
  Rahmen von kooperativen Arbeitszusammenhängen nutzen.
\end{itemize}

\subsection{Analyse-, Entwurfs-, Realisierungs- und
Projektmanagement-Kompetenzen\label{/mi-2017/selbstbericht/0200-qualifikationsziele/0000-qualifikationsziele}}\label{analyse--entwurfs--realisierungs--und-projektmanagement-kompetenzenpathlabelmi-2017selbstbericht0200-qualifikationsziele0000-qualifikationsziele}

Die Studierenden \ldots{}

\begin{itemize}
\tightlist
\item
  haben die Fähigkeit, mit Aufgabenstellern und zukünftigen
  Systemnutzern zu kommunizieren und zu kooperieren und sich schnell in
  neue Anwendungskontexte einzuarbeiten.
\item
  können bekannte Problemstellungen im Anwendungskontext erkennen und
  sind mit den zugehörigen Lösungsmustern vertraut.
\item
  erkennen Inkonsistenzen und können mit unklaren Anforderungen umgehen.
\item
  können komplexe Domänen modellieren und große Anwendungsprobleme durch
  geeignete Schnittstellen in Teilprobleme zerlegen.
\item
  haben solide Kenntnisse in der Software-Architektur um Systeme aus
  Hard- und Software zu konstruieren, welche die Anforderungen
  vollständig erfüllen.
\item
  können Mensch-Technik-Schnittstellen anwendungsgerecht und ergonomisch
  gestalten.
\item
  berücksichtigen beim Entwurf die Umsetzung nichtfunktionaler
  Anforderungen, wie Sicherheit, Performanz, Skalierbarkeit,
  Wartbarkeit, Erweiterbarkeit und Zuverlässigkeit.
\item
  beherrschen gängige Programmierparadigmen und moderne
  Entwicklungsmethoden um professionell größere Programmsysteme zu
  erstellen und sorgfältig testen zu können.
\item
  haben die Fähigkeit sich in vorhandenen Quelltext einzuarbeiten und
  diesen sinnvoll weiter zu entwickeln.
\item
  haben Kenntnisse über Konfigurations-, Change-, Release- und
  Deployment-Management.
\item
  können Arbeitsprozesse gestalten und insbesondere die eigene und
  anderer Personen Arbeit organisieren, sie sind teamfähig und in der
  Lage sich konstruktiv mit Konzepten und Lösungsvorschlägen auseinander
  zu setzen.
\item
  haben gelernt, auch unter begrenzten Ressourcen Lösungen zu
  erarbeiten, die allgemein anerkannten Qualitätsstandards genügen und
  von allen Beteiligten akzeptiert werden.
\end{itemize}

\subsection{Technologische
Kompetenzen\label{/mi-2017/selbstbericht/0200-qualifikationsziele/0000-qualifikationsziele}}\label{technologische-kompetenzenpathlabelmi-2017selbstbericht0200-qualifikationsziele0000-qualifikationsziele}

Die Studierenden \ldots{}

\begin{itemize}
\tightlist
\item
  haben Kenntnisse über moderne Betriebssysteme, Rechnerarchitekturen
  und Rechnernetze sowie deren Anwendung in konkreten Problemstellungen
  und Anwendungskontexten.
\item
  sind in der Lage die Infrastruktur für verteilte Systeme unter Nutzung
  von Middleware zu entwerfen.
\item
  beherrschend den Prozess vom Datenbankentwurf bis zum Betrieb des
  datenbankgestützten Anwendungssystems sowie Datenanalyse und
  Grundlagen des maschinellen Lernens.
\item
  haben fundierte Kenntnisse zu Sicherheitsmaßnahmen und -mechanismen.
\end{itemize}

\subsection{Fachübergreifende
Kompetenzen\label{/mi-2017/selbstbericht/0200-qualifikationsziele/0000-qualifikationsziele}}\label{fachuxfcbergreifende-kompetenzenpathlabelmi-2017selbstbericht0200-qualifikationsziele0000-qualifikationsziele}

Die Studierenden \ldots{}

\begin{itemize}
\tightlist
\item
  sind in der Lage, Aufgaben in verschiedenen Anwendungsfeldern unter
  gegebenen technischen, ökonomischen, ökologischen und sozialen
  Randbedingungen mit den Mitteln der Informatik zu bearbeiten und
  entsprechende Systeme zu entwickeln.
\item
  können sich eigenständig in neue Themenbereiche einarbeiten und
  Problemstellungen, Technologien und wissenschaftliche Erkenntnisse im
  Umfeld der Medieninformatik erkennen und in ihrem Arbeitsumfeld
  einbeziehen und das erworbene Wissen effizient in die Lösung aktueller
  und auch zukünftiger Frage- und Problemstellungen einbringen und
  anwenden.
\item
  verfügen über betriebswirtschaftliche Grundkenntnisse, da die Planung,
  Entwicklung und Nutzung aller Informatiksysteme unter wirtschaftlichen
  Rahmenbedingungen stattfinden.
\item
  haben juristische Grundkenntnisse um rechtsverbindliche Dokumente wie
  Rahmenvereinbarungen, projektspezifische Verträge, Lizenz- oder
  Nutzungsverträge aushandeln zu können. Darüber hinaus können sie die
  gesetzliche Basis von Sicherheitsaspekten, als auch Fragen des
  Urheberrechts und der Produkthaftung berücksichtigen.
\item
  haben einen Einblick in berufsethische Rahmenbedingungen erhalten und
  können die Auswirkungen ihrer Arbeit auf die zukünftigen Nutzer sowie
  auf die Gesellschaft in ihren sozialen, wirtschaftlichen,
  arbeitsorganisatorischen, psychologischen und rechtlichen Aspekten
  einschätzen.
\end{itemize}

\subsection{Soziale Kompetenzen und
Selbstkompetenzen\label{/mi-2017/selbstbericht/0200-qualifikationsziele/0000-qualifikationsziele}}\label{soziale-kompetenzen-und-selbstkompetenzenpathlabelmi-2017selbstbericht0200-qualifikationsziele0000-qualifikationsziele}

Die Studierenden \ldots{}

\begin{itemize}
\tightlist
\item
  verfügen über kommunikative Kompetenzen, um ihre Ideen und
  Lösungsvorschläge schriftlich oder mündlich überzeugend zu
  präsentieren, abweichende Positionen zu erkennen und in eine sach- und
  interessengerechte Lösung zu integrieren -- und zwar auch dann, wenn
  die informatische Sprech- und Denkweisen dem Kommunikationspartner
  nicht geläufig sind.
\item
  kennen ihre berufliche Rolle, die damit verbundenen Erwartungen und
  ggf. vorhandene Rollenkonflikte in Kommunikationssituationen und
  können zur Konfliktlösung beitragen. Dazu sind auch Kenntnisse im
  Konfliktmanagement erforderlich, um in kontroversen Diskussionen
  zielorientiert zu argumentieren und mit Kritik sachlich umzugehen.
  Vorhandene Missverständnisse zwischen Gesprächspartnern müssen
  frühzeitig erkannt und abgebaut werden können.
\end{itemize}

\subsection{Methoden- und
Transferkompetenz\label{/mi-2017/selbstbericht/0200-qualifikationsziele/0000-qualifikationsziele}}\label{methoden--und-transferkompetenzpathlabelmi-2017selbstbericht0200-qualifikationsziele0000-qualifikationsziele}

Die Studierenden \ldots{}

\begin{itemize}
\tightlist
\item
  sind in der Lage sich selbstständig neues Wissen anzueigenen und zu
  erkennen, welches Wissen relevant ist.
\item
  haben die Kompetenz zum wissenschaftlichen Arbeiten.
\item
  können (Informatik-)systeme mit systematischen Verfahren empirisch
  evaluieren.
\item
  sind in der Lage, neue informatische Methoden in eine oft historisch
  gewachsene betriebliche Praxis einzuführen.
\item
  haben die Fähigkeit, einen existierenden Anwendungskontext zu
  analysieren, zu bewerten und aktuelle problemadäquate informatische
  Methoden auf diesen Kontext zu übertragen, sowie den derart neu
  generierten Anwendungskontext zu evaluieren.
\end{itemize}

\subsection{Medienkompetenz\label{/mi-2017/selbstbericht/0200-qualifikationsziele/0000-qualifikationsziele}}\label{medienkompetenzpathlabelmi-2017selbstbericht0200-qualifikationsziele0000-qualifikationsziele}

Die Studierenden \ldots{}

\begin{itemize}
\tightlist
\item
  können eine Perspektive der Medienkonzeption einnehmen, haben eine
  mediengestalterische Grundkompetenz entwickelt und sind in der Lage,
  bzgl. der Kommunikationsziele eine geeignete Medienauswahl zu treffen.
\item
  können organisationale, soziale, gestalterische und kulturelle
  Kontexte, Vorgaben und Regeln erschließen, analysieren, definieren und
  unter Berücksichtigung weiterer fachlicher Perspektiven angemessene
  Gestaltungsziele formulieren.
\item
  kennen die Gestaltungsdimensionen von Medien und besitzen aktive
  Vokabularien zur Beschreibung und Realisierung angemessener
  Konzeptionen.
\item
  können die Realisationen bezüglich der Zielsetzungen kritisch
  diskutieren.
\end{itemize}

\section{Qualifikationsziele Medieninformatik
Bachelor\label{/mi-2017/selbstbericht/0200-qualifikationsziele/0000-qualifikationsziele}}\label{qualifikationsziele-medieninformatik-bachelorpathlabelmi-2017selbstbericht0200-qualifikationsziele0000-qualifikationsziele}

\subsection{Leitbild\label{/mi-2017/selbstbericht/0200-qualifikationsziele/0000-qualifikationsziele}}\label{leitbildpathlabelmi-2017selbstbericht0200-qualifikationsziele0000-qualifikationsziele}

Das folgende Leitbild steht über dem Studiengang Medieninformatik
Bachelor:

\emph{Der Studiengang soll die Absolventinnen und Absolventen befähigen,
in interdisziplinären Teams digitale Prozesse und Systeme zur
Produktion, Bearbeitung und Distribution medienbasierter Informationen
aus informatischer, ökonomischer und sozialer Perspektive zu
analysieren, konzipieren, adaptieren, realisieren und dokumentieren.}

\subsection{Ziele des zu reakkreditierenden
Studiengangs\label{/mi-2017/selbstbericht/0200-qualifikationsziele/0000-qualifikationsziele}}\label{ziele-des-zu-reakkreditierenden-studiengangspathlabelmi-2017selbstbericht0200-qualifikationsziele0000-qualifikationsziele}

Mit dem 6-semestrigen Bachelorstudiengang Medieninformatik sollen die
Absolventinnen und Absolventen fachliche Qualifikation, Kompetenz,
Fertigkeiten und Fähigkeiten im Hinblick auf die verantwortliche, sowie
menschen- und medienadäquate Umsetzung von Konzepten und Verfahren aus
der Informatik erlangen. Sie verfügen über tiefgehendes Verständnis um
informatikspezifische Probleme und Aufgaben, wie sie in
interdisziplinären medienrelevanten Softwareprojekten typisch sind, auf
fachlicher Ebene diskutieren und fundierte Entscheidungen treffen zu
können. Die Absolventinnen und Absolventen sollen in der Lage sein,
eigen- oder fremdformulierte Problemstellungen in Aufgabenstellungen zu
überführen, Zielsetzungen auf unterschiedlichen Hierarchieebenen zu
formulieren und diese Zielsetzungen mit den Konzepten,
Vorgehensmodellen, Methoden und Arbeitstechniken der Informatik
strukturiert und systematisch in Teams zu bearbeiten, Handlungs- und
Lösungs-Alternativen kritisch zu diskutieren und begründet Abwägungen zu
treffen, Ergebnisse kritisch sowie in Hinblick auf die Zielsetzungen zu
bewerten und weitere Perspektiven aufzuzeigen.

Die Fähigkeit, konstruktiv in einem interdisziplinären Team zu arbeiten,
eigene Beiträge oder die anderer Teammitglieder fachlich kritisch zu
würdigen und im Projekt zu berücksichtigen, stellt eine wichtige
soziale, methodische und fachliche Qualifikation dar, die durch den
Studienabschluss erreicht werden soll. Kommunikations- und
Präsentationskompetenzen sind für die berufliche Praxis, gerade wegen
der Interdisziplinarität und Arbeitsteiligkeit vieler Softwareprojekte
mit medialer Ausrichtung, von besonderer Bedeutung.

\subsection{Darstellung der durch das Studium zu erreichenden
Lernergebnisse\label{/mi-2017/selbstbericht/0200-qualifikationsziele/0000-qualifikationsziele}}\label{darstellung-der-durch-das-studium-zu-erreichenden-lernergebnissepathlabelmi-2017selbstbericht0200-qualifikationsziele0000-qualifikationsziele}

Zusammenfassend lassen sich für den Bachelorstudiengang folgende
übergeordnete, sich gegenseitig ergänzende und teils auch überlappende
Studienziele definieren.

Absolventinnen und Absolventen des Bachelorstudiengangs
Medieninformatik:

\begin{itemize}
\tightlist
\item
  haben ein Verständnis für anwendbare Techniken und Methoden in der
  Wertschöpfungskette aus Medienkonzeption, -produktion, -bearbeitung,
  -distribution und -nutzung und für deren Grenzen entwickelt und
  fachliches als auch fachübergreifendes Wissen der Informatik und der
  Medieninformatik erlangt und ihre Fähigkeit zur Abstraktion und
  Modellierung sowie zum Operieren in formalen Welten mit methodischen
  und analytischen Ansätzen erlernt.
\item
  können im Team Problemstellungen verschiedenen Bereichen der
  Medieninformatik grundlagenbasiert, systemanalytisch und
  multiperspektivisch analysieren, formulieren, formalisieren und lösen,
  sowie solche Lösungen kritisch zu evaluieren. Sie haben dafür ein
  kritisches Bewusstsein über die neueren Erkenntnisse und Entwicklungen
  in der Medieninformatik entwickelt und kennen nicht-technische
  Auswirkungen ihrer praktischen Tätigkeit.
\item
  haben die Fähigkeit zur Einarbeitung in informatikfremde Sachverhalte
  und technologische Problemlösungsmethoden.
\item
  erarbeiteten sich Medienkompetenzen und können Konzeptionen und
  Informationen, bezüglich ihrer Struktur, Nutzung und ihres
  Managements, modellieren, unter Berücksichtigung fachlicher,
  organisatorischer, sozialer und kultureller Kontexte sowie Vorgaben
  und Regeln, angemessene Gestaltungsziele formulieren, sowie
  Konzeptionen im Kontext etablierter fachlicher Theorien und Konzepte
  einordnen, analysieren, diskutieren und bewerten.
\item
  haben anhand praxisnaher Projekte und Fallstudien die Kompetenz
  erworben, eigenverantwortlich und professionell Projekte im Umfeld der
  Medieninformatik durchführen zu können, sowie die Fähigkeit zur
  effektiven und effizienten Kommunikation und zur Teamarbeit erlangt.
\item
  verfügen über Wissen bezüglich der Rahmenbedingungen von
  Software-gestützen Systemen und Prozessen und haben ihre Fähigkeit zum
  methodischen und systematischen Vorgehen, der Auswahl und der
  Durchführung von Arbeits- und Dokumentationstechniken erlangt und sind
  fähig Methoden, Konzepte und Techniken bei der Problemlösung
  auszuwählen, anzuwenden und deren Anwendung zu begründen.
\item
  wurden an Probleme und Fragestellungen der Medieninformatik
  herangeführt und können auch Problemstellungen, Technologien und
  wissenschaftliche Erkenntnisse im Umfeld der Medieninformatik erkennen
  und in ihrem Arbeitsumfeld einbeziehen sowie selbst wissenschaftlich
  arbeiten und Beiträge zur Weiterentwicklung der Medieninformatik als
  Disziplin leisten.
\item
  haben ihre Fähigkeit zum lebenslangen Lernen aufgebaut und können sich
  selbständig in neue, für die Medieninformatik relevante, Theorien,
  Methoden und Techniken, sowohl aus theoretischer als auch aus
  technischer Sichtweise, einarbeiten und ihre eigene Rolle im
  professionellen Kontext hinterfragen und weiterentwickeln.
\end{itemize}

\subsection{Weiterführende
Dokumente\label{/mi-2017/selbstbericht/0200-qualifikationsziele/0000-qualifikationsziele}}\label{weiterfuxfchrende-dokumentepathlabelmi-2017selbstbericht0200-qualifikationsziele0000-qualifikationsziele}

\begin{itemize}
\tightlist
\item
  \href{https://www.th-koeln.de/studium/medieninformatik-bachelor_2379.php}{Website
  des Medieninformatik Bachelor}
\item
  \href{https://www.th-koeln.de/studium/medieninformatik-bachelor--ordnungen-und-formulare_3963.php}{Ordnungen
  zum Medieninformatik Bachelor}
\item
  \href{https://th-koeln.github.io/mi-2017/anhaenge/ba-abschlussarbeiten_2010-2014_.pdf}{Themen der
  Abschlussabeiten des Medieninformatik Bachelor 2010 bis 2014}
\item
  \href{https://th-koeln.github.io/mi-2017/anhaenge/ba-studienverlaufsplan.pdf}{Studienverlaufsplan
  Medieninformatik Bachelor}
\item
  \href{https://th-koeln.github.io/mi-2017/download/modulbeschreibungen-bachelor.pdf}{Modulhandbuch
  Medieninformatik Bachelor}
\item
  \href{https://th-koeln.github.io/mi-2017/anhaenge/ba-Ziele-Module-Matrix-Medieninformatik-Bachelor.pdf}{Ziele-Module-Matrix
  Medieninformatik Bachelor}
\item
  \href{https://th-koeln.github.io/mi-2017/anhaenge/ba-zeugnis.pdf}{Beispielzeugnis und Diploma
  Supplement Medieninformatik Bachelor}
\item
  \href{https://th-koeln.github.io/mi-2017/anhaenge/stat-profil-studienanfaenger-2017.pdf}{Profil der
  Studienanfänger}
\end{itemize}

\section{Qualifikationsziele Medieninformatik
Master\label{/mi-2017/selbstbericht/0200-qualifikationsziele/0000-qualifikationsziele}}\label{qualifikationsziele-medieninformatik-masterpathlabelmi-2017selbstbericht0200-qualifikationsziele0000-qualifikationsziele}

\subsection{Leitbild Medieninformatik
Master\label{/mi-2017/selbstbericht/0200-qualifikationsziele/0000-qualifikationsziele}}\label{leitbild-medieninformatik-masterpathlabelmi-2017selbstbericht0200-qualifikationsziele0000-qualifikationsziele}

Der konsekutive Masterstudiengang ist auf das folgende Leitbild
ausgerichtet:

\emph{Absolventinnen und Absolventen des Masterstudiengangs
Medieninformatik sind befähigt, an der Analyse komplexer
informatik-spezifischer Aufgabenstellungen im Kontext interaktiver, oft
web-basierter, multimedialer Systeme an leitender Stelle in
interdisziplinären Entwicklungsteams mitzuwirken, Lösungskonzepte
verantwortlich zu entwerfen und zu realisieren, kritisch einzuordnen und
zu evaluieren sowie in der fachlichen Öffentlichkeit zu kommunizieren
und zu verwerten.}

\subsection{Ziele des zu reakkreditierenden Studiengangs
\label{/mi-2017/selbstbericht/0200-qualifikationsziele/0000-qualifikationsziele}}\label{ziele-des-zu-reakkreditierenden-studiengangs-pathlabelmi-2017selbstbericht0200-qualifikationsziele0000-qualifikationsziele}

Im konsekutiven 4-semestrigen Masterstudiengang Medieninformatik werden
die, im Rahmen des ersten berufsbefähigenden Studiums erworbenen,
fachlichen und fachübergreifenden Fähigkeiten, sowie die sozialen
Kompetenzen vertieft und erweitert. Der Masterstudiengang
Medieninformatik befähigt die Absolventinnen und Absolventen, auf dem
Stand von Wissenschaft und Technik an der Analyse komplexer
informatik-spezifischer Aufgabenstellungen im Kontext multimedialer
Informations-und Kommunikationssystem an leitender Stelle mitzuwirken,
Lösungskonzepte verantwortlich zu entwerfen und interdisziplinäre
Entwicklungsteams zu führen. Dazu erwerben die Studierenden die
Kompetenz, umfangreiche und zum Teil auch gegenläufige Anforderungen zu
ermitteln und unter sozialen wie wirtschaftlichen Kosten-Nutzen-Aspekten
zu hinterfragen, Lösungsarchitekturen und Lösungsstrategien zu entwerfen
oder Referenzmodelle für neue Aufgabenstellungen zu entwickeln. Zudem
werden die Studierenden in Teilbereichen der Medieninformatik an
aktuelle Forschungsthemen herangeführt. Sie erwerben Methoden des
Selbstmanagements, um im Berufsalltag an vorderster Wissensfront
Aufgaben bewältigen zu können.

\subsection{Darstellung der durch das Studium zu erreichenden
Lernergebnisse
\label{/mi-2017/selbstbericht/0200-qualifikationsziele/0000-qualifikationsziele}}\label{darstellung-der-durch-das-studium-zu-erreichenden-lernergebnisse-pathlabelmi-2017selbstbericht0200-qualifikationsziele0000-qualifikationsziele}

Zusammenfassend lassen sich für den Master Medieninformatik folgende
übergeordnete, sich gegenseitig ergänzende und teils auch überlappende
Studienziele definieren.

Absolventinnen und Absolventen des Masterstudiengangs Medieninformatik:

\begin{itemize}
\tightlist
\item
  haben das im Rahmen ihres ersten berufsbefähigenden Studiums erworbene
  fachliche und fachübergreifende Wissen der Informatik und insbes. der
  Medieninformatik vertieft und ihre Fähigkeit zur Abstraktion und
  Modellierung sowie zum Operieren in formalen Welten mit erweitertem
  methodischen und analytischen Ansatz verbreitert. Sie haben ein
  umfassendes Verständnis für anwendbare Techniken und Methoden in der
  Wertschöpfungskette aus Medienkonzeption, -produktion, -bearbeitung,
  -distribution und -nutzung und für deren Grenzen entwickelt.
\item
  sind dazu befähigt, in leitender Position Problemstellungen aus neuen
  und in der Entwicklung begriffenen Bereichen der Medieninformatik
  grundlagenbasiert, systemanalytisch und multiperspektivisch zu
  analysieren, zu formulieren, zu formalisieren und zu lösen, sowie
  solche Lösungen kritisch zu evaluieren. Sie haben dafür ein kritisches
  Bewusstsein über die neueren Erkenntnisse und Entwicklungen in der
  Informatik und insbesondere der Medieninformatik entwickelt und kennen
  nicht-technische Auswirkungen ihrer praktischen Tätigkeit auf und
  innerhalb von sozio-technischen Systemen. Dazu gehört auch die
  Fähigkeit zur Einarbeitung in informatikfremde Sachverhalte und
  technologische Problemlösungsmethoden.
\item
  entwickelten ihre Medienkompetenzen in wichtigen Kernfächern weiter
  und können Konzeptionen und Informationen bezüglich ihrer Struktur,
  Nutzung und ihres Managements modellieren, unter Berücksichtigung
  fachlicher, organisatorischer, sozialer und kultureller Kontexte sowie
  Vorgaben und Regeln, angemessene Gestaltungsziele formulieren, sowie
  Konzeptionen im Kontext etablierter wissenschaftlicher Theorien
  einordnen, analysieren, diskutieren und bewerten.
\item
  haben anhand praxisnaher Projekte und Fallstudien die Kompetenz
  erworben, eigenverantwortlich und professionell Projekte im Umfeld der
  Medieninformatik zu organisieren als auch durchführen zu können, sowie
  die Fähigkeit zur effektiven und effizienten Kommunikation und zur
  Teamarbeit erweitert und vertieft. Sie erwerben Wissen bzgl.
  kultureller Rahmenbedingungen menschlichen Handelns, kennen Konzepte
  der Ethik und können diese handlungsleitend integrieren. Sie haben
  dabei ihre Fähigkeit zum methodischen Vorgehen, der Auswahl und der
  Durchführung von Arbeits- und Dokumentationstechniken vertieft und
  sind fähig, innovative Methoden bei der Problemlösung auszuwählen,
  anzuwenden und deren Anwendung zu begründen.
\item
  wurden an forschungsnahe Fragestellungen der Medieninformatik
  herangeführt und können auch zukünftige Problemstellungen,
  Technologien und wissenschaftliche Erkenntnisse im Umfeld der
  Medieninformatik erkennen und in ihrem Arbeitsumfeld einbeziehen sowie
  selbst wissenschaftlich arbeiten und -- etwa im Rahmen einer
  Dissertation - Beiträge zur Weiterentwicklung der Medieninformatik als
  wissenschaftlicher Disziplin leisten. Sie können wissenschaftliche
  Arbeiten für unterschiedliche Zielgruppen aufbereiten sowie fundiert
  und überzeugend präsentieren und können Kritikpunkte und abweichende
  Positionen verstehen, bewerten und angemessen in eigene
  wissenschaftliche Arbeiten einfließen lassen
\item
  haben ihre Fähigkeit zum lebenslangen Lernen gefestigt und können sich
  selbständig und schnell in neue, für die Medieninformatik relevante
  Theorien, Methoden und Techniken, sowohl aus theoretischer,
  wissenschaftlicher. praktischer als auch aus technischer Sichtweise,
  einarbeiten und ihre eigene Rolle im professionellen Kontext
  hinterfragen und weiterentwickeln.
\end{itemize}

\subsection{Weiterführende
Dokumente\label{/mi-2017/selbstbericht/0200-qualifikationsziele/0000-qualifikationsziele}}\label{weiterfuxfchrende-dokumentepathlabelmi-2017selbstbericht0200-qualifikationsziele0000-qualifikationsziele-1}

\begin{itemize}
\tightlist
\item
  \href{https://www.th-koeln.de/studium/medieninformatik-master_3729.php}{Website
  des Medieninformatik Master}
\item
  \href{https://www.th-koeln.de/studium/medieninformatik-master--ordnungen-und-formulare_3724.php}{Ordnungen
  zum Medieninformatik Master}
\item
  \href{https://th-koeln.github.io/mi-2017/anhaenge/ma-studienverlaufsplan.pdf}{Studienverlaufsplan
  Medieninformatik Master}
\item
  \href{https://th-koeln.github.io/mi-2017/download/modulbeschreibungen-master.pdf}{Modulhandbuch
  Medieninformatik Master}
\item
  \href{https://th-koeln.github.io/mi-2017/anhaenge/ma-MIMPO_Entwurf_20170218.pdf}{Prüfungsordnung
  Medieninformatik Master(entwurf)}
\item
  \href{https://th-koeln.github.io/mi-2017/anhaenge/ma-Ziele-Module-Matrix-Medieninformatik-Master.pdf}{Ziele-Module-Matrix
  Medieninformatik Master}
\item
  \href{https://th-koeln.github.io/mi-2017/anhaenge/ma-zeugnis.pdf}{Beispielzeugnis und Diploma
  Supplement Medieninformatik Master}
\end{itemize}

\chapter{Konzeptionelle Einordnung des Studiengangs in das
Studiensystem\label{/mi-2017/selbstbericht/0300-konzeptionelle-einordnung/0000-konzeptionelle-einordnung}}\label{konzeptionelle-einordnung-des-studiengangs-in-das-studiensystempathlabelmi-2017selbstbericht0300-konzeptionelle-einordnung0000-konzeptionelle-einordnung}

Beide zu reakkreditierende Studiengänge entsprechen weiterhin

\begin{enumerate}
\def\labelenumi{\arabic{enumi}.}
\item
  den Anforderungen des Qualifikationsrahmens für deutsche
  Hochschulabschlüsse vom 21.04.2005 in der jeweils gültigen Fassung,
\item
  den Anforderungen der Ländergemeinsamen Strukturvorgaben für die
  Akkreditierung von Bachelor- und Masterstudiengängen vom 10.10.2003 in
  der jeweils gültigen Fassung,
\item
  den landesspezifischen Strukturvorgaben für die Akkreditierung von
  Bachelor- und Masterstudiengängen,
\item
  der verbindlichen Auslegung und Zusammenfassung von (1.) bis (3.)
  durch den Akkreditierungsrat.
\end{enumerate}

Darüber hinaus wurden bei der Überarbeitung der Studiengänge die
aktuellen Empfehlungen für Bachelor- und Masterprogramme im Studienfach
Informatik (Stand: 01.07.2016) der Gesellschaft für Informatik zugrunde
gelegt.

\section{Weiterführende
Dokumente\label{/mi-2017/selbstbericht/0300-konzeptionelle-einordnung/0000-konzeptionelle-einordnung}}\label{weiterfuxfchrende-dokumentepathlabelmi-2017selbstbericht0300-konzeptionelle-einordnung0000-konzeptionelle-einordnung}

\begin{itemize}
\tightlist
\item
  \href{https://th-koeln.github.io/mi-2017/anhaenge/ba-studienverlaufsplan.pdf}{Studienverlaufsplan
  Medieninformatik Bachelor}
\item
  \href{https://th-koeln.github.io/mi-2017/download/modulbeschreibungen-bachelor.pdf}{Modulhandbuch
  Medieninformatik Bachelor}
\item
  \href{https://th-koeln.github.io/mi-2017/anhaenge/ba-Ziele-Module-Matrix-Medieninformatik-Bachelor.pdf}{Ziele-Module-Matrix
  Medieninformatik Bachelor}
\item
  \href{https://th-koeln.github.io/mi-2017/anhaenge/ba-zeugnis.pdf}{Beispielzeugnis und Diploma
  Supplement Medieninformatik Bachelor}
\item
  \href{https://www.th-koeln.de/studium/medieninformatik-master_3729.php}{Website
  des Medieninformatik Master}
\item
  \href{https://www.th-koeln.de/studium/medieninformatik-master--ordnungen-und-formulare_3724.php}{Ordnungen
  zum Medieninformatik Master}
\item
  \href{https://th-koeln.github.io/mi-2017/anhaenge/ma-studienverlaufsplan.pdf}{Studienverlaufsplan
  Medieninformatik Master}
\item
  \href{https://th-koeln.github.io/mi-2017/download/modulbeschreibungen-master.pdf}{Modulhandbuch
  Medieninformatik Master}
\item
  \href{https://th-koeln.github.io/mi-2017/anhaenge/ma-Ziele-Module-Matrix-Medieninformatik-Master.pdf}{Ziele-Module-Matrix
  Medieninformatik Master}
\item
  \href{https://th-koeln.github.io/mi-2017/anhaenge/ma-zeugnis.pdf}{Beispielzeugnis und Diploma
  Supplement Medieninformatik Master}
\end{itemize}

\chapter{Studiengangskonzept\label{/mi-2017/selbstbericht/0400-studiengangskonzept/0000-studiengangskonzept}}\label{studiengangskonzeptpathlabelmi-2017selbstbericht0400-studiengangskonzept0000-studiengangskonzept}

\section{Zielgruppe und
Studienform\label{/mi-2017/selbstbericht/0400-studiengangskonzept/0000-studiengangskonzept}}\label{zielgruppe-und-studienformpathlabelmi-2017selbstbericht0400-studiengangskonzept0000-studiengangskonzept}

Das Studienangebot richtet sich primär an Studierende des
deutschsprachigen Raumes. Ausländische Studienbewerber werden durch ein
etabliertes, durch das Sekretariat für internationale Studierende
betreutes Verfahren nach Nachweis der Kenntnisse der deutschen Sprache
aufgenommen. Das Lehrangebot wird in beiden Studiengängen in Form eines
klassischen Präsenzstudiums erbracht.

\section{Zusammensetzung der
Studierendenschaft\label{/mi-2017/selbstbericht/0400-studiengangskonzept/0000-studiengangskonzept}}\label{zusammensetzung-der-studierendenschaftpathlabelmi-2017selbstbericht0400-studiengangskonzept0000-studiengangskonzept}

Die Darstellung des Profils der Studierendenschaft wird hier an der
Lehreinheit Informatik beziehungsweise am Campus Gummersbach
dargestellt\footnote{\href{https://th-koeln.github.io/mi-2017/anhaenge/stat-profil-studienanfaenger-2017.pdf}{Profil
  der Studienanfänger}}. Der Anteil weiblicher Studierender am Campus
Gummersbach schwankt in den letzten Jahren um knapp 20\%. Weibliche
Studierende benötigen im Mittel ein Semester weniger zu einem
erfolgreichen Abschluss (Basis: Lehreinheit Informatik) und erzielen
dabei eine geringfügig bessere Abschlussnote als die männlichen
Studierenden, sodass von einer Chancengleichheit auszugehen ist. 16,9\%
der Studierenden am Campus Gummersbach verfügen über eine ausländische
Staatsangehörigkeit. Die stärkste Gruppe stellen europäische Ausländer
mit knapp 60\%, gefolgt von asiatischen und afrikanischen Ausländern mit
jeweils 18 \%. Die anderen Kontinente spielen eine untergeordnete Rolle.
Die Studiendauer ausländischer Studierender ist knapp zwei Semester
länger als die Studiendauer deutscher Studierender. Die Abschlussnote
ist mit 2,4 schlechter als die deutscher Absolventen (2,0). Studierende
mit einer allgemeinen Hochschulreife benötigen im Mittel ein Semester
weniger als Studierende mit einer FH-Reife oder einer fachbezogenen
Hochschulreife.

\section{Eingesetzte
Lehrformen\label{/mi-2017/selbstbericht/0400-studiengangskonzept/0000-studiengangskonzept}}\label{eingesetzte-lehrformenpathlabelmi-2017selbstbericht0400-studiengangskonzept0000-studiengangskonzept}

Die Lehrinhalte und Veranstaltungsformen orientieren sich an den
Notwendigkeiten des jeweiligen Moduls, so dass neben Vorlesung und
Praktika eine Reihe anderer Lehrformen, wie seminaristischer Unterricht,
Flipped Classroom und Workshops zum Einsatz kommen. Die Medieninformatik
Studiengänge sind von einer starken Projektorientierung geprägt. Hierbei
sind sowohl die Projekt- als auch die Teamgrößen sehr unterschiedlich.
Die Projektgrößen gehen von 2,5 bis 12 Creditpoints und die Teamgrößen
von 2er bis zu 9er Teams. Die im Selbststudium zu erbringenden
Creditpoints der nicht projektorientierten Module betragen im Bachelor
ca. 50\% und im Master ca. 60\%.

\section{Studienkonzept Medieninformatik
Bachelor\label{/mi-2017/selbstbericht/0400-studiengangskonzept/0000-studiengangskonzept}}\label{studienkonzept-medieninformatik-bachelorpathlabelmi-2017selbstbericht0400-studiengangskonzept0000-studiengangskonzept}

Um die Qualifikationsziele des sechssemestrigen Studiengangs zu
erreichen, das Studium sinnvoll in das Lehrportfolio der Fakultät
einbetten zu können und um eine größtmögliche Zufriedenheit bei den
Studierenden zu erzielen, ist das Studiengangskonzept wie folgt
aufgebaut.

\subsection{Studienphasen und
-säulen\label{/mi-2017/selbstbericht/0400-studiengangskonzept/0000-studiengangskonzept}}\label{studienphasen-und--suxe4ulenpathlabelmi-2017selbstbericht0400-studiengangskonzept0000-studiengangskonzept}

\begin{figure}[htbp]
\centering
\includegraphics[width=\columnwidth]{../anhaenge/bilder/ba-studienphasen.png}
\caption{Studienphasen des Bachelorstudiengangs Medieninformatik}
\end{figure}

Die organisatorischen und inhaltliche Klammer bilden die drei,
aufeinander aufbauenden, Studienphasen: Grundlagen, Vertiefung und
Spezialisierung. Die Module innerhalb der Phasen, gliedern sich in zwei
Säulen: Informatik Kernmodule und Medieninformatik-spezifische Module.
Die Informatik Kernmodule werden auch von anderen Informatik
Studiengängen der Fakultät 10 genutzt. Dadurch werden Synergien erzeugt
und alle Informatik Studenten der Fakultät können auf die gleiche
Wissensbasis zurückgreifen. Durch die Durchmischung von Studierenden
unterschiedlicher Studiengänge können hier unter den Studierenden
bereits interdisziplinäre Kontakte geknüpft werden. Ein weiterer Vorteil
dieser Konstruktion ist eine gute Durchlässigkeit von Studierenden beim
Studiengangswechsel, sofern sie feststellen, dass ein anderer
Studiengang am Campus eher ihren Fähigkeiten und Neigungen entspricht.
Naturgemäß sind die Informatik Kernmodule, die zumeist
Grundlagencharakter haben, im Grundlagenteil des Studiums verankert.

Um den Studierenden möglichst früh mit den Herangehensweisen und
Perspektiven der Medieninformatik vertraut zu machen und sie bei ihrer
Identifikation mit der Domäne zu unterstützen, wird im ersten Semester
das Modul ``Einführung in die Medieninformatik'' mit einem Gewicht von
10 Creditpoints angeboten. Im zweiten Semester übernimmt das Modul
``Mensch-Computer Interaktion'' mit einem Gewicht von 10 Creditpoints
diese Aufgabe. Das Gewicht der Medieninformatik-spezifischen Module
nimmt mit jedem Semester zu.

\subsection{Sinnvolle Staffelung der
Module\label{/mi-2017/selbstbericht/0400-studiengangskonzept/0000-studiengangskonzept}}\label{sinnvolle-staffelung-der-modulepathlabelmi-2017selbstbericht0400-studiengangskonzept0000-studiengangskonzept}

\begin{figure}[htbp]
\centering
\includegraphics[width=\columnwidth]{../anhaenge/bilder/ba-studienverlaufsplan.png}
\caption{Studienverlaufsplan des Bachelorstudiengangs Medieninformatik}
\end{figure}

Die ersten Modulen des Informatik Kerns bauen vor allem mathematische,
algorithmische und und grundlegende Kenntnisse, Fähigkeiten und
Fertigkeiten auf. Im Kontrast dazu, werden im Modul ``Einführung in die
Medieninformatik'' vielfältige Perspektiven, Konzepte und
Arbeitstechniken der Medieninformatik, quasi im Vorgriff auf das
kommende Studium, vorgestellt und in einem Projekt, mit eher forschendem
Charakter, angewendet. Hiermit wird den Studierenden ein Ausblick auf
das weitere Studium und die notwendigen Arbeitsweisen und -techniken
gegeben.

Im weiteren Studienverlauf sind die Medieninformatik-spezifischen Module
an den groben Phasen der Produktentwicklung ausgerichtet: Konzeption,
Realisierung und Reflexion. Die Module ``Mensch-Computer Interaktion'',
``Screendesign'' und ``Web-Architekturen'' vermitteln überwiegend
Herangehensweisen, Techniken und Konzepte für die Konzeptionsphase eines
Projekts. Das Modul ``Audiovisuelles Medienprojekt'' adressiert sowohl
Themen rund um die Konzeption als auch in Richtung Projektrealisierung
und bildet den Übergang zum Vertiefungsmodul im vierten Semester, in dem
Realisierungsaspekte im Vordergrund stehen.

Mit dem ``Entwicklungsprojekt'' im fünften Semester werden zunehmend
Fragestellungen und Herangehensweisen zur Reflexion eines Projekts, z.B.
als Vorbereitung einer weiteren Projektiteration, fokussiert.

Um den Studierende fundierte Entwicklungskompetenz zu vermitteln, wird
ein Modulstrang aus dem Informatik Kern verwendet, der ausgehend den
Prinzipien der Objektorientierung und einfachen Algorithmen (Algorithmen
und Programmierung 1) über komplexere Prinzipien der
Algorithmenentwicklung (Algorithmen und Programmierung 2) hin zu
Anwendung unterschiedlicher Programmierkonzepte (Paradigmen der
Programmierung) und Prinzipien, Methoden und Techniken der
modellbasierten Softwareentwicklung (Softwareentwicklung) diese
Kompetenz sukzessive auf- und ausbaut.

\subsection{Individuelle
Vertiefungsmöglichkeiten\label{/mi-2017/selbstbericht/0400-studiengangskonzept/0000-studiengangskonzept}}\label{individuelle-vertiefungsmuxf6glichkeitenpathlabelmi-2017selbstbericht0400-studiengangskonzept0000-studiengangskonzept}

Im vierten Semester wird zur Fachvertiefung entsprechen der persönlichen
Neigung ein Vertiefungsmodul mit einem Gewicht von 20 Creditpoints
angeboten. Hier stehen drei Vertiefungsrichtungen zur Auswahl: Visual
Computing, Social Computing und Web-Development. Die Studierenden haben
in diesem Modul die Möglichkeit, entsprechend der persönlichen Neigung,
ein Themenfeld tief zu durchdringen und damit eine
Spezialisierungsrichtung vorzubereiten. Durch das Zusammenfassen
mehrerer Module werden im Vertiefungsmodul konsistente Regularien, sowie
inhaltliche und organisatorische Zusammenhänge geschaffen.

Das Entwicklungsprojekt im fünften Semester bietet die Möglichkeit zur
weiteren Fachvertiefung entsprechend der persönlichen Neigungen. Im
gleichen Semester ist ein Wahlpflichtmodul verankert, welches aus dem
Wahlpflichtkatalog der Informatik, oder aus den nicht im Pflichtangebot
der Medieninformatik befindlichen Pflichtmodulen der anderen Informatik
Studiengänge der Fakultät gewählt werden kann.

Zusammen mit dem Praxisprojekt und der Bachelorarbeit stehen somit 60
Creditpoints, also ein Drittel der Studienleistungen, zur individuellen
Fachvertiefung zur Verfügung.

\subsection{Projektorientierung, Aufbau der Projektgrößen und
interdisziplinäre
Projekterfahrung\label{/mi-2017/selbstbericht/0400-studiengangskonzept/0000-studiengangskonzept}}\label{projektorientierung-aufbau-der-projektgruxf6uxdfen-und-interdisziplinuxe4re-projekterfahrungpathlabelmi-2017selbstbericht0400-studiengangskonzept0000-studiengangskonzept}

Projektorientierung und forschendes Lernen sind seit der
Erstakkreditierung des Studiengangs elementare Bestandteile des
Studienkonzepts. Diese Ansätze haben in den letzten Jahren vermehrt
Einzug in verschiedene Module erhalten. Um hier die Studierenden
einerseits nicht zu überfordern, sie aber aber trotzdem an größere und
komplexere Projekte und Fragestellungen heranzuführen, werden die
Projektanteile und -gewichte im Studiengang behutsam aufgebaut. Das
Module ``Einführung in die Medieninformatik'' startet im ersten Semester
mit einem Projektanteil von 50\% (2,5 CP) um die Studierenden initial
mit dem Lehrformat ``Projekt'' im Hochschulkontext vertraut zu machen.
Das Gewicht der Projekte wird dann im dritten und vierten Semester auf 5
Creditpoints erhöht. Das Modul ``Entwicklungsprojekt'' im fünften
Semester ist mit einem Gewicht von 10 Creditpoints ausgestattet und
leitet über zum Praxisprojekt (10 CP) und der Bachelorarbeit (12 CP).
Das Projektgewicht von 12 Creditpoints wird später, im konsekutiven
Masterstudiengang, weiter geführt. Ausgewählte Projekte können über die
ProfiL2-Projektbörse\footnote{\href{http://projektboerse-profil2.th-koeln.de/}{Homepage
  ProfiL2 Projektbörse}} auch für Studierende anderer Fakultäten
zugänglich gemacht werden.

Der Projektanteil des Vertiefungsmoduls im vierten Fachsemester soll
auch genutzt werden, um Erfahrungen in interdisziplinärer Projektarbeit
sammeln zu können. Dabei liegt die Fokussierung auf der Bedeutung und
Funktion interdisziplinärer Arbeitsprozesse. Die Studierenden sollen
lernen in heterogenen Teams zu agieren und Entscheidungen zu treffen,
sowie Verständnis für die Methoden und Denkweisen anderer Disziplinen zu
entwickeln und über die eigenen Fachgrenzen hinaus zielgerichtet
kommunizieren und arbeiten zu können. Desweiteren sollen sie üben, sich
selbständig zu organisieren, ihre eigene Rolle im interdisziplinären
Team zu finden und ihre eigene Fachperspektive in ein Projekt
einzubringen. Die strukturelle Verankerung interdisziplinäre
Projektarbeit erfolgt entweder im interdisziplinären Großprojekt,
welches in der ProfiL2 Projektwoche der TH Köln zentral angeboten und
organisiert wird, oder in eigenen, fachübergreifenden Projektangeboten.

\subsection{Wissenschaftliches
Arbeiten\label{/mi-2017/selbstbericht/0400-studiengangskonzept/0000-studiengangskonzept}}\label{wissenschaftliches-arbeitenpathlabelmi-2017selbstbericht0400-studiengangskonzept0000-studiengangskonzept}

Wissenschaftliches Arbeiten vermittelt der Bachelor-Studiengang
Medieninformatik beginnend im ersten Semester in der Veranstaltung
``Einführung in die Medieninformatik''. Diese thematisiert und fordert
Themen wie Recherche von und Umgang mit Quellen, adäquater Aufbau von
Dokumenten und Verwendung adäquater (Fach-)Sprache. Darauf greifen u.a.
alle projektorientierten Veranstaltungen bei der Erstellung von
Projektdokumentationen, bei Vorträgen und bei Poster Präsentationen
zurück. Im Rahmen des Praxisprojekt-Seminars wird das Thema im Hinblick
auf die anschließende Erstellung der Bachelorarbeit vertieft und eine
eingehende individuelle Auseinandersetzung mit dem Thema abgerufen.

\subsection{Weiterführende
Dokumente\label{/mi-2017/selbstbericht/0400-studiengangskonzept/0000-studiengangskonzept}}\label{weiterfuxfchrende-dokumentepathlabelmi-2017selbstbericht0400-studiengangskonzept0000-studiengangskonzept}

\begin{itemize}
\tightlist
\item
  \href{https://th-koeln.github.io/mi-2017/anhaenge/ba-abschlussarbeiten_2010-2014_.pdf}{Themen der
  Abschlussabeiten des Medieninformatik Bachelor 2010 bis 2014}
\item
  \href{https://th-koeln.github.io/mi-2017/anhaenge/ba-studienverlaufsplan.pdf}{Studienverlaufsplan
  Medieninformatik Bachelor}
\item
  \href{https://th-koeln.github.io/mi-2017/download/modulbeschreibungen-bachelor.pdf}{Modulhandbuch
  Medieninformatik Bachelor}
\item
  \href{https://th-koeln.github.io/mi-2017/anhaenge/ba-Ziele-Module-Matrix-Medieninformatik-Bachelor.pdf}{Ziele-Module-Matrix
  Medieninformatik Bachelor}
\item
  \href{https://th-koeln.github.io/mi-2017/anhaenge/ba-zeugnis.pdf}{Beispielzeugnis und Diploma
  Supplement Medieninformatik Bachelor}
\end{itemize}

\section{Studienkonzept Medieninformatik
Master\label{/mi-2017/selbstbericht/0400-studiengangskonzept/0000-studiengangskonzept}}\label{studienkonzept-medieninformatik-masterpathlabelmi-2017selbstbericht0400-studiengangskonzept0000-studiengangskonzept}

Der viersemestrige Masterstudiengang baut konsekutiv auf das
Bachelorprogramm auf. Im Gegensatz zum Bachelorstudium, sind hier
Einschreibungen im Sommer- und Wintersemester möglich. Dies führt zu
unterschiedlichen Studienverlaufplänen in Abhängigkeit vom
Einschreibesemester.

Aufgrund der bereits oben betonten Notwendigkeit einer professionellen
Differenzierung und Profilschärfung strukturiert sich der
Masterstudiengang Medieninformatik neben dem gemeinsamen
Grundlagenstudium und der starken Projektorientierung in mehrere
Studienschwerpunkte.

\subsection{Studienschwerpunkte\label{/mi-2017/selbstbericht/0400-studiengangskonzept/0000-studiengangskonzept}}\label{studienschwerpunktepathlabelmi-2017selbstbericht0400-studiengangskonzept0000-studiengangskonzept}

\begin{figure}[htbp]
\centering
\includegraphics[width=\columnwidth]{../anhaenge/bilder/ma-schwerpunkte.png}
\caption{Studienschwerpunkte des Masterprogramms}
\end{figure}

Zur individuellen Schwerpunktbildung bietet des Masterprogramm vier
Möglichkeiten, die alle auf den im Bachelor gelegten Themenbieten
aufbauen: \emph{Human-Computer Interaction}, \emph{Social Computing},
\emph{Visual Computing} und \emph{Weaving the Web}. Für Studierende, die
ein generalistisch geprägtes Studium bevorzugen wird der Studienpfad
\emph{Multiperspective Product Development} angeboten, der sich aus
ausgewählten Modulen der anderen Schwerpunkte und des Wahlplichtkatalogs
speist. Dieser wird aus organisatorischen Gründen auch als Schwerpunkt
aufgeführt.

\paragraph{Mensch-Computer
Interaktion\label{/mi-2017/selbstbericht/0400-studiengangskonzept/0000-studiengangskonzept}}\label{mensch-computer-interaktionpathlabelmi-2017selbstbericht0400-studiengangskonzept0000-studiengangskonzept}

Medieninformatik und Mensch-Computer Interaktion stehen in vielerlei
Hinsicht in einem engen Zusammenhang. So beheimatet etwa der Fachbereich
„Mensch-Computer Interaktion`` der GI e.V. die Fachgruppe
„Medieninformatik``\footnote{\href{http://fb-mci.gi.de/mensch-computer-interaktion-mci/fachgruppen/medieninformatik.html}{Fachbereich
  Mensch-Computer-Interaktion (MCI)/ Fachgruppe Medieninformatik}}. Im
Zusammenhang mit der „third wave of HCI`` (Susan Bødker, 2006\footnote{Bødker,
  S.: When Second Wave HCI Meets Third Wave Challenges. Proc. 4th Nordic
  Conference on Human-Computer Interaction: Changing Roles, ACM, 2006,
  pp.~1-8.} und 2016\footnote{Bødker, S.: Third-wave HCI, 10 Years Later
  - Participation and Sharing. ACM Interactions, Vol. 22, 2015,
  pp.~24-31.}) wird die aktuelle Bedeutung der Disziplin der
Mensch-Computer Interaktion für die Gestaltung interaktiver Systeme und
insbesondere ihre Rolle für die Medieninformatik deutlich. Nach Bødker
besteht eine aktuelle Herausforderung der 3rd wave of HCI insbesondere
darin, dass sich die Trennlinie von Technologienutzung zwischen
beruflichem/gewerblichem und privatem Bereich mehr und mehr auflöst.
Medieninformatik befasst sich insbesondere mit interaktiven und
multimedialen Systemen in gewerblichen und privaten Nutzungskontexten
und adressiert demnach die Herausforderungen der 3rd wave of HCI.

Dieser Schwerpunkt adressiert Kompetenzen, Fähigkeiten und Fertigkeiten
die im Zusammenhang mit der Leitung und dem Management von
Entwicklungsprojekten innovativer, interaktiver Systeme stehen. Dies
umfasst, die Nutzungskontexte in verschiedensten Anwendungsbereichen
kritisch zu analysieren, Problemfelder zu identifizieren, Anforderungen
zu spezifizieren, angemessene Vorgehen zur Lösungsentwicklung zu
konzipieren und Gestaltungslösungen zu entwickeln und zu evaluieren.
Absolventen dieses Schwerpunktes arbeiten als UX-Architects, Interaction
Designer oder in Positionen mit ähnlichen Rollenbezeichnungen in
Unternehmen/Institutionen und sind zentrale Entscheidungsträger, wenn es
um die Entwicklung interaktiver Systeme aus Nutzungs -oder
Nutzerperspektive geht.

Neben den vielfältigen weiterentwickelten Kompetenzen (formale,
analytische, methodologische, gestalterische, technologische, etc.)
haben die Studierenden die Befähigung zum fachlichen Diskurs vertieft
und implementieren mit ihrer Kommunikationskompetenz eine wichtige
Schnittstelle für die verschiedenen Stakeholder und Gewerke.

\paragraph{Multi-Perspective Product
Development\label{/mi-2017/selbstbericht/0400-studiengangskonzept/0000-studiengangskonzept}}\label{multi-perspective-product-developmentpathlabelmi-2017selbstbericht0400-studiengangskonzept0000-studiengangskonzept}

Der Schwerpunkt „Multi-Perspective Product Development'' ist eher
generalistisch ausgeprägt und befähigt die Studierenden neben der
wissenschaftlichen Arbeit zur Mitwirkung in und Konzeption und Führung
von Projekten der Medieninformatik mit ihrer typischen Heterogenität,
welche von methodologischen über technologische bis hin zu
soziotechnischen Aspekten reicht. Chakterisierende Merkmale solcher
Projekte sind:

\begin{itemize}
\tightlist
\item
  Berücksichtigung von und Kommunikation mit Stakeholdern mit jeweils
  eigenen Perspektiven, die durch ihre Fachsprache, Methoden und
  Techniken sowie entsprechende Fähigkeiten, Verantwortlichkeiten und
  Kompetenzen definiert werden.
\item
  Heterogene soziale, technologische und ökonomische Rahmenbedingungen
  wie z.B.
\item
  die Anwendung von unterschiedlichen, agilen bis hin zu
  „schwergewichtigen'' Vorgehensmodellen,
\item
  lokale Zusammenarbeit in kleinen Teams bis hin zu dezentraler
  Zusammenarbeit in großen, international und interdisziplinär
  aufgestellten Teams,
\item
  ein breites Spektrum der Projektgegenstände von kleinen, nativen Apps
  für mobile Geräte bis hin zu großen, geschäftskritischen,
  internationalisierbaren und responsiven Web-Anwendungen,
\item
  ein breites Spektrum der Projektkontexte von kleinen Inhouse-Projekten
  bis hin zu großen, organisationsübergreifenden internationalen
  Projekten.
\end{itemize}

\paragraph{Social
Computing\label{/mi-2017/selbstbericht/0400-studiengangskonzept/0000-studiengangskonzept}}\label{social-computingpathlabelmi-2017selbstbericht0400-studiengangskonzept0000-studiengangskonzept}

Im Schwerpunkt „Social Computing`` werden die Wechselwirkungen zwischen
Gesellschaft und Informatik in den Mittelpunkt gestellt. Rechnersysteme
und Netzwerke werden von Menschen intentional gestaltet, ausgerichtet an
gesellschaftlichen Normen, Prozessen und Bedürfnissen. Gleichzeitig
beeinflussen IT-Systeme diese gesellschaftlichen Normen und verändern
Prozesse in allen Lebensbereichen. Die verantwortungsbewusste Konzeption
und Realisierung von soziotechnischen Systemen (z.B. Social Software,
Online Communities, e-Health, e-Government und e-Learning Angebote)
sowie die empirische Evaluation existierender Systeme sind zentrale
Ziele. Lösungen sollen unter ganzheitlichen Gesichtspunkten entwickelt
werden. Verschiedene Wertvorstellungen und Interessen unterschiedlicher
Stakeholder müssen identifiziert und berücksichtig werden.

Der Schwerpunkt verbindet daher Theorien, Modelle und Methodik der
Human- und Sozialwissenschaften mit anwendungsorientierter Informatik.
Studierende sollen in der Lage sein, computergestützte Systeme nach
ethischen, politischen, sozialen und psychologischen Kriterien bewerten,
planen und umsetzen zu können.

Ziel ist es, soziale Innovation durch digitale Anwendungen entstehen zu
lassen. Neben den empirischen Methoden werden Designmethoden vermittelt,
sowohl auf der konzeptionellen als auch auf der softwaretechnischen
Implementierungsebene, um robuste, sichere und flexible Systeme zu
gestalten.

\paragraph{Visual
Computing\label{/mi-2017/selbstbericht/0400-studiengangskonzept/0000-studiengangskonzept}}\label{visual-computingpathlabelmi-2017selbstbericht0400-studiengangskonzept0000-studiengangskonzept}

Der Studienschwerpunkt „Visual Computing'' steht an der Schnittstelle
von Computergrafik, Computer Vision, Mensch-Maschine-Kommunikation,
Bild- und Videoverarbeitung, sowie Visualisierung.

Ziel des Studienschwerpunktes Visual Computing ist es, den Studierenden
ein solides Fundament bildbasierter und bildgebender Verfahren zu
vermitteln, indem die Entwicklung praktischer Algorithmen und Programme
anhand ihrer theoretischen Grundlagen erlernt wird. Zusätzlich sollen
die Studierenden in die Lage versetzt werden, die entwickelten
Applikationen zu bewerten, zu präsentieren und auf ihre ethischen
Konsequenzen hin zu prüfen.

Die hohe Interdisziplinarität ist ein Innovationsfaktor und bietet
Schlüsseltechnologien zur Lösung aktueller Problemstellungen in der
Informatik, wie z.B. Virtual Engineering, Visual Analytics, Virtual- und
Augmented Reality, Medizintechnik, Robotik, Animation und Bildsynthese.
Anwendungen des Visual Computing finden sich in den verschiedensten
Bereichen, z.B. in der Unterhaltungsindustrie (Visuelle Effekte,
Computerspiele, Filmindustrie, 360° und 3D Videos), der Medizin
(medizinische Bildverarbeitung, digitale Operationsplanung), der
Automobilindustrie (Fahrerassistenzsysteme), der industriellen Fertigung
(visuelle Qualitätskontrolle), der Internettechnologien und Mobilgeräte
(Remote Rendering, Multimediale Datenbanken, Augmented Reality
Anwendungen) und der digitalen Fotografie.

\paragraph{Weaving the
Web\label{/mi-2017/selbstbericht/0400-studiengangskonzept/0000-studiengangskonzept}}\label{weaving-the-webpathlabelmi-2017selbstbericht0400-studiengangskonzept0000-studiengangskonzept}

Im Studienschwerpunkt „Weaving the Web'' wird die Entwicklung von
Produkten und Diensten im Web in den Mittelpunkt gestellt. Dabei wird
der gesamte Lebenszyklus von der Erarbeitung einer Vision, der
eigentlichen Software Entwicklung bis hin zu der Verwertung als Produkt
und/oder Publikation adressiert.

Als charakterisierende Merkmale für die Entwicklung von Produkten und
Diensten im Web stehen:

\begin{itemize}
\tightlist
\item
  die Einbettung in ein Netz von Prozessen und Informationsflüssen, die
  Dienste, Informationen, Personen und Geräte im Web zusammenfassen,
\item
  der Fokus auf Offenheit, sowohl bei den verwendeten Technologien,
  Frameworks und Plattformen als auch die Haltung in der Kommunikation
  im Team und gegenüber der Community und
\item
  die konsequente Anwendung agiler Vorgehensmodelle sowie die Nutzung
  des Wissens und des kreativen Potentials von Nutzern durch Community
  Managenent.
\end{itemize}

Der Titel des Schwerpunkts „Weaving the Web'' wurde gewählt, da neben
dem klassischen Software Engineering vor allem auch die Integration
eigener Produkte und Dienste in das Web thematisiert wird.

\subsection{Studienphasen und
-säulen\label{/mi-2017/selbstbericht/0400-studiengangskonzept/0000-studiengangskonzept}}\label{studienphasen-und--suxe4ulenpathlabelmi-2017selbstbericht0400-studiengangskonzept0000-studiengangskonzept-1}

\begin{figure}[htbp]
\centering
\includegraphics[width=\columnwidth]{../anhaenge/bilder/ma-struktur.png}
\caption{Struktur der ersten drei Studiensemester des Masterprogramms
bei Studienstart im Wintersemester}
\end{figure}

Jedes der ersten drei Fachsemester steht unter einer übergreifenden
Leitfrage. Diese Fragen sind, ähnlich wie im Bachelorprogramm, am groben
Ablauf der Produktentwicklung ausgerichtet: ``Vision \& Konzept'',
``Entwicklung'' und ``Assessment/Evaluation, Forschung und Verwertung''.
Die Leitfragen sind vor allem für die Projekte relevant. Das vierte
Fachsemester wird komplett von der Masterthesis ausgefüllt.

Die ersten beiden Studiensemester setzen sich aus drei strukturierenden
Elementen zusammen: Kernmodule, Schwerpunktmodule und Projekt. Die
Kernmodule ``Spezielle Gebiete der Mathematik'', ``Computerethik'' und
``Research Methods'' sind für alle Studierende des Studienprogramms
verbindlich, wogegen die Schwerpunktmodule abhängig vom jeweiligen
Schwerpunkt sind.

Im dritten Semester können die Studierenden drei Module aus dem Katalog
aller Module der Informatik Masterstudiengänge der Fakultät wählen und
damit ihre Studieninhalte entsprechend ihres Schwerpunkts und der
persönlichen Neigung ausprägen.

Jedes der ersten drei Fachsemester hat ein dezidiertes Projekt, das auf
die jeweilige Leitfrage ausgerichtet ist. An den Projekten können
Studierende aus den verschiedenen Schwerpunkten teilnehmen, um hier eine
Fragestellung gezielt aus verschiedenen Perspektiven zu beleuchten. Der
Grundgedanke ist, dass in jedem Semester nur die Phase der
entsprechenden Leitfrage durchlaufen und so abgeschlossen und
aufbereitet wird, das im nächsten Semester oder zu einem späteren
Zeitpunkt eine anderes Team auf den Ergebnissen aufsetzen kann. Diese
Herangehensweise hat folgende Vorteile: - Bearbeitung von komplexen
(Forschungs-) Fragestellungen - Erleben und Einüben von sequentiellen
arbeitsteiligen Prozessschritten - Sensibilisierung für professionelle
Ergebnissicherung - Skalierbarkeit von Projekten auf verschiedene Teams
in Abhängigkeit von der jeweiligen Projektphase

Die Projekte können mit zusätzlichen Lehrveranstaltungen, die auf die
jeweilige Leitfrage einzahlen, angereichert werden.

\subsection{Wissenschaftliches
Arbeiten\label{/mi-2017/selbstbericht/0400-studiengangskonzept/0000-studiengangskonzept}}\label{wissenschaftliches-arbeitenpathlabelmi-2017selbstbericht0400-studiengangskonzept0000-studiengangskonzept-1}

Die Fortsetzung des Masterstudiengang durch ein Promotionsstudium ist
eine Möglichkeit, die von einer erheblichen Anzahl unserer Absolventen
wahrgenommen wird. Um dafür die erforderlichen Kompetenzen im
wissenschaftlichen Arbeiten zu erreichen aber auch um Interesse am
wissenschaftlichen Arbeiten zu wecken, wird dieses Thema im Master
konsequent verfolgt. Neben expliziten, auf Vermittlung von
Methodenwissen ausgelegten Modulen wie ``Research Methods'' und
``Computer Ethik'' wird vor allem in den Schwerpunktprojekten
wiisenschaftliches Arbeiten gefordert. Im Projektteil ``Vision und
Konzept'' wird im Rahmen der Veranstaltung ``Advanced Seminar'' wird
Aquisition, Bewertung und Erschließung des aktuellen Forschungsstandes
aus wissenschaftlicher Literatur thematisiert. Im Projektteil
``Umsetzug'' wird eine kritische Diskussion der eingesetzten Methoden
und Architekturen praktiziert. Im Projektteil ``Verwertung'' wird auch
die Publikation der Ergebnisse auf Konferenzen, Workshops oder in
sozialen Medien wit github angestrebt.

\subsection{Internationalisierung\label{/mi-2017/selbstbericht/0400-studiengangskonzept/0000-studiengangskonzept}}\label{internationalisierungpathlabelmi-2017selbstbericht0400-studiengangskonzept0000-studiengangskonzept}

Die inhaltliche Offenheit des dritten Fachsemesters mit drei
Wahlpflichtmodulen und einem Projekt ermöglicht den Studierenden ein
Auslandssemester, da hier die Anrechnung von Studienleistung problemlos
möglich ist.

\subsection{Weiterführende
Dokumente\label{/mi-2017/selbstbericht/0400-studiengangskonzept/0000-studiengangskonzept}}\label{weiterfuxfchrende-dokumentepathlabelmi-2017selbstbericht0400-studiengangskonzept0000-studiengangskonzept-1}

\begin{itemize}
\tightlist
\item
  \href{https://th-koeln.github.io/mi-2017/anhaenge/ma-studienverlaufsplan.pdf}{Studienverlaufsplan
  Medieninformatik Master}
\item
  \href{https://th-koeln.github.io/mi-2017/download/modulbeschreibungen-master.pdf}{Modulhandbuch
  Medieninformatik Master}
\item
  \href{https://th-koeln.github.io/mi-2017/anhaenge/ma-MIMPO_Entwurf_20170218.pdf}{Prüfungsordnung
  Medieninformatik Master(entwurf)}
\item
  \href{https://th-koeln.github.io/mi-2017/anhaenge/ma-Ziele-Module-Matrix-Medieninformatik-Master.pdf}{Ziele-Module-Matrix
  Medieninformatik Master}
\item
  \href{https://th-koeln.github.io/mi-2017/anhaenge/ma-zeugnis.pdf}{Beispielzeugnis und Diploma
  Supplement Medieninformatik Master}
\end{itemize}

\chapter{Studierbarkeit\label{/mi-2017/selbstbericht/0500-studierbarkeit/0000-studierbarkeit}}\label{studierbarkeitpathlabelmi-2017selbstbericht0500-studierbarkeit0000-studierbarkeit}

\section{Bachelor
Medieninformatik\label{/mi-2017/selbstbericht/0500-studierbarkeit/0000-studierbarkeit}}\label{bachelor-medieninformatikpathlabelmi-2017selbstbericht0500-studierbarkeit0000-studierbarkeit}

\subsection{Zugangsvoraussetzungen\label{/mi-2017/selbstbericht/0500-studierbarkeit/0000-studierbarkeit}}\label{zugangsvoraussetzungenpathlabelmi-2017selbstbericht0500-studierbarkeit0000-studierbarkeit}

Als Voraussetzung für die Aufnahme eines Bachelorstudiums
Medieninformatik wird die Fachhochschulreife oder eine als gleichwertig
anerkannte Vorbildung gefordert. Für die nächste Einschreibephase wurde,
aufgrund der großen Nachfrage, eine ortliche Zulassungsbeschränkung für
alle Informatik Bachelorstudiengänge der Fakultät vereinbart.

\subsection{Allgemeine/fachgebundene Hochschulreife,
Fachhochschulreife, einschlägige
Berufserfahrung\label{/mi-2017/selbstbericht/0500-studierbarkeit/0000-studierbarkeit}}\label{allgemeinefachgebundene-hochschulreife-fachhochschulreife-einschluxe4gige-berufserfahrungpathlabelmi-2017selbstbericht0500-studierbarkeit0000-studierbarkeit}

Voraussetzung für den Zugang zum Bachelorstudium ist die
Fachhochschulreife oder eine als gleichwertig anerkannte Vorbildung.

\subsection{Praktika/Berufserfahrung\label{/mi-2017/selbstbericht/0500-studierbarkeit/0000-studierbarkeit}}\label{praktikaberufserfahrungpathlabelmi-2017selbstbericht0500-studierbarkeit0000-studierbarkeit}

keine

\subsection{Fremdsprachenkenntnisse,
Deutschkenntnisse\label{/mi-2017/selbstbericht/0500-studierbarkeit/0000-studierbarkeit}}\label{fremdsprachenkenntnisse-deutschkenntnissepathlabelmi-2017selbstbericht0500-studierbarkeit0000-studierbarkeit}

Fremdsprachenkenntnisse, die über das Maß der durch den schulischen
Abschluss gegebenen Fremdsprachenkenntnisse hinausgehen, sind nicht
gefordert. Die Deutschkenntnisse ausländischer Studierender werden
i.d.R. durch Ablegen der deutschen Sprachprüfung für den Hochschulzugang
(DSH II) oder eine äquivalente Prüfung nachgewiesen; für nähere
Informationen sowie Einzelfallregelungen ist das International Office
der TH Köln zuständig\footnote{\href{https://www.th-koeln.de/internationales/international-office_1986.php}{Website
  International Office}}.

\subsection{Eignungsfeststellung\label{/mi-2017/selbstbericht/0500-studierbarkeit/0000-studierbarkeit}}\label{eignungsfeststellungpathlabelmi-2017selbstbericht0500-studierbarkeit0000-studierbarkeit}

Keine

\section{Master
Medieninformatik\label{/mi-2017/selbstbericht/0500-studierbarkeit/0000-studierbarkeit}}\label{master-medieninformatikpathlabelmi-2017selbstbericht0500-studierbarkeit0000-studierbarkeit}

Die Einschreibung zum Studium erfolgt bezogen auf den gewünschten
Studienschwerpunkt.

\subsection{Zugangsvoraussetzungen\label{/mi-2017/selbstbericht/0500-studierbarkeit/0000-studierbarkeit}}\label{zugangsvoraussetzungenpathlabelmi-2017selbstbericht0500-studierbarkeit0000-studierbarkeit-1}

Als Voraussetzung für die Aufnahme des Studiums wird der erfolgreiche
Abschluss eines Hochschulstudiums in einem Studiengang der Informatik
mit dem Mindestabschlussgrad „Bachelor of Science`` oder eines anderen
einschlägigen Studiengangs gefordert. Ein Studiengang gilt als
einschlägig, wenn dieser Studiengang Informatik-Inhalte von mindestens
75 ECTS beinhaltet. Die Entscheidung über die Einschlägigkeit trifft im
Zweifel der Prüfungsausschuss.

Studienbewerberinnen und -bewerber, die die Qualifikation nach Absatz 1
besitzen und zusätzlich Kenntnisse und Fähigkeiten auf andere Weise als
durch ein Studium erworben haben, sind nach dem Ergebnis der
Einstufungsprüfung gem. § 49 Abs. 12 45 HG in einem entsprechenden
Abschnitt des Studienganges zum Studium zuzulassen, soweit nicht
Regelungen über die Vergabe von Studienplätzen entgegenstehen. Das
Nähere regelt die Einstufungsprüfungsordnung der Technischen Hochschule
Köln.

\subsection{Fremdsprachenkenntnisse,
Deutschkenntnisse\label{/mi-2017/selbstbericht/0500-studierbarkeit/0000-studierbarkeit}}\label{fremdsprachenkenntnisse-deutschkenntnissepathlabelmi-2017selbstbericht0500-studierbarkeit0000-studierbarkeit-1}

Fremdsprachenkenntnisse, die über das Maß der durch den schulischen
Abschluss gegebenen Fremdsprachenkenntnisse hinausgehen, sind nicht
gefordert. Die Deutschkenntnisse ausländischer Studierender werden
i.d.R. durch Ablegen der deutschen Sprachprüfung für den Hochschulzugang
(DSH II) oder eine äquivalente Prüfung nachgewiesen; für nähere
Informationen sowie Einzelfallregelungen ist das International Office
der TH Köln zuständig\footnote{\href{https://www.th-koeln.de/internationales/international-office_1986.php}{Website
  International Office}}.

\section{Struktur\label{/mi-2017/selbstbericht/0500-studierbarkeit/0000-studierbarkeit}}\label{strukturpathlabelmi-2017selbstbericht0500-studierbarkeit0000-studierbarkeit}

Im Anhang sind die Studienverlaufspläne der einzelnen Studiengänge
enthalten, für die eine Akkreditierung beantragt wird. Das Studium
umfasst im Bachelor jeweils insgesamt 180 ECTS Punkte und 144
Semesterwochenstunden. Dies entspricht durchschnittlich 24 SWS je
Semester. Die Inhalte der Module sind in dem entsprechenden
Modulhandbuch dargestellt.

Das Masterstudium umfasst 120 ECTS Punkte bei 48 SWS Präsenzzeit, was
einer durchschnittlichen Präsenzzeit von 16 SWS pro Semester entspricht.

\section{Arbeitslast\label{/mi-2017/selbstbericht/0500-studierbarkeit/0000-studierbarkeit}}\label{arbeitslastpathlabelmi-2017selbstbericht0500-studierbarkeit0000-studierbarkeit}

Die Bachelor- und Masterstudiengänge sind durchgängig mit 30
ECTS-Punkten im Semester durchkalkuliert, was einer Arbeitslast von 900
Stunden pro Semester entspricht. Wenn man ein Semester mit 24 Wochen
veranschlagt, wobei die Prüfungszeit und Prüfungsvorbereitung
mitgerechnet ist, ergibt sich eine Wochenarbeitszeit von 900 h / 24 =
37,5 Stunden. Eine Veranstaltung mit 5 Creditpoints und 4 SWS, 2 SWS
Vorlesung + 2 SWS Übung hat in der Regel einen Arbeitsaufwand von 5 x 30
= 150 Stunden. Bei durchschnittlich 18 Semesterwochen entspricht dies
einem Anteil von 2 h x 18 = 36 Stunden Vorlesung, 2 h x 18 = 36 Stunden
Übung, also 72 Stunden Präsenzanteil und 78 Stunden Selbststudium
inklusive Klausurvorbereitung und Nachbereitung der Präsenzanteile. Dies
entspricht in etwa einer Aufteilung der Gesamtzeit in 50\% für
Präsenzstudium und in 50 \% für Selbststudium.

Die Lehrveranstaltungen des Masterstudiengangs sind mit 6 Creditpoints
ausgestattet, was bei einem Modul mit 4 SWS einem Verhältnis von 40\%
für Präsenzstudium und 60 \% für Selbststudium entspricht.

\section{Leistungspunktesystem\label{/mi-2017/selbstbericht/0500-studierbarkeit/0000-studierbarkeit}}\label{leistungspunktesystempathlabelmi-2017selbstbericht0500-studierbarkeit0000-studierbarkeit}

Die Module der beantragten Studiengänge werden mit ECTS-Punkten
bewertet, um europaweite Vergleichbarkeit gemäß den Bologna-Richtlinien
zu ermöglichen.

\section{Prüfungen\label{/mi-2017/selbstbericht/0500-studierbarkeit/0000-studierbarkeit}}\label{pruxfcfungenpathlabelmi-2017selbstbericht0500-studierbarkeit0000-studierbarkeit}

Viele Fachprüfungen der Bachelorstudiengänge, vor allem der
Grundlagenfächer, werden in Form einer Klausur angeboten. Bei vielen
Pflichtmodulen, den meisten Wahlpflichtfächer und natürlich im
Kolloquium zur Bachelorarbeit sind mündliche Prüfungen vorgesehen, die
oft durch Referate und Präsentationen unterstützt werden. Die Anzahlen
der Modulprüfungen liegen bei den Bachelorstudiengängen zwischen 28 und
30 und sind so über die sechs Semester verteilt, dass es zu keinen
Häufungen mit mehr als sechs Prüfungen in einem Semester kommt.

Im Masterstudium ist der Schwerpunkt der Prüfungsformen in Richtung
mündlicher Prüfungen, Präsentationen und wissenschaftlicher
Ausarbeitungen gelegt. Bei einer Gesamtzahl von 15 Modulprüfungen fallen
maximal fünf Prüfungen pro Semester an.

\section{Studien/Prüfungsordnungen\label{/mi-2017/selbstbericht/0500-studierbarkeit/0000-studierbarkeit}}\label{studienpruxfcfungsordnungenpathlabelmi-2017selbstbericht0500-studierbarkeit0000-studierbarkeit}

Die Studien- und Prüfungsordnungen\footnote{\href{https://www.th-koeln.de/studium/medieninformatik-bachelor--ordnungen-und-formulare_3963.php}{Prüfungsordnung
  Medieninformatik Bachelor}}\footnote{\href{https://th-koeln.github.io/mi-2017/anhaenge/ma-MIMPO_Entwurf_20170218.pdf}{Prüfungsordnung
  Medieninformatik Master (Entwurf)}} der laufenden Studiengänge sind
dem Anhang dieses Berichts beigefügt. Sie sind außerdem über die Website
der Hochschule abrufbar. Der Studienverlaufsplan entspricht der
Studienordnung. Nach Zustimmung der Gutachter zu den in den erläuterten
Änderungen im Rahmen der Reakkreditierung werden die überarbeiteten
Prüfungsordnungen, bzw. Studienverlaufspläne zeitnah vorgelegt.

\section{Diploma
Supplement\label{/mi-2017/selbstbericht/0500-studierbarkeit/0000-studierbarkeit}}\label{diploma-supplementpathlabelmi-2017selbstbericht0500-studierbarkeit0000-studierbarkeit}

Das Diploma Supplement der zur Reakkrediterung beantragten Studiengänge
ist im Anhang des Dokuments zu finden.

\section{Maßnahmen zur Beratung von Studieninteressierten und
Studierenden\label{/mi-2017/selbstbericht/0500-studierbarkeit/0000-studierbarkeit}}\label{mauxdfnahmen-zur-beratung-von-studieninteressierten-und-studierendenpathlabelmi-2017selbstbericht0500-studierbarkeit0000-studierbarkeit}

Die Medieninformatik beteiligt sich an folgenden Veranstaltungen zur
Beratung von Studieninteressierten:

Regelmäßig wird im Mai ein „Schnupperstudium`` durchgeführt, an dem rund
150 Schüler, teilweise mit ihren Lehrern teilnehmen, um die Hochschule
kennen zu lernen.

Das Medieninformatik beteiligt sich regelmäßig mit eigenen
Veranstaltungen an dem bundesweit jährlich stattfindenden Girls-Day, an
dem rund 50 Schülerinnen speziell für ein Informatik-Studium oder ein
ingenieurwissenschaftliches Studium in Gummersbach begeistert werden
sollen.

Dazu kommen Laborführungen für Schülergruppen verschiedener Schulen
sowie die Präsentation des Campus Gummersbach außerhalb der Hochschule:
- auf der „Overather Ausbildungsbörse``, - der „Ausbildungsbörse
Bergneustadt'', - der „Mädchenmesse" des Oberbergischen Kreises, - dem
„Tag der Offenen Tür" des Berufskollegs Dieringhausen (Gummersbach),der
„Weiterbildungsmesse Oberberg``, - sowie die Teilnahme an anderen,
unregelmäßig durchgeführten Veranstaltungen zur Studien- und Berufswahl.

Das Institut für Informatik beteiligt sich jährlich am „Tag der offenen
Tür`` der TH-Köln im September und an Informationsveranstaltungen der
umliegenden Gymnasien und anderer weiterführender Schulen, die
potenzielle Studienanfängerinnen und Studienanfänger an die
Qualifizierung für ein Hochschulstudium heranführen. Alle diese Angebote
werden sehr gut aufgenommen und sind stark frequentiert.

Darüber hinaus bietet die Medieninformatik einige Veranstaltungen (siehe
außercurriculare Maßnahmen) wie den jährlichen Showcase an, um hier auch
eine Plattform für Studieninteressierte zu schaffen. Diese Angebote
werden gut angenommen.

\chapter{Prüfungssystem\label{/mi-2017/selbstbericht/0600-pruefungssystem/0000-pruefungssystem}}\label{pruxfcfungssystempathlabelmi-2017selbstbericht0600-pruefungssystem0000-pruefungssystem}

\section{Prüfungsprozesse\label{/mi-2017/selbstbericht/0600-pruefungssystem/0000-pruefungssystem}}\label{pruxfcfungsprozessepathlabelmi-2017selbstbericht0600-pruefungssystem0000-pruefungssystem}

Der Campus Gummersbach der TH Köln hat ganzheitliches
Qualitäts-Management-System nach ISO 9001:2008 umgesetzt. Die
beschriebenen Prozesse sind verbindlich für alle Mitarbeiterinnen und
Mitarbeiter. Die Prozessbeschreibung beinhaltet die Vorgehensweisen,
Teilprozesse, Schnistellen, mitgeltenden Informationen, Zuständigkeiten
und Verantwortlichkeiten für die einzelnen Prozesse. Darüber sind auch
alle prüfungsrelvanten Prozesse abgebildet. Die folgenden
Prozessdokumentationen können diesbezüglich eingesehen werden.

\begin{itemize}
\tightlist
\item
  \href{https://th-koeln.github.io/mi-2017/anhaenge/Prozessbeschreibung_PruefungsprozessUeberblick.pdf}{Überblick
  über den Prüfungsprozess}
\item
  \href{https://th-koeln.github.io/mi-2017/anhaenge/Prozessbeschreibung_Pruefungsordnungsprozess.pdf}{Prüfungsordnungsprozess}
\item
  \href{https://th-koeln.github.io/mi-2017/anhaenge/Prozessbeschreibung_Pruefungplanung.pdf}{Prozess zu
  Prüfungsplanung}
\item
  \href{https://th-koeln.github.io/mi-2017/anhaenge/Prozessbeschreibung_Pruefungdurchfuehren.pdf}{Prozess
  zu Prüfunungsdurchführung}
\item
  \href{https://th-koeln.github.io/mi-2017/anhaenge/Prozessbeschreibung_vonPruefungzuruecktreten.pdf}{Prozess
  beim Rücktritt von Prüfungen}
\item
  \href{https://th-koeln.github.io/mi-2017/anhaenge/Prozessbeschreibung_EinspruchPruefungsergebnisse.pdf}{Prozess
  beim Einspruch gegen Prüfungsergebnisse}
\end{itemize}

\section{Studien/Prüfungsordnungen\label{/mi-2017/selbstbericht/0600-pruefungssystem/0000-pruefungssystem}}\label{studienpruxfcfungsordnungenpathlabelmi-2017selbstbericht0600-pruefungssystem0000-pruefungssystem}

Die Prüfungsordnungen regeln das Studium und die Prüfungen in den
Studiengängen Medieninformatik Bachelor und Medieninformatik Master an
der TH Köln. Auf der Grundlage dieser Prüfungsordnungen erstellt die TH
Köln einen Studienverlaufsplan und ein Modulhandbuch. Der
Studienverlaufsplan dient als Empfehlung an die Studierenden für einen
sachgerechten Aufbau des Studiums. Das Modulhandbuch beschreibt Inhalt
und Aufbau des Studiums unter Berücksichtigung der wissenschaftlichen
und hochschuldidaktischen Entwicklung und der Anforderungen der
beruflichen Praxis. Die Prüfungsordnungen und Studienverlaufspläne
werden über die Website der TH Köln zugänglich gemacht.

\begin{itemize}
\tightlist
\item
  \href{https://www.th-koeln.de/studium/medieninformatik-bachelor--ordnungen-und-formulare_3963.php}{Prüfungsordnung
  Medieninformatik Bachelor}
\item
  \href{https://th-koeln.github.io/mi-2017/anhaenge/ba-studienverlaufsplan.pdf}{Studienverlaufsplan
  Medieninformatik Bachelor}
\item
  \href{https://th-koeln.github.io/mi-2017/download/modulbeschreibungen-bachelor.pdf}{Modulhandbuch
  Medieninformatik Bachelor}
\item
  \href{https://th-koeln.github.io/mi-2017/anhaenge/ma-MIMPO_Entwurf_20170218.pdf}{Prüfungsordnung
  Medieninformatik Master (Entwurf)}
\item
  \href{https://th-koeln.github.io/mi-2017/anhaenge/ma-studienverlaufsplan.pdf}{Studienverlaufsplan
  Medieninformatik Master}
\item
  \href{https://th-koeln.github.io/mi-2017/download/modulbeschreibungen-master.pdf}{Modulhandbuch
  Medieninformatik Master}
\end{itemize}

\section{Prüfungsplanung\label{/mi-2017/selbstbericht/0600-pruefungssystem/0000-pruefungssystem}}\label{pruxfcfungsplanungpathlabelmi-2017selbstbericht0600-pruefungssystem0000-pruefungssystem}

Die Prüfungsplanung wird über das Hochschulplanungssystem abgebildet.

\begin{itemize}
\tightlist
\item
  \href{https://th-koeln.github.io/mi-2017/anhaenge/ba-pruefungsplan_mi.pdf}{Beispielhafter
  Prüfungsplan}
\end{itemize}

\section{Prüfungsstatistiken\label{/mi-2017/selbstbericht/0600-pruefungssystem/0000-pruefungssystem}}\label{pruxfcfungsstatistikenpathlabelmi-2017selbstbericht0600-pruefungssystem0000-pruefungssystem}

Zur Analyse der Prüfungserfolge und der Lernergebnisorientierung liegen
eine Reihe von statistischen Daten vor. Eine wesentliche Analyse auf
Basis dieser Daten ist die Überprüfung der Studierbarkeit. Als Indikator
für eine gute Studierbarkeit, kann die Anzahl der abgelegten Prüfungen
im vorgesehenen Fachsemester des Moduls angesehen werden. Ziel ist es,
dass die Studierenden Prüfungen möglichst im selben Semester ablegen, in
dem das Modul im Studienverlaufsplan verortet ist. Gelingt dies nicht,
so kann ein Studienabschluss innerhalb der Regelstudienzeit nur schwer
realisiert werden.

\begin{itemize}
\tightlist
\item
  \href{https://th-koeln.github.io/mi-2017/anhaenge/ba-pruefungsplan_mi.pdf}{Analyse der
  Prüfungsteilnahme}
\item
  \href{https://th-koeln.github.io/mi-2017/anhaenge/ba-pruefungen-fehlversuche-und-ruecktritte.pdf}{Tabelle
  über Fehlversuche und Rücktritte}
\end{itemize}

\chapter{Studiengangsbezogene
Kooperationen\label{/mi-2017/selbstbericht/0700-studiengangsbezogene-kooperationen/0000-studiengangsbezogene-kooperationen}}\label{studiengangsbezogene-kooperationenpathlabelmi-2017selbstbericht0700-studiengangsbezogene-kooperationen0000-studiengangsbezogene-kooperationen}

\section{Hochschulinterne
Zusammenarbeit\label{/mi-2017/selbstbericht/0700-studiengangsbezogene-kooperationen/0000-studiengangsbezogene-kooperationen}}\label{hochschulinterne-zusammenarbeitpathlabelmi-2017selbstbericht0700-studiengangsbezogene-kooperationen0000-studiengangsbezogene-kooperationen}

\subsection{Fakultätsübergreifende
Zusammenarbeit\label{/mi-2017/selbstbericht/0700-studiengangsbezogene-kooperationen/0000-studiengangsbezogene-kooperationen}}\label{fakultuxe4tsuxfcbergreifende-zusammenarbeitpathlabelmi-2017selbstbericht0700-studiengangsbezogene-kooperationen0000-studiengangsbezogene-kooperationen}

Innerhalb der Hochschule wird eine enge Kooperation mit einigen in Köln
angesiedelten Fakultäten gepflegt. Zu nennen sind hier vor allem die
Bereiche Design, Informations-, Medien- und Elektrotechnik, Wirtschafts-
und Rechtswissenschaften, Informations- und
Kommunikationswissenschaften, Architektur, Sporthochschule sowie
angewandte Sozialwissenschaften.

Mit der „Köln International School of Design`` als interne Einrichtung
der „Fakultät für Kulturwissenschaften`` (Fakultät 02) und der „Fakultät
für angewandte Sozialwissenschaften`` (Fakultät 01) werden gemeinsame
Lehrveranstaltungen durchgeführt. Das Modul „Wissensmanagement`` des
Masterstudiengangs bspw. wurde bereits mehrfach von der „Fakultät für
Wirtschafts- und Rechtswissenschaften`` der TH Köln (Fakultät 04)
importiert.

Die Errichtung eines kooperativen Studiengangs eScience ist geplant.

Ein gemeinsamer Studiengang ``Code \& Context'' mit der Fakultät für
Informations-, Medien- und Elektrotechnik sowie der KISD ist in Planung,
der sich u.A. aus Inhalten der zu reakkreditierenden Studiengänge speist
und mit dem entsprechende Synergien geschöpft werden sollen.

Das Institut für Informatik und das Institut für Automation \& IT sind
ferner im Forschungsschwerpunkt COSA synergetisch verbunden.

\subsection{Fakultätsinterne
Zusammenarbeit\label{/mi-2017/selbstbericht/0700-studiengangsbezogene-kooperationen/0000-studiengangsbezogene-kooperationen}}\label{fakultuxe4tsinterne-zusammenarbeitpathlabelmi-2017selbstbericht0700-studiengangsbezogene-kooperationen0000-studiengangsbezogene-kooperationen}

Innerhalb der Fakultät sind die Institute „Betriebswirtschaftliches
Institut Gummersbach (BIG)`` und das „Institut für Distance Learning \&
Further Education (IDF)`` mit verschiedenen Modulen in die Bachelor- und
Masterstudiengängen involviert.

Innerhalb der Fakultät 10 für Informatik und Ingenieurwissenschaften
besteht naturgemäß in der Lehre, Forschung und Entwicklung eine enge
Zusammenarbeit mit den in Gummersbach angesiedelten
ingenieurwissenschaftlichen Instituten und Forschungsschwerpunkten. Dies
drückt sich in einer Vielzahl von gemeinsamen Projekten, betreuten
Abschlussarbeiten sowie einem fachübergreifenden Lehrexport und Import
zwischen den beiden Lehreinheiten aus.

\section{Externe Kooperation mit Hochschulen und
Firmen\label{/mi-2017/selbstbericht/0700-studiengangsbezogene-kooperationen/0000-studiengangsbezogene-kooperationen}}\label{externe-kooperation-mit-hochschulen-und-firmenpathlabelmi-2017selbstbericht0700-studiengangsbezogene-kooperationen0000-studiengangsbezogene-kooperationen}

\subsection{Kooperationen mit nationalen
Hochschulen\label{/mi-2017/selbstbericht/0700-studiengangsbezogene-kooperationen/0000-studiengangsbezogene-kooperationen}}\label{kooperationen-mit-nationalen-hochschulenpathlabelmi-2017selbstbericht0700-studiengangsbezogene-kooperationen0000-studiengangsbezogene-kooperationen}

Seit mehreren Jahren besteht im Rahmen von Promotionsvorhaben eine einge
Kooperation mit der Universität Paderborn, s-lab, Prof.Dr.~Gregor
Engels. Es werden regelmäßig Lehrveranstaltungen von promovierenden
Mitarbeitern durchgeführt. Weitere Kooperationen insbesondere im Bereich
der Qualitätssicherung bestehen mit der Hochschule Bremen,
Prof.~Dr.~Andreas Spillner, und der Hochschule Bremerhaven,
Prof.~Dr.~Karin Vosseberg. Im Bereich Forschendes und Projektbasiertes
Lehren und Lernen besteht eine Kooperation mit der
Humanwissenschaftlichen Fakultät der Universität Köln, Dr.~Dirk Rohr.

\subsection{Kooperationen mit internationalen
Hochschulen\label{/mi-2017/selbstbericht/0700-studiengangsbezogene-kooperationen/0000-studiengangsbezogene-kooperationen}}\label{kooperationen-mit-internationalen-hochschulenpathlabelmi-2017selbstbericht0700-studiengangsbezogene-kooperationen0000-studiengangsbezogene-kooperationen}

Erste Kooperationsprojekte mit ausländischen Hochschulen datieren auf
den Beginn der 80-iger Jahre. Damals wurde eine Kooperation
(Erasmus-Kontrakt) mit der École Centrale de Lille abgeschlossen. Unter
den gleichen formalen Bedingungen existiert seit einigen Jahren eine
Kooperation mit Institut National Polytechnique de Grenoble und der
École de l'expertise informatique (EPITECH) mit denen ein regelmäßiger
Studierenden und Dozentenaustausch stattfindet.

Seit 1994 existiert die Partnerschaft mit der staatlichen Universität
für das Verkehrswesen in Moskau (Moskowskij Gosudarstwennyi Universitet
Putej Soobschtschenija -- kurz MIIT). Bisher wurden über 20 russische
Studierende und Doktoranden i.d.R. in 1-jährigen Studien-, Praxis- und
Forschungsaufenthalte durch die Fakultät betreut. Umgekehrt sind bisher
ca. 10 deutsche Studierende und wissenschaftliche Mitarbeiter an die
russische Partnerhochschule zwecks Durchführung von Studien- und
Forschungsprojekten bzw. Kurzzeitdozenturen gegangen.

Als Partner der Fakultät für Wirtschaftswissenschaften (Fakultät 04) der
TH Köln hat das Institut für Informatik entscheidend am Aufbau eines
Studiengangs für Wirtschaftsinformatik an der Staatlichen Akademie für
das Bauwesen in Nishnij Novgorod, Russland mitgewirkt. Hieraus
resultierten mehrere Austauschprojekte auf Studierenden- und
Hochschullehrerebene.

2003 wurde ein Partnerschaftsabkommen mit der Ho Tschi Minh Universität
in Saigon, Vietnam geschlossen. Ein regelmäßiger Austausch von
Professoren findet statt.

Seit Mitte der 90-iger Jahre existiert eine formelle Partnerschaft mit
der University of Clemson in South Carolina, USA. Hier werden regelmäßig
Studierende nach USA zwecks Anfertigung von Abschlussarbeiten entsandt.
Angestrebt werden ferner Kooperationen mit der University of Maryland
und der University of Austin, Texas.

Die Universidad de Burgos (Spanien) ist seit Ende 2008 Partnerhochschule
des Instituts für Informatik der TH Köln. Ziel der Partnerschaft ist
einerseits ein regelmäßiger Studierenden und Dozentenaustausch; so fand
in der Zeit vom 6. Juli bis zum 19. Juli in Burgos eine ``Summer
School'' mit 42 deutschen und spanischen Studierenden zum Thema ``WEB \&
Information Management in a Modern World'' statt, der von der TH Köln
seitens Prof.~Dr.~Heide Faeskorn-Woyke, Prof.~Dr.~Stefan Karsch und
Prof.~Dr.~Hans Ludwig Stahl sowie von der Hochschule Burgos seitens
Prof.~Dr.~Ana Maria Lara Palma und Prof.~Dr.~Emilio Corchado organisiert
und geleitet wurde. Andererseits dient die Partnerschaft der
Durchführung kooperativer Promotionsvorhaben; Ende 2009 wurden die
Promotionsvorhaben zweier wissenschaftlicher Mitarbeiter des Instituts
für Informatik offiziell gestartet.

Mit der UEM (Universidad Europea de Madrid) wird das ERASMUS-Abkommen
genutzt, um Studierenden ein Studiensemester in Madrid und umgekehrt
auch in Gummersbach anzubieten. Diverse studentische Gruppen haben
seitdem an dem Austausch teilgenommen zwischen den beiden Standorten.

Derzeit läuft ein internationales Web-Development Projekt mit
Studierenden der Medieninformatik und Studierenden der Université Paris
13, Technological Educational Institute of Athens, Universidade de
Coimbra und Technical University of Sofia.

Im folgenden findet sich eine komplette Liste mit aktuellen
internationalen Kooperationen mit anderen Hochschulen:

\begin{itemize}
\tightlist
\item
  Bangladesch, Rajshahi University\\
\item
  Belgien, Odisee vzw
\item
  Bulgarien, VARNENSKI SVOBODEN UNIVERSITET ``CHERNORIZETS HRABAR''
\item
  Bulgarien, Technical University of Sofia
\item
  Frankreich, INSTITUT NATIONAL POLYTECHNIQUE DE GRENOBLE\\
\item
  Frankreich, EPITECH- L'ECOLE DE L'EXPERTISE INFORMATIQUE
\item
  Frankreich, Université Paris 13
\item
  Griechenland, PANEPISTIMIO IOANNINON
\item
  Griechenland, Technological Educational Institute of Athens
\item
  Japan, Kobe Institut\\
\item
  Jordanien, German-Jordanian University\\
\item
  Kroatien, SVEUČILIŠTE U SPLITU
\item
  Litauen, Siaures Lietuvos Kolegija
\item
  Mexiko, Instituto Politécnico Nacional (IPN)\\
\item
  Mexiko, Instituto Tecnológico y de Estudios Superiores de Monterrey
  (ITESM)\\
\item
  Niederlande, Fontys Hogescholen
\item
  Niederlande, UNIVERSITEIT LEIDEN\\
\item
  Norwegen, Westerdals Hogskole - Oslo School of Arts, Communication and
  Technology AS
\item
  Österreich, CAMPUS 02 Fachhochschule der Wirtschaft
\item
  Österreich, Fachhochschule Oberösterreich
\item
  Polen, Higher Vocational School in Krosno
\item
  Portugal, Universidade de Coimbra
\item
  Rumänien, UNIVERSITATEA DIN CRAIOVA
\item
  Russische Föderation, Moskauer Staatliche Universität für
  Verkehrswesen\\
\item
  Schweden, Kungliga Tekniska högskolan (KTH)
\item
  Schweiz, ``HES-SO Haute École Spécialisée de Suisse Occidentale Haute
  école de gestion de Genève''
\item
  Serbien, University of Novi Sad\\
\item
  Spanien, Universidad de Burgos
\item
  Spanien Universidad de León
\item
  Spanien Universidad Europea de Madrid
\item
  Spanien UNIVERSITAT DE VALENCIA (ESTUDI GENERAL) UVEG
\item
  Südkorea, Chonbuk National University\\
\item
  Südkorea, University of Seoul\\
\item
  Südkorea, Hanyang University\\
\item
  Türkei, ISTANBUL ÜNIVERSITESI\\
\item
  Türkei, Istanbul Teknik Üniversitesi\\
\item
  Türkei, Marmara Üniversitesi
\item
  Türkei, Isik Üniversitesi
\item
  Türkei, MALTEPE ÜNIVERSITESI
\item
  Vietnam, Ho Tschi Minh Universität
\end{itemize}

Insgesamt absolvieren durchschnittlich 10 Studenten Praktika
(Praxissemester) im Ausland, durch Erasmus--Programme werden ca. 20
Studenten jährlich unterstützt, die entweder nach Gummersbach kommen
oder ein Semester im Ausland verbringen. Mit den oben angegebenen
Hochschulen bestehen Erasmus-Kontakte und andere Partnerschaftsabkommen,
um dem Austausch einen formalen Rahmen zu geben.

\subsection{Firmen
Kooperationen\label{/mi-2017/selbstbericht/0700-studiengangsbezogene-kooperationen/0000-studiengangsbezogene-kooperationen}}\label{firmen-kooperationenpathlabelmi-2017selbstbericht0700-studiengangsbezogene-kooperationen0000-studiengangsbezogene-kooperationen}

Durch zwei jährlich stattfindenden Veranstaltungen im Studiengang
Medieninformatik werden immer wieder neue Kontakte zu Unternehmen und
Institutionen geknüpft: - Bei dem jeweils zu Beginn des Wintersemresters
stattfindenden ``Medieninformatik Showcase'' werden herausagende
studentische Projektarbeiten präsentiert und es wird über Keynotes und
Diskussionsveranstaltungen der Kontakt zu Unternehmen gepflegt. - Bei
der jeweils zum Ende des Wintersemesters stattfindenden
``Medieninformatik Projektbörse'' stellen ausgewählte Unternehmen
Kopperationsmöglichkeiten dar und es entstehen in der Regel zunächst
durch die gemeinsame Betreuung von Abschlussarbeiten so immer wieder
neue Kontakte.

Zu anderen Hochschulen oder Institutionen bestehen im Bereich der
Informatik Verbindungen. Für die Medieninformatik von besonderer
Bedeutung sind die Verbindungen zur Kunsthochschule für Medien in Köln,
zum Frauenhofer Institut in Schloss Birlinghofen und zu einigen Firmen
aus dem RTL Firmenverbund sowie zum WDR. Seit 2005 lobt das Cologne
Broadcasting Center der RTL Gruppe (CBC) jährlich den CBC-Preis aus, mit
denen drei Abschlussarbeiten aus den Studiengängen der Medieninformatik
prämiert werden.

Das ``IT-Forum Oberberg e.V.'' ist eine Initiative und ein
Zusammenschluss interessierter - vorwiegend Oberbergischer- Unternehmen
und Gewerbetreibender der IT-Branche (IT-Anbieter und -Nachfrager), der
Industrie- und Handelskammer zu Köln - Zweigstelle Oberberg, sowie
Bildungsträgern wie der Technischen Hochschule Köln - Campus Gummersbach
und dem Berufskolleg des Oberbergischen Kreises. Es hat mittlerweile 56
Mitglieder und veranstaltet regelmäßig Leistungsschauen, an denen sich
das Institut für Informatik beteiligt.

Die Bachelorarbeiten und Master-Thesen werden auf praktische
Themenstellungen mit Forschungsbezug aus Unternehmen oder auf
Aufgabenstellungen aus den Forschungsaktivitäten am Institut für
Informatik ausgerichtet. Hier kann auch eine langjährige Zusammenarbeit
mit rheinischen Unternehmen wie der Telekom, Vodavone, der Deutschen
Post, Bayer Leverkusen und Kölner Unternehmen wie RTL, dem WDR, dem LMR,
der Nuro-Media GmbH oder Metafusion verwiesen werden, bei denen eine
Vielzahl von Abschlussarbeiten aus dem Bachelor und Masterstudiengang
Medieninformatik stattgefunden haben. Zudem wurde eine Vielzahl von
Projekt- und Abschlussarbeiten bei dem Broadcast Center Europe (BCE) in
Luxemburg, einem Mitglied der RTL-Gruppe, durchgeführt.

Darüber hinaus findet sich im Personalhandbuch\footnote{\href{http://bit.ly/2mpcbWN}{Personalhandbuch
  der des Instituts für Informatik}} eine Vielzahl von Hinweisen
einzelner Kolleginnen und Kollegen darüber, mit welchen Firmen sie
kooperieren. Im Rahmen von Abschlussarbeiten und Projektarbeiten finden
sich so ein Vielzahl regionaler Firmen bei denen Abschlussarbeiten und
Projektarbeiten bereits in erfolgreicher Kooperation durchgeführt
wurden, so beispielsweise die Cologne Broadcasting Company (CBC), Inovex
GmbH, CLAAS, Telexiom AG oder Miltenyi Biotec GmbH.

Seitens der »Nachwuchsförderung« kooperiert die Fakultät 10 mit
zahlreichen Gymnasien und Berufskollegs in der Region.

\chapter{Personal \&
Ausstattung\label{/mi-2017/selbstbericht/0800-ausstattung/0000-ausstattung}}\label{personal-ausstattungpathlabelmi-2017selbstbericht0800-ausstattung0000-ausstattung}

\section{Weiterbildung\label{/mi-2017/selbstbericht/0800-ausstattung/0000-ausstattung}}\label{weiterbildungpathlabelmi-2017selbstbericht0800-ausstattung0000-ausstattung}

Das Kompetenzteam Hochschuldidaktik der TH Köln bietet für alle
Lehrenden hochschuldidaktische Fort- und Weiterbildungen in vielfältiger
Form an. Um die Lern- und Lehrkultur stetig weiter zu entwickeln wurden
eine Vielzahl von Programmen (z.B. ProfiL2, Exzellente Lehre, Come
in-Commit, etc.) entwickelt, die fest in den Hochschulalltag integriert
sind. Details zu den vielfältigen Aktivitäten und Ressourcen finden sich
beschrieben unter:
https://www.th-koeln.de/hochschule/lehr--und-lernkultur\_6277.php. Das
Weiterbildungsprogramm wurde von den Lehrenden vielfältig wahrgenommen,
unter anderem:

\begin{itemize}
\tightlist
\item
  Workshop ``Problembasiert und projektorientiert lehren und lernen'',
  (Netzwerk Hochschuldidaktische Weiterbildung, 2012)
\item
  Workshop ``Lernportfolios als Semesterbericht und Prüfungsformat'',
  (Netzwerk Hochschuldidaktische Weiterbildung, 2014)
\item
  Workshop ``Lehrstörung meets Improtheater - mit Impro und Humor neue
  Perspektive und Lösungen für schwierige Lernsituationen entdecken''
  (Netzwerk Hochschuldidaktische Weiterbildung, 2015)
\item
  Workshop ``Lehren - Lernen - Prüfen'' (Netzwerk Hochschuldidaktische
  Weiterbildung, 2015)
\item
  1/2-jähriger Workshop zum Fakultätsmultiplikator
  ``Kompetenzorientierte Prüfungen entwickeln'' (TH Köln 2014)
\item
  Workshop ``Prüfen von Kompetenzen in projektorientierten
  Lehrveranstaltungen'' (TH Köln 2014)
\item
  Workshop ``CU in Projects - Inspirierendes Lehren und Lernen'' (TH
  Köln, 2015)
\item
  Workshop ``Forschendes Lernen und Prüfen'' (TH Köln 2016)
\item
  Workshop ``Flipped Classroom in Theorie und Praxis'' (TH Köln, 2015)
\item
  Workshop ``Lehrportfolio Werkstatt'' (TH Köln, 2015)
\item
  Workshop ``Visual Basics'' (TH Köln, 2016)
\item
  Workshop ``Sichtbar'' (TH Köln, 2016)
\end{itemize}

Daneben haben Lehrende und wissenschaftliche Mitarbeiter an
wissenschaftlichen Konferenzen teilgenommen. Unter anderem:

\begin{itemize}
\tightlist
\item
  International ACM Web Science Conference '15 in Oxford (K. Fischer mit
  paper)
\item
  GI-Konferenz Software Engineering 2012 in Berlin, 2013 in Aachen und
  2017 in Hannover (M. Winter mit paper)
\item
  6th+7th International Workshop on Model-Based Verification and
  Validation (MVV 2016 und 2017, Wien und Prag, M. Winter program
  comittee)
\item
  World Usability Days: Sustainable User Experience (UX) 2016 in Köln
  (G. Hartmann Tagungsleitung)
\item
  Internationales WebDev-Projekt 2016 - Studentisches Web-Projekt in
  Kooperation mit europäischen Hochschulen (C. Noss Organisation)
\item
  Eurographics Conference on Rendering Techniques (EGSR) 2016 in Dublin
  (M. Eisemann best paper award)
\item
  Eurographics 2015 in Zürich (M. Eisemann mit paper)
\item
  Eurographics 2016 in Lissabon (M. Eisemann mit paper)
\item
  HiGraphics Workshop 2016 und 2017 (M. Eisemann)
\item
  Siggraph 2015 (M. Eisemann mit paper)
\item
  Vision, Modeling, and Visualization Conference 2016 (M. Eisemann best
  paper award)
\item
  FabCon 3.D und Rapid.Tech 2016 (M. Eisemann)
\item
  Mensch und Computer Konferenz 2016 (C. Noss und M. Eisemann)
\item
  NordiCHI '16 Workshop Designing e-Health Services for Patients \&
  Relatives 2016 GothenburgSweden (C. Grünloh mit paper)
\item
  Medical Informatics Europe (MIE 2016), in conjunction with Health -
  Exploring Complexity: An Interdisciplinary Systems Approach (HEC 2016)
  2016 in München (C. Grünloh mit paper)
\end{itemize}

\section{Räumliche Ausstattung und
Hardware\label{/mi-2017/selbstbericht/0800-ausstattung/0000-ausstattung}}\label{ruxe4umliche-ausstattung-und-hardwarepathlabelmi-2017selbstbericht0800-ausstattung0000-ausstattung}

\subsection{Verleih\label{/mi-2017/selbstbericht/0800-ausstattung/0000-ausstattung}}\label{verleihpathlabelmi-2017selbstbericht0800-ausstattung0000-ausstattung}

\begin{itemize}
\tightlist
\item
  12 x Audiovisuelle Produktionssets, bestehend aus jeweils P2 Panasonic
  HD Kamera Sachtler Kamerastativ Sennheiser Richtmikrofon und mobilen
  Audiomischer
\item
  4 x kompakte Panasonic HD Kameras
\item
  1 x Canon EOS 5D Mark III mit verschiedenen Wechselobjektiven:
\item
  1 x GoPro Hero 3 Black Edition
\item
  1 x DJI Ozmo Gimbal Kamera
\item
  9 x Lichtset im Koffer, mit jeweils 3 x 750 W ARRI Scheinwerfer
\item
  Diverses Zubehör für Licht, Ton und Video wie Lichtstative,
  Fieldmonitore, Speichermedien
\item
  Verschiedene Smartphones und Tablets: Nexus 5, Nexus 9 und iPad Pro
  13''
\item
  1 Oculus Rift VR Virtual Reality Head-mounted Display
\item
  1 HTC Vive Virtual Reality Head-mounted Display
\end{itemize}

\subsection{Nachbearbeitung\label{/mi-2017/selbstbericht/0800-ausstattung/0000-ausstattung}}\label{nachbearbeitungpathlabelmi-2017selbstbericht0800-ausstattung0000-ausstattung}

\begin{itemize}
\tightlist
\item
  5 x Mac Pro stationär, mit Adobe Production Suite CS 6
\item
  12 x iMac mobil, mit Adobe Production Suite CS 6
\item
  1 x Tonkabine mit Neumann Großmembranmikrofon, Mac Pro mit Logic X und
  Mackie 1402 Tonmischpult
\end{itemize}

\subsection{Studio\label{/mi-2017/selbstbericht/0800-ausstattung/0000-ausstattung}}\label{studiopathlabelmi-2017selbstbericht0800-ausstattung0000-ausstattung}

\begin{itemize}
\tightlist
\item
  Greenbox mit festmontierter und variabler Beleuchtung
\item
  Bildmischer Panasonic AV-HS400A und Audiomischer Behringer
  AB1222FX-Pro
\item
  Mac Pro mit Adobe Production Suite CS 6 zur Digitalisierung und
  Nachbearbeitung der Studioproduktionen
\end{itemize}

\subsection{MI-Projektraum\label{/mi-2017/selbstbericht/0800-ausstattung/0000-ausstattung}}\label{mi-projektraumpathlabelmi-2017selbstbericht0800-ausstattung0000-ausstattung}

\begin{itemize}
\tightlist
\item
  Eyetracking System (SMI-Vision, 120 Hz) mit Laptop und
  Auswertungssoftware
\item
  Eye-Tracking Brille (Tobii) für mobile Nutzungskontexte
\end{itemize}

\subsection{Innovationsräume\label{/mi-2017/selbstbericht/0800-ausstattung/0000-ausstattung}}\label{innovationsruxe4umepathlabelmi-2017selbstbericht0800-ausstattung0000-ausstattung}

\paragraph{Interaktive
Wände:\label{/mi-2017/selbstbericht/0800-ausstattung/0000-ausstattung}}\label{interaktive-wuxe4ndepathlabelmi-2017selbstbericht0800-ausstattung0000-ausstattung}

\begin{itemize}
\tightlist
\item
  2 miteinander verbundene SMART Boards
\item
  1 Microsoft Surface Hub
\item
  Interaktive Doppelprojektion Nureva Span (in Anschaffung)
\end{itemize}

\paragraph{Interaktive Ein- und
Ausgabegeräte:\label{/mi-2017/selbstbericht/0800-ausstattung/0000-ausstattung}}\label{interaktive-ein--und-ausgabegeruxe4tepathlabelmi-2017selbstbericht0800-ausstattung0000-ausstattung}

\begin{itemize}
\tightlist
\item
  HP Sprout
\item
  Interaktiver Tabletop
\item
  Arcade Machine
\item
  Grafiktablets
\end{itemize}

\paragraph{Weiterhin:\label{/mi-2017/selbstbericht/0800-ausstattung/0000-ausstattung}}\label{weiterhinpathlabelmi-2017selbstbericht0800-ausstattung0000-ausstattung}

\begin{itemize}
\tightlist
\item
  Design Thinking Materialien (Boxen, Legosteine, Innovationskarten)
\item
  3D Drucker
\end{itemize}

\subsection{Medieninformatik Pool/
Projektraum\label{/mi-2017/selbstbericht/0800-ausstattung/0000-ausstattung}}\label{medieninformatik-pool-projektraumpathlabelmi-2017selbstbericht0800-ausstattung0000-ausstattung}

Dieser Raum ist mit flexiblen Tischen und Stühlen ausgestattet und wird
sowohl für Seminare und Workshops als auch für Projektarbeit genutzt.
Der Raum kann von allen Medieninformatik-Studierenden außerhalb der
Veranstaltungszeiten genutzt werden. Darüber hinaus stehen den
Studierenden noch weitere, kleinere Projekträume im Medieninformatik
Flur zur Verfügung.

\section{Lehrende in der
Medieninformatik\label{/mi-2017/selbstbericht/0800-ausstattung/0000-ausstattung}}\label{lehrende-in-der-medieninformatikpathlabelmi-2017selbstbericht0800-ausstattung0000-ausstattung}

\subsection{Prof.~Dr.~Birgit
Bertelsmeier\label{/mi-2017/selbstbericht/0800-ausstattung/0000-ausstattung}}\label{prof.dr.birgit-bertelsmeierpathlabelmi-2017selbstbericht0800-ausstattung0000-ausstattung}

\begin{itemize}
\tightlist
\item
  Bachelor: Datenbanken 1
\end{itemize}

\subsection{Prof.~Dr.~Boris
Naujoks\label{/mi-2017/selbstbericht/0800-ausstattung/0000-ausstattung}}\label{prof.dr.boris-naujokspathlabelmi-2017selbstbericht0800-ausstattung0000-ausstattung}

\begin{itemize}
\tightlist
\item
  Master: Spezielle Gebiete der Mathematik
\end{itemize}

\subsection{Prof.~Dr.~Christian
Kohls\label{/mi-2017/selbstbericht/0800-ausstattung/0000-ausstattung}}\label{prof.dr.christian-kohlspathlabelmi-2017selbstbericht0800-ausstattung0000-ausstattung}

\begin{itemize}
\tightlist
\item
  Bachelor: Paradigmen der Programmierung
\item
  Bachelor: Algorithmen und Programmierung 2
\item
  Bachelor: Social Computing
\item
  Master: Computerethik
\item
  Master: Soziotechnische Entwurfsmuster
\end{itemize}

\subsection{Prof.~Christian
Noss\label{/mi-2017/selbstbericht/0800-ausstattung/0000-ausstattung}}\label{prof.christian-nosspathlabelmi-2017selbstbericht0800-ausstattung0000-ausstattung}

\begin{itemize}
\tightlist
\item
  Bachelor: Screendesign
\item
  Bachelor: Web Development
\item
  Bachelor: Praxisprojektseminar
\item
  Master: Web Technologien
\end{itemize}

\subsection{Prof.~Dr.~Frank
Victor\label{/mi-2017/selbstbericht/0800-ausstattung/0000-ausstattung}}\label{prof.dr.frank-victorpathlabelmi-2017selbstbericht0800-ausstattung0000-ausstattung}

\begin{itemize}
\tightlist
\item
  Bachelor: Algorithmen und Programmierung 1
\end{itemize}

\subsection{Prof.~Dr.~Gerhard
Hartmann\label{/mi-2017/selbstbericht/0800-ausstattung/0000-ausstattung}}\label{prof.dr.gerhard-hartmannpathlabelmi-2017selbstbericht0800-ausstattung0000-ausstattung}

\begin{itemize}
\tightlist
\item
  Bachelor: Mensch-Computer Interaktion
\item
  Master: Design Methodologies
\item
  Master: Interaction Design
\item
  Master: Research Methods
\item
  Master: Angewandte Statistik für die Mensch-Computer Interaktion
\end{itemize}

\subsection{Prof.~Dr.~Heide
Faeskorn-Woyke\label{/mi-2017/selbstbericht/0800-ausstattung/0000-ausstattung}}\label{prof.dr.heide-faeskorn-woykepathlabelmi-2017selbstbericht0800-ausstattung0000-ausstattung}

\begin{itemize}
\tightlist
\item
  Bachelor: Datenbanken 1
\end{itemize}

\subsection{Prof.~Dr.~Holger
Günther\label{/mi-2017/selbstbericht/0800-ausstattung/0000-ausstattung}}\label{prof.dr.holger-guxfcntherpathlabelmi-2017selbstbericht0800-ausstattung0000-ausstattung}

\begin{itemize}
\tightlist
\item
  Bachelor: Projektmanagement
\end{itemize}

\subsection{Prof.~Hans
Kornacher\label{/mi-2017/selbstbericht/0800-ausstattung/0000-ausstattung}}\label{prof.hans-kornacherpathlabelmi-2017selbstbericht0800-ausstattung0000-ausstattung}

\begin{itemize}
\tightlist
\item
  Bachelor: Audiovisuelles Medienprojekt
\item
  Bachelor: Visual Computing
\item
  Master: Storytelling und Narrative Strukturen
\end{itemize}

\subsection{Prof.~Dr.~Hans L.
Stahl\label{/mi-2017/selbstbericht/0800-ausstattung/0000-ausstattung}}\label{prof.dr.hans-l.-stahlpathlabelmi-2017selbstbericht0800-ausstattung0000-ausstattung}

\begin{itemize}
\tightlist
\item
  Bachelor: Kommunikationstechnik und Netze
\end{itemize}

\subsection{Prof.~Dr.~Kristian
Fischer\label{/mi-2017/selbstbericht/0800-ausstattung/0000-ausstattung}}\label{prof.dr.kristian-fischerpathlabelmi-2017selbstbericht0800-ausstattung0000-ausstattung}

\begin{itemize}
\tightlist
\item
  Bachelor: Grundlagen des Web
\item
  Bachelor: Web Development
\item
  Master: Web Architekturen
\item
  Master: Netzwerk-und Graphentheorie
\end{itemize}

\subsection{Prof.~Dr.~Lutz
Köhler\label{/mi-2017/selbstbericht/0800-ausstattung/0000-ausstattung}}\label{prof.dr.lutz-kuxf6hlerpathlabelmi-2017selbstbericht0800-ausstattung0000-ausstattung}

\begin{itemize}
\tightlist
\item
  Bachelor: Betriebssysteme und verteilte Systeme
\end{itemize}

\subsection{Prof.~Dr.~Matthias
Böhmer\label{/mi-2017/selbstbericht/0800-ausstattung/0000-ausstattung}}\label{prof.dr.matthias-buxf6hmerpathlabelmi-2017selbstbericht0800-ausstattung0000-ausstattung}

\begin{itemize}
\tightlist
\item
  Bachelor: Betriebssysteme und verteilte Systeme
\end{itemize}

\subsection{Prof.~Dr.~Martin
Eisemann\label{/mi-2017/selbstbericht/0800-ausstattung/0000-ausstattung}}\label{prof.dr.martin-eisemannpathlabelmi-2017selbstbericht0800-ausstattung0000-ausstattung}

\begin{itemize}
\tightlist
\item
  Bachelor: Theoretische Informatik 1
\item
  Bachelor: Theoretische Informatik 2
\item
  Bachelor: Visual Computing
\item
  Master: Bildbasierte Computergrafik
\item
  Master: Visualisierung
\end{itemize}

\subsection{Prof.~Dr.~Monika
Engelen\label{/mi-2017/selbstbericht/0800-ausstattung/0000-ausstattung}}\label{prof.dr.monika-engelenpathlabelmi-2017selbstbericht0800-ausstattung0000-ausstattung}

\begin{itemize}
\tightlist
\item
  Bachelor: BWL I - Grundlagen
\end{itemize}

\subsection{Prof.~Dr.~Mario
Winter\label{/mi-2017/selbstbericht/0800-ausstattung/0000-ausstattung}}\label{prof.dr.mario-winterpathlabelmi-2017selbstbericht0800-ausstattung0000-ausstattung}

\begin{itemize}
\tightlist
\item
  Bachelor: Informatik, Recht und Gesellschaft
\item
  Bachelor: Projektmanagement
\item
  Bachelor: Softwaretechnik
\item
  Master: Qualitätssicherung und Qualitätsmanagement
\end{itemize}

\subsection{Prof.~Dr.~Stefan
Karsch\label{/mi-2017/selbstbericht/0800-ausstattung/0000-ausstattung}}\label{prof.dr.stefan-karschpathlabelmi-2017selbstbericht0800-ausstattung0000-ausstattung}

\begin{itemize}
\tightlist
\item
  Bachelor: Einführung in Betriebssysteme und Rechnerarchitektur
\item
  Master: Sicherheit, Privatsphäre und Vertrauen im Netz
\end{itemize}

\subsection{Prof.~Dr.~Wolfgang
Konen\label{/mi-2017/selbstbericht/0800-ausstattung/0000-ausstattung}}\label{prof.dr.wolfgang-konenpathlabelmi-2017selbstbericht0800-ausstattung0000-ausstattung}

\begin{itemize}
\tightlist
\item
  Bachelor: Mathematik 1
\item
  Bachelor: Mathematik 2
\item
  Master: Spezielle Gebiete der Mathematik
\end{itemize}

Alle Lehrenden des Instituts für Informatik finden Sie auf der Website
der TH Köln unter \href{http://bit.ly/2mbFEYc}{http://bit.ly/2mpcbWN}

\chapter{Transparenz, Dokumentation, Qualitätssicherung und
Weiterentwicklung\label{/mi-2017/selbstbericht/0900-transparenz-und-dokumentation/0000-transparenz-und-dokumentation}}\label{transparenz-dokumentation-qualituxe4tssicherung-und-weiterentwicklungpathlabelmi-2017selbstbericht0900-transparenz-und-dokumentation0000-transparenz-und-dokumentation}

\section{Beschreibung des Qualitätssicherungssystems der
Studiengänge\label{/mi-2017/selbstbericht/0900-transparenz-und-dokumentation/0000-transparenz-und-dokumentation}}\label{beschreibung-des-qualituxe4tssicherungssystems-der-studienguxe4ngepathlabelmi-2017selbstbericht0900-transparenz-und-dokumentation0000-transparenz-und-dokumentation}

Der Senat der TH Köln hat am 12. Dezember 2013 die Ordnung für die
Evaluation von Studium und Lehre in ihrer dritten Fassung verabschiedet,
die den Verfahrensablauf und die Verfahrensschritte von
Evaluationsverfahren an der TH Köln regelt. Die TH Köln verfügt mit dem
Hochschulreferat 4 Qualitätsmanagement über eine zentrale
Organisationseinheit für die Entwicklung und Durchführung von
Evaluationsverfahren. Zudem hat der Fachausschuss des Studiengangs einen
Qualitätsbeauftragten benannt, der als interner Ansprechpartner für die
zentral organisierten Evaluationen und Studierendenbefragungen fungiert
und neben dem Vorsitzenden des Fachausschusses direkter Ansprechpartner
für die Umsetzung von Qualitätsverbesserungsmaßnahmen ist.

Die Evaluationsordnung beinhaltet auch Befragungen zur Qualität des
Studiums, wie sie nach § 7 HZG in Nordrhein-Westfalen vorgeschrieben
sind.

Das Qualitätsmanagement der Programme beinhaltet eine semesterweise
Evaluation der Module sowie deren Unterbestandteile, Projekte und Kurse.
Die Lehrveranstaltungsbewertungen des Studiengangs werden kontinuierlich
evaluiert. Die Evaluationsergebnisse werden zentral durch das
Hochschulreferat 4 ausgewertet und aufbereitet. Die Ergebnisse von
Lehrveranstaltungsbewertungen werden den Studierenden zurück gemeldet,
um einen Dialog über die Qualität der Lehre zu initiieren und im Rahmen
der quantitativen Ergebnisanalyse sichtbar gewordene Kritikpunkte
inhaltlich-qualitativ weiter aufarbeiten zu können.

Die individuellen Ergebnisse der Lehrveranstaltungsbewertungen werden
darüber hinaus auch vom Qualitätsbeauftragten gesichtet, um bei
signifikanten Qualitätsproblemen einzelner Lehrender oder in einzelnen
Lehrbereichen über Gespräche mit den betroffenen Kolleginnen und
Kollegen Lösungsmöglichkeiten für die sichtbar gewordenen Probleme zu
entwickeln.

\section{Organisatorische
Prozesse\label{/mi-2017/selbstbericht/0900-transparenz-und-dokumentation/0000-transparenz-und-dokumentation}}\label{organisatorische-prozessepathlabelmi-2017selbstbericht0900-transparenz-und-dokumentation0000-transparenz-und-dokumentation}

Der Campus Gummersbach der TH Köln ist der erste Campus einer
öffentlichen Hochschule in Nordrhein-Westfalen und einer von ganz
wenigen in Deutschland, der ein ganzheitliches
Qualitäts-Management-System nach ISO 9001:2008 umgesetzt hat.

Auf Basis des Fakultätsentwicklungsplans 2010-2015 wurde ein
Qualitäts-Management-System mit fünf wesentlichen Handlungsfeldern
definiert: - Qualität der Lehre - Strategische Studienprogramme -
Internationalisierung - Forschung und Wissenstransfer -
Standortentwicklung und Infrastruktur

Die vereinbarten Qualitätsziele wurden an quantifizierbaren Kenngrößen
oder beschlossenen Maßnahmen orientiert. Die Verbesserungsmaßnahmen und
-programme werden jährlich in einer Management-Review evaluiert. Als
zentrales Dokument des Qualitäts-Management-Systems wurde das
QM-Handbuch für unsere Mitarbeiter und Mitarbeiterinnen, Studierenden
und Forschungspartner angefertigt. Dafür verlieh der TÜV Rheinland nach
umfangreicher Prüfung (Audit) das Zertifikat mit der Klassifizierung
``Premium'' für ein besonders hochwertiges System.

\section{Beispielhafte
Prozessdokumentionen\label{/mi-2017/selbstbericht/0900-transparenz-und-dokumentation/0000-transparenz-und-dokumentation}}\label{beispielhafte-prozessdokumentionenpathlabelmi-2017selbstbericht0900-transparenz-und-dokumentation0000-transparenz-und-dokumentation}

\begin{itemize}
\tightlist
\item
  \href{https://th-koeln.github.io/mi-2017/anhaenge/Prozessbeschreibung_PruefungsprozessUeberblick.pdf}{Überblick
  über den Prüfungsprozess}
\item
  \href{https://th-koeln.github.io/mi-2017/anhaenge/Prozessbeschreibung_Pruefungsordnungsprozess.pdf}{Prüfungsordnungsprozess}
\item
  \href{https://th-koeln.github.io/mi-2017/anhaenge/Prozessbeschreibung_Pruefungplanung.pdf}{Prozess zur
  Prüfungsplanung}
\item
  \href{https://th-koeln.github.io/mi-2017/anhaenge/Prozessbeschreibung_Pruefungdurchfuehren.pdf}{Prozess
  zur Prüfunungsdurchführung}
\item
  \href{https://th-koeln.github.io/mi-2017/anhaenge/Prozessbeschreibung_vonPruefungzuruecktreten.pdf}{Prozess
  beim Rücktritt von Prüfungen}
\item
  \href{https://th-koeln.github.io/mi-2017/anhaenge/Prozessbeschreibung_EinspruchPruefungsergebnisse.pdf}{Prozess
  beim Einspruch gegen Prüfungsergebnisse}
\item
  \href{https://th-koeln.github.io/mi-2017/anhaenge/Prozessbeschreibung-Absolventenzufriedenheit.pdf}{Prozess
  Absolventenzufriedenheit}
\item
  \href{https://th-koeln.github.io/mi-2017/anhaenge/Prozessbeschreibung-Evaluationsprozesse.pdf}{Prozess
  Evaluationsprozesse}
\item
  \href{https://th-koeln.github.io/mi-2017/anhaenge/Prozessbeschreibung-Partnerzufriedenheit.pdf}{Prozess
  Partnerzufriedenheit}
\item
  \href{https://th-koeln.github.io/mi-2017/anhaenge/Prozessbeschreibung-Rankings.pdf}{Prozess Rankings}
\item
  \href{https://th-koeln.github.io/mi-2017/anhaenge/Prozessbeschreibung-Studierendenzufriedenheit.pdf}{Prozess
  Studierendenzufriedenheit}
\end{itemize}

\chapter{Geschlechtergerechtigkeit, Chancengleichheit und Studieren mit
Beeinträchtigungen\label{/mi-2017/selbstbericht/1100-geschlechtergerechtigkeit-und-chancengleichheit/0000-geschlechtergerechtigkeit-und-chancengleichheit}}\label{geschlechtergerechtigkeit-chancengleichheit-und-studieren-mit-beeintruxe4chtigungenpathlabelmi-2017selbstbericht1100-geschlechtergerechtigkeit-und-chancengleichheit0000-geschlechtergerechtigkeit-und-chancengleichheit}

\section{Gleichstellung und Chancengleichheit der
Geschlechter\label{/mi-2017/selbstbericht/1100-geschlechtergerechtigkeit-und-chancengleichheit/0000-geschlechtergerechtigkeit-und-chancengleichheit}}\label{gleichstellung-und-chancengleichheit-der-geschlechterpathlabelmi-2017selbstbericht1100-geschlechtergerechtigkeit-und-chancengleichheit0000-geschlechtergerechtigkeit-und-chancengleichheit}

Die TH Köln betrachtet Gleichstellung und Chancengleichheit der
Geschlechter als auch die Integration von Studierenden mit
Beeinträchtigungen als Querschnittsaufgaben. Dabei wird Gleichstellung
als integrierter Bestandteil von Lehre und Forschung verstanden, auf die
Vereinbarkeit von Studium und Familie beziehungsweise Beruf und Familie
geachtet sowie für eine ausgewogene Beteiligung von Männern und Frauen
an den Entscheidungsstrukturen in Lehre, Forschung und Verwaltung
gesorgt. Darüber hinaus wird der Anteil der Frauen bei den Professuren,
Mitarbeiterstellen und den Studierenden in denjenigen Fächern, in denen
sie unterrepräsentiert sind, kontinuierlich erhöht.

Es wird die Aufstellung und Einhaltung der Frauenförderpläne
kontrolliert. Des Weiteren werden bei einem „Girl's Day`` spezielle
Veranstaltungen für interessierte Frauen, bezüglich der
Informatikstudiengänge, angeboten. Alle Konzepte und Maßnahmen für
Geschlechtergerechtigkeit und Chancengleichheit finden auf die zu
akkreditierenden Studiengänge Anwendung.

Fernerhin hat die TH Köln das Audit familiengerechte
Hochschule\footnote{\href{https://www.th-koeln.de/hochschule/familienfreundlichkeit_3759.php}{Profilseite
  zu Family Matters auf der Website der TH Köln}} der „berufundfamilie
gemeinnützigen GmbH'' erfolgreich durchgeführt. Im Rahmen der
Auditierung wurden der Bestand familienorientierter Maßnahmen
begutachtet und weiterführende Zielvorgaben zur Verwirklichung
familiengerechter Studienbedingungen sowie einer familienbewussten
Personalpolitik definiert. Die Hochschule ist in 2015 erfolgreich
re-auditiert worden.

\section{Förderung der
Chancengleichheit\label{/mi-2017/selbstbericht/1100-geschlechtergerechtigkeit-und-chancengleichheit/0000-geschlechtergerechtigkeit-und-chancengleichheit}}\label{fuxf6rderung-der-chancengleichheitpathlabelmi-2017selbstbericht1100-geschlechtergerechtigkeit-und-chancengleichheit0000-geschlechtergerechtigkeit-und-chancengleichheit}

Die Konzepte zur Förderung der Chancengleichheit gelten insbesondere für
Studierende in besonderen Lebenslagen (z.B. Studierende mit Kind), für
Studierende mit Beeinträchtigung oder für Studierende mit spezifischem
sozialem Hintergrund.

Die TH Köln versteht sich als familiengerechte Hochschule und bietet
verschiedene Beratungsangebote und Serviceleistungen für studierende
Eltern an, um die Vereinbarkeit von Studium/Beruf und Familie besser zu
ermöglichen. Im Herbst 2009 wurde das Programm „Educational Diversity''
der TH Köln aufgesetzt. Die Grundidee von Educational Diversity ist die
Umsetzung einer gelebten, die Unterschiedlichkeit der Studierenden als
kreatives Potenzial begreifenden, Lehr- und Lerncommunity. Alle Akteure
stehen im direkten Kontakt miteinander und werden durch eine webbasierte
Lehr- und Lerncommunity unterstützt.

Das Programm „Educational Diversity``\footnote{\href{https://www.th-koeln.de/hochschule/educational-diversity_5710.php}{Programm
  Educational Diversity der Technischen Hochschule Köln}} der TH Köln
hat zum Ziel, die Verschiedenartigkeit der Studierenden zu erkennen und
durch hochschuldidaktische Differenzierung das Potenzial jedes/jeder
einzelnen Studierenden optimal zu fördern. Auch die Dozent und
Dozentinnen der Informatikstudiengänge beteiligen sich an diesen
Programmen.

Für die Umsetzung der Chancengleichheit von Männern und Frauen hat die
Hochschule in ihrem Entwicklungsplan vier Ziele benannt:

\begin{enumerate}
\def\labelenumi{\arabic{enumi}.}
\tightlist
\item
  Die Ermöglichung einer geschlechtsunabhängigen Studienfachwahl für
  Schülerinnen und Schüler.
\item
  Die Erhöhung des Frauenanteils bei den wissenschaftlichen
  Beschäftigten der TH Köln, insbesondere bei den Professorinnen,
  wissenschaftlichen Mitarbeiterinnen und Lehrbeauftragten.
\item
  Die Verbesserung der Vereinbarkeit von Studium bzw. Beruf und Familie.
\item
  Die Umsetzung bzw. Unterstützung genderbezogener Projekte in Lehre und
  Forschung.
\end{enumerate}

Die Umsetzung dieser Ziele und die Einbettung in die bestehenden
Handlungsfelder der Hochschule werden in der amtlichen
Mitteilung\footnote{\href{http://www.fh-koeln.de/mam/downloads/deutsch/hochschule/profil/gleichstellung/gleichstellungskonzept.pdf}{Gleichstellungskonzept
  der Technischen Hochschule Köln}} näher erläutert. Der TH Köln ist es
ein besonderes Anliegen, mit den umgesetzten Maßnahmen die Vereinbarkeit
von Beruf und Familie bzw. Studium und Familie zu unterstreichen und
damit eine Kulturveränderung innerhalb der Hochschule zu bewirken, denn
damit werden indirekt Karrierehemmnisse von Frauen abgebaut.

\section{Studieren mit
Beeinträchtigung\label{/mi-2017/selbstbericht/1100-geschlechtergerechtigkeit-und-chancengleichheit/0000-geschlechtergerechtigkeit-und-chancengleichheit}}\label{studieren-mit-beeintruxe4chtigungpathlabelmi-2017selbstbericht1100-geschlechtergerechtigkeit-und-chancengleichheit0000-geschlechtergerechtigkeit-und-chancengleichheit}

Statistisch gesehen sind rund 1.800 Studierende an der TH Köln
beeinträchtigt, d.h. behindert oder chronisch erkrankt. Auch wenn die
Beeinträchtigung im täglichen Umgang bei ca. zwei Dritteln dieser
Studierenden nicht direkt wahrgenommen wird, setzt sich die TH Köln mit
dem Thema „Behinderung und Studium`` auseinander. Ein entsprechendes
Beratungsangebot für Studierende und Lehrende wird dafür bereit gestellt
und durch die Beauftragte für Studierende mit Beeinträchtigung\footnote{\href{https://www.th-koeln.de/studium/studieren-mit-beeintraechtigung_169.php}{Website:
  Studieren mit Beeinträchtigungen}} koordiniert und realisiert. Das
Beratungsspektrum ist vielfältig. Viele Studierende haben Sorgen und
fühlen sich unsicher. Auch manche Mitarbeiterinnen und Mitarbeiter haben
Informationsbedarf.

Um „Berührungsängste`` abzubauen und Handlungssicherheit zu geben,
werden diverse Handreichungen bereit gestellt. Hierin werden, auf Basis
von Erhebungen des Deutschen Studentenwerkes, Informationen über die
Situation von Studierenden mit Beeinträchtigung vorgehalten. Weiterhin
werden grundlegende Kenntnisse über die am häufigsten vorkommenden
Beeinträchtigungen vermittelt und Hinweise gegeben, wie sich gegenüber
beeinträchtigten Studierenden verhalten werden kann. Auch das Thema
Nachteilsausgleich wird hier adressiert. Für Lehrende sei beispielhaft
die Broschüre \emph{Studieren mit Beeinträchtigung - Handreichung für
Lehrende und Beschäftigte der TH Köln}\footnote{\href{https://www.th-koeln.de/mam/downloads/deutsch/studium/beratung/beeintraechtigung/a5brosch__re_beeintr__chtigung_2016.pdf}{PDF:
  Studieren mit Beeinträchtigung - Handreichung für Lehrende und
  Beschäftigte der TH Köln. Zentrale Studienberatung, TH Köln, 2016}}
angeführt.

\chapter{Auflagen zur
Akkreditierung\label{/mi-2017/selbstbericht/auflagen/0000-auflagen}}\label{auflagen-zur-akkreditierungpathlabelmi-2017selbstberichtauflagen0000-auflagen}

Die Akkreditierungskommission der ASIIN hat die Studiengänge
Medieninformatik Bachelor und Medieninformatik Master der TH Köln am 30.
Juni 2017 akkreditiert. Hierbei wurden einige Empfehlungen und Auflagen
formuliert. In diesem Dokument finden sich die Erläuterungen zur
fristgerechten Erfüllung der ausgesprochenen Auflagen.

\section{Auflagen für alle
Studiengänge\label{/mi-2017/selbstbericht/auflagen/0000-auflagen}}\label{auflagen-fuxfcr-alle-studienguxe4ngepathlabelmi-2017selbstberichtauflagen0000-auflagen}

\subsection{A1. (AR
2.3)\label{/mi-2017/selbstbericht/auflagen/0000-auflagen}}\label{a1.-ar-2.3pathlabelmi-2017selbstberichtauflagen0000-auflagen}

\begin{siderules}
Die Modulbeschreibungen müssen angemessen über die Voraussetzungen für
die Teilnahme, die Voraussetzungen für die Vergabe von Kreditpunkten und
Notenbildung sowie den Arbeitsaufwand informieren.
\end{siderules}

In den Modulbeschreibungen wurden die entsprechenden Angaben überprüft
und, soweit erforderlich, angepasst. Darüber hinaus wurde innerhalb der
Modulhandbücher jeweils eine kurze Einleitung an den Anfang gestellt.
Diese enthält einen graphischen Studienverlaufsplan, um insgesamt eine
bessere Verständlichkeit und Übersichtlichkeit herzustellen. Darüber
hinaus wurden Hyperlinks innerhalb der Handbücher farblich
hervorgehoben, um eine besser Handhabbarkeit und schnellere Navigation
im jeweiligen Handbuch zu ermöglichen.

Die aktuellen Modulhandbücher sind über folgende URL erreichbar:

\begin{itemize}
\tightlist
\item
  \href{http://www.medieninformatik.th-koeln.de/download/modulbeschreibungen-bachelor-bpo4.pdf}{Modulhandbuch
  Medieninformatik Bachelor}
\item
  \href{http://www.medieninformatik.th-koeln.de/download/modulbeschreibungen-master-mpo4.pdf}{Modulhandbuch
  Medieninformatik Master}
\end{itemize}

\subsection{A2. (AR 2.8)
\label{/mi-2017/selbstbericht/auflagen/0000-auflagen}}\label{a2.-ar-2.8-pathlabelmi-2017selbstberichtauflagen0000-auflagen}

\begin{siderules}
Die in Kraft gesetzten und veröffentlichten Ordnungen inklusive der
angepassten Diploma Supplements für beide Studiengänge sind vorzulegen.
\end{siderules}

Die Prüfungsordnung und der zugehörige Studienverlaufsplan für den
Medieninformatik Bachelor wurden mit der \textbf{Änderungssatzung vom
24.11.2017} in Kraft gesetzt und auf der Website der TH Köln
veröffentlicht:
\href{https://www.th-koeln.de/studium/medieninformatik-bachelor--ordnungen-und-formulare_3963.php}{Medieninformatik
(Bachelor) -- Ordnungen und Formulare}.

Hier findet sich ein Muster für das zugehörige
\href{https://th-koeln.github.io/mi-2017/download/auflagen/THK-DS-MIF-MIB-PO4.pdf}{Diploma
Supplement für den Medieninformatik Bachelor}

Die Prüfungsordnung und der zugehörige Studienverlaufsplan für den
Medieninformatik Master wurden mit der \textbf{Prüfungsordnung vom
13.07.2017} in Kraft gesetzt und auf der Website der TH Köln
veröffentlicht:
\href{https://www.th-koeln.de/studium/medieninformatik-master--ordnungen-und-formulare_3724.php}{Medieninformatik
(Master) -- Ordnungen und Formulare}.

Hier findet sich ein Muster für das zugehörige
\href{https://th-koeln.github.io/mi-2017/download/auflagen/THK-DS-MIF-MIM-PO4.pdf}{Diploma
Supplement für den Medieninformatik Master}.

Bitte beachten Sie, dass das Diploma Supplement auf speziellem Papier
gedruckt wird, auf dem ein Farbkeil (links) vorgedruckt ist, daher ist
dieser in der PDF-Datei nicht enthalten. Die mit ``«\ldots{}»''
gekennzeichneten Felder, werden bei Ausgabe mit den personalisierten
Daten der jeweiligen Absolvent*innen befüllt.

\section{Auflagen für den
Masterstudiengang\label{/mi-2017/selbstbericht/auflagen/0000-auflagen}}\label{auflagen-fuxfcr-den-masterstudiengangpathlabelmi-2017selbstberichtauflagen0000-auflagen}

\subsection{A3. (AR 2.1)
\label{/mi-2017/selbstbericht/auflagen/0000-auflagen}}\label{a3.-ar-2.1-pathlabelmi-2017selbstberichtauflagen0000-auflagen}

\begin{siderules}
Für die fünf Spezialisierungen des Masterstudiengangs ist eine
gleichmäßige detaillierte Beschreibung der Berufsfelder in den
Qualifikationszielen vorzunehmen. Die Module sind als eigenständige
Lehr-/Lerneinheiten darzustellen (unabhängig von der Darstellung der
Schwerpunkte, zu denen sie gehören).
\end{siderules}

Die Beschreibung der Spezialisierungen des Masterstudiengangs wurden um
den Punkt «Berufsbilder» ergänzt. Aufgrund der fachbedingten
Unterschiede der Spezialisierungen, sind diese Erläuterungen teilweise
unterschiedlich aufgebaut, verfolgen aber alle das Ziel, den
Studierenden und Studieninteressierten klar zu machen, welche
beruflichen Perspektiven sich mit der jeweiligen Spezialisierung
eröffnen.

Die Schwerpunktspezifische Pflichtmodule der jeweiligen Spezialisierung
sind jetzt als farblich gekennzeichnete Hyperlinks im Handbuch
hinterlegt und führen direkt zur entsprechenden Modulbeschreibung des
jeweiligen Moduls. Die Modulbeschreibungen selbst, enthalten im
Factsheet zum Modul jetzt den Punkt «Pflichtmodul(e) im Schwerpunkt», um
die entsprechenden Zugehörigkeiten transparent zu machen.

\begin{figure}[htbp]
\centering
\includegraphics[width=\columnwidth]{../anhaenge/bilder/hyperlinks.png}
\caption{Hyperlinks zu den Modulbeschreibungen}
\end{figure}

\begin{figure}[htbp]
\centering
\includegraphics[width=\columnwidth]{../anhaenge/bilder/pm-im-schwerpunkt.png}
\caption{Schwerpunktzugehörigkeit des jeweiligen Moduls}
\end{figure}

\chapter{Kommentare zu dargelegten Defiziten oder
Unklarheiten\label{/mi-2017/selbstbericht/stellungnahme/0000-stellungnahme}}\label{kommentare-zu-dargelegten-defiziten-oder-unklarheitenpathlabelmi-2017selbstberichtstellungnahme0000-stellungnahme}

\section{zu Kriterium 2.1: prägnantere Darstellung der adressierten
Berufsfelder der Studienschwerpunkte im
Master\label{/mi-2017/selbstbericht/stellungnahme/0000-stellungnahme}}\label{zu-kriterium-2.1-pruxe4gnantere-darstellung-der-adressierten-berufsfelder-der-studienschwerpunkte-im-masterpathlabelmi-2017selbstberichtstellungnahme0000-stellungnahme}

\subsection{Auszug aus dem Bericht der
Gutachter\label{/mi-2017/selbstbericht/stellungnahme/0000-stellungnahme}}\label{auszug-aus-dem-bericht-der-gutachterpathlabelmi-2017selbstberichtstellungnahme0000-stellungnahme}

\begin{siderules}
In den genannten Qualifikationszielen sehen die Gutachter weitestgehend
eine Qualifikation zur Aufnahme einer angemessenen Berufstätigkeit,
merken jedoch an, dass die Beschreibungen der beruflichen Ausrichtung
bei den Studienrichtungen im Masterstudiengang mitunter detaillierter
ausfallen könnten. Sie betonen, dass gerade bei einer derartigen
Aufspaltung in fünf Richtungen für Studieninteressierte deutlich werden
muss, welche Berufsperspektiven mit welcher Studienrichtung verknüpft
werden.
\end{siderules}

Zum besseren Verständnis der Studienschwerpunkte wird die Grundidee der
Schwerpunkte, als auch die damit verbundenen Kompetenzen und
Berufsperspektiven in den einschlägigen Dokumenten (Homepage, Broschüre,
etc.) in nächster Zeit dokumentiert und veröffentlicht. Anbei zwei
exemplarische Berufsbildbeschreibungen für den Schwerpunkt
Human-Computer Interaction:

\textbf{Usability Engineers} arbeiten entweder direkt im Unternehmen
oder in der Beratung von Unternehmen. Ihre maßgebliche Aufgabe ist es,
über den gesamten Lebenszyklus für eine hohe Gebrauchstauglichkeit
interaktiver sozio-technischer Systeme zu sorgen. Dazu wenden sie
Prinzipien, Vorgehensweisen, Methoden und Arbeitstechniken der Disziplin
„Mensch-Computer-Interaktion`` an. Sie planen Entwicklungsprozesse,
analysieren Lebens- und Nutzungskontexte von Nutzergruppen, analysieren
und spezifizieren Nutzungsanforderungen, entwerfen Gestaltungslösungen
und analysieren/evaluieren diese. Darüber hinaus kommunizieren sie mit
allen Berufsgruppen, die bei der Konzeption, Gestaltung, Entwicklung,
Evaluation und dem Betrieb dieser interaktiven Systeme beteiligt sind
und übernehmen damit quasi die Rolle eines Anwalts der Benutzer.

\textbf{Interaction Designer} konzipieren und gestalten die vielfältigen
Beziehungen zwischen Menschen und Technologien. Diese Beziehungen sind
unter anderem ökonomischer, sozialer, ökologischer, kulturell/ethischer
aber auch ästhetischer Art. Anders als bei der eher
ingenieurwissenschaftlichen Herangehensweise der Usability Engineers
denken und handeln Interaction Designer vornehmlich aus der
Designperspektive. Dies bedeutet, dass Interaction Designer in ähnlichen
Projekten tätig sind, aber mit einer ausgeprägten kreativen
Problemlösungskompetenz auf methodischer Ebene sowie einer reflektierten
und eigenverantwortlichen Entscheidungskompetenz ausgestattet sind. Sie
können sicherstellen, dass sich Technologie nach gewünschten
Wertmaßstäben nahtlos und positiv in den Lebensalltag von Menschen
eingliedert. Damit geht Interaction Design weit über die reine
Konzeption und Gestaltung von Eingaben und Ausgabe an der
Benutzungsschnittstelle (User Interface Design) hinaus.

\section{zu Kriterium 2.3: Umfang des Moduls Theoretische
Informatik\label{/mi-2017/selbstbericht/stellungnahme/0000-stellungnahme}}\label{zu-kriterium-2.3-umfang-des-moduls-theoretische-informatikpathlabelmi-2017selbstberichtstellungnahme0000-stellungnahme}

\subsection{Auszug aus dem Bericht der
Gutachter\label{/mi-2017/selbstbericht/stellungnahme/0000-stellungnahme}}\label{auszug-aus-dem-bericht-der-gutachterpathlabelmi-2017selbstberichtstellungnahme0000-stellungnahme-1}

\begin{siderules}
Grundsätzlich kommt man darin überein, dass die Theoretische Informatik
gewinnbringend für Studierende sein kann, die Gutachter geben aber zu
bedenken, dass der Umfang von zwei Modulen die Entfaltungsmöglichkeiten
in anderen, der Medieninformatik näheren Themenbereichen, einschränken
kann.
\end{siderules}

\subsection{Stellungnahme der
Hochschule\label{/mi-2017/selbstbericht/stellungnahme/0000-stellungnahme}}\label{stellungnahme-der-hochschulepathlabelmi-2017selbstberichtstellungnahme0000-stellungnahme}

Die Theoretische Informatik (TI) wird im Medieninformatik Bachelor von
den Programmverantwortlichen als essentiell mit dem jetzigen Umfang von
10 CP angesehen. Ein Ziel bei der Überarbeitung des Studiengangs war es,
formale, algorithmische, mathematische und Realisierungskompetenzen
systematischer und nachhaltiger aufzubauen und im Vergleich zum Status
Quo zu verbessern. TI bereitet dabei ein solides Fundament, welches vor
allem das algorithmische Denken und Abstraktionsvermögen stärkt und
damit die Grundlagen zur Softwaremodellierung legt. An der TH Köln wird
die Theorie in der Theoretischen Informatik auch immer mit konkretem
Praxisbezug vermittelt, damit die Studierenden die Konzepte in späteren
Veranstaltungen wiedererkennen und anwenden können.

\subsection{Beispiele:\label{/mi-2017/selbstbericht/stellungnahme/0000-stellungnahme}}\label{beispielepathlabelmi-2017selbstberichtstellungnahme0000-stellungnahme}

\begin{itemize}
\tightlist
\item
  Mengen und Relationen werden durch Bezüge zu Constructive Solid
  Geometry aus der Computergrafik und 3D Druck oder Beziehungen in
  Sozialen Netzwerken dargestellt
\item
  Boolesche Algebra mit Bezug auf Entwurf von Schaltelementen und
  Künstliche Intelligenz zur Lösung logischer Probleme
\item
  Sprachen und Grammatiken, sowie Endliche Automaten, Kellerautomaten,
  Petri-Netze mit konkretem Bezug auf Syntax-Checker,
  Softwaremodellierung und Aufbau von Abstraktionsvermögen durch
  Abbildung alltäglicher Probleme auf eben genannte Darstellungsformen
\end{itemize}

Durch die Turing-Maschinen werden zudem wichtige Informatikkonzepte, wie
das Zerteilen großer Probleme in lösbare Teilprobleme, eine der
wichtigsten Kompetenzen für Informatiker, geübt und auch die Praktische
Umsetzung trainiert, ohne dass man größere Programmiererfahrung
benötigen würde.

Um diesen Praxisbezug in geeigneter Form zu vermitteln ist der
verhältnismäßig hohe Umfang von 10 CP für die TI aus Sicht der
Programmverantwortlichen vollkommen gerechtfertigt. Der Umfang wurde
seitens der Programmverantwortlichen in der Vorbereitung zur
Reakkreditierung jedoch ebenfalls hinlänglich diskutiert. Eine mögliche
Option wäre gewesen, die TI auf 5 CP zu reduzieren. Dann hätten aber
viele Inhalte in anwendungsnähere Module (wie Computergrafik und
Animation, Paradigmen der Programmierung, etc.) verschoben werden
müssen, was den inhaltlichen Umfang der anwendungsnäheren Module
vergrößert hätte. Somit wurde sich darauf verständigt das Modul bei
einem Umfang von 10 CP zu belassen.

\section{\texorpdfstring{zu Kriterium 2.3: Inhaltliche Ausrichtung
des Moduls ``Medienrecht, Medien und
Gesellschaft''\label{/mi-2017/selbstbericht/stellungnahme/0000-stellungnahme}}{zu Kriterium 2.3: Inhaltliche Ausrichtung des Moduls Medienrecht, Medien und Gesellschaft\label{/mi-2017/selbstbericht/stellungnahme/0000-stellungnahme}}}\label{zu-kriterium-2.3-inhaltliche-ausrichtung-des-moduls-medienrecht-medien-und-gesellschaftpathlabelmi-2017selbstberichtstellungnahme0000-stellungnahme}

\subsection{Auszug aus dem Bericht der
Gutachter\label{/mi-2017/selbstbericht/stellungnahme/0000-stellungnahme}}\label{auszug-aus-dem-bericht-der-gutachterpathlabelmi-2017selbstberichtstellungnahme0000-stellungnahme-2}

\begin{siderules}
Die Gutachter loben die Präsenz dieser beiden Themenbereiche, die in der
Medieninformatik eine immer größere Rolle einnehmen, betonen aber, dass
gerade im Bereich Recht die spezifische inhaltliche Ausrichtung auf
Medienrecht, Internetrecht und Urheberrecht noch stärker betont werden
könnte.
\end{siderules}

\subsection{Stellungnahme der
Hochschule\label{/mi-2017/selbstbericht/stellungnahme/0000-stellungnahme}}\label{stellungnahme-der-hochschulepathlabelmi-2017selbstberichtstellungnahme0000-stellungnahme-1}

Hier folgen die Programmverantwortlichen der Argumentation der
Gutachter. Bislang handelt es sich bei diesem Modul um ein Modul, dass
für alle Informatik Bachelor Studiengänge am Campus Gummersbach
gemeinsam angeboten wird. Dementsprechend ist die inhaltliche
Ausrichtung der Lehrveranstaltung Medienrecht eher generisch. Die
Programmverantwortlichen streben eine Medieninformatik-spezifische
Lehrveranstaltung an, bei der die oben genannten Themen mehr im Fokus
stehen.

\section{zu Kriterium 2.3: Defizite bei den
Modulbeschreibungen\label{/mi-2017/selbstbericht/stellungnahme/0000-stellungnahme}}\label{zu-kriterium-2.3-defizite-bei-den-modulbeschreibungenpathlabelmi-2017selbstberichtstellungnahme0000-stellungnahme}

\subsection{Auszug aus dem Bericht der
Gutachter\label{/mi-2017/selbstbericht/stellungnahme/0000-stellungnahme}}\label{auszug-aus-dem-bericht-der-gutachterpathlabelmi-2017selbstberichtstellungnahme0000-stellungnahme-3}

\begin{siderules}
In Bezug auf die Modulbeschreibungen stellen die Gutachter noch einige
Defizite fest, die im Gespräch mit den Programmverantwortlichen
eingeräumt werden.
\end{siderules}

\subsection{Stellungnahme der
Hochschule\label{/mi-2017/selbstbericht/stellungnahme/0000-stellungnahme}}\label{stellungnahme-der-hochschulepathlabelmi-2017selbstberichtstellungnahme0000-stellungnahme-2}

Hier folgen die Programmverantwortlichen der Argumentation der
Gutachter. Um in diesem Punkt eine Verbesserung zu erzielen, wird
derzeit ein Leitfaden für die Modulbeschreibungen in der
Medieninformatik entwickelt und in Kürze Anwendung finden.

\section{zu Kriterium 2.3: Verständlichere Darstellung des
Schwerpunktkonzepts im
Master\label{/mi-2017/selbstbericht/stellungnahme/0000-stellungnahme}}\label{zu-kriterium-2.3-verstuxe4ndlichere-darstellung-des-schwerpunktkonzepts-im-masterpathlabelmi-2017selbstberichtstellungnahme0000-stellungnahme}

\subsection{Auszug aus dem Bericht der
Gutachter\label{/mi-2017/selbstbericht/stellungnahme/0000-stellungnahme}}\label{auszug-aus-dem-bericht-der-gutachterpathlabelmi-2017selbstberichtstellungnahme0000-stellungnahme-4}

\begin{siderules}
Die Gutachter sehen es als notwendig an, hier eine verständlichere
Darstellungsform zu wählen, die den Studierenden das Konzept, die
Strukturierung, die Inhalte und die Anforderungen der Schwerpunkte
zugänglich macht.
\end{siderules}

\subsection{Stellungnahme der
Hochschule\label{/mi-2017/selbstbericht/stellungnahme/0000-stellungnahme}}\label{stellungnahme-der-hochschulepathlabelmi-2017selbstberichtstellungnahme0000-stellungnahme-3}

Hier folgen die Programmverantwortlichen der Argumentation der
Gutachter. Wie bereits erwähnt, wird zum besseren Verständnis der
Studienschwerpunkte, die Grundidee der Schwerpunkte, als auch die damit
verbundenen Kompetenzen und Berufsperspektiven in den einschlägigen
Dokumenten (Homepage, Broschüre, etc.) in nächster Zeit dokumentiert und
veröffentlicht.

\section{zu Kriterium 2.7: Verhältnis von Studiengangsplätzen und
Studierenden
\label{/mi-2017/selbstbericht/stellungnahme/0000-stellungnahme}}\label{zu-kriterium-2.7-verhuxe4ltnis-von-studiengangspluxe4tzen-und-studierenden-pathlabelmi-2017selbstberichtstellungnahme0000-stellungnahme}

\subsection{Auszug aus dem Bericht der
Gutachter\label{/mi-2017/selbstbericht/stellungnahme/0000-stellungnahme}}\label{auszug-aus-dem-bericht-der-gutachterpathlabelmi-2017selbstberichtstellungnahme0000-stellungnahme-5}

\begin{siderules}
Die Gutachter sind in Anbetracht des großen Engagements der Lehrenden
zwar davon überzeugt, dass alles getan wird, um der großen
Studierendenzahl gerecht zu werden, verweisen aber darauf, dass
langfristig das Verhältnis von Studienplätzen und aufgenommenen
Studierenden wieder angeglichen werden muss. Dies gilt insbesondere mit
Blick auf die Tatsache, dass die Hochschulpaktmittel im Laufe des
Akkreditierungszeitraums auslaufen werden.
\end{siderules}

\subsection{Stellungnahme der
Hochschule\label{/mi-2017/selbstbericht/stellungnahme/0000-stellungnahme}}\label{stellungnahme-der-hochschulepathlabelmi-2017selbstberichtstellungnahme0000-stellungnahme-4}

Um trotz der aktuellen Überlast angemessene Lehr-Lern Arrangements zu
realisieren, wurden bislang konkrete Maßnahmen ergriffen: - für die
Unterstützung bei Lehrveranstaltungen und Projekten sind eine Reihe von
wissenschaftlichen Mitarbeitern, Lehrbeauftragen und Tutoren eingestellt
worden. Die Finanzierung erfolgt aus Hochschulpaktmitteln und Mitteln
zur ``Verbesserung der Qualität der Lehre'' - Module werden im
Team-Teaching konzipiert und durchgeführt, dabei wird dem
verantwortlichen Dozenten ein Lehrbauftragter zur Seite gestellt, so
dass Lehrveranstaltungen und Workshops parallel durchgeführt werden
können. In diesen Arrangement sind in der Regel auch ein
Wissenschaftlicher Mitarbeiter und ein Tutor beteiligt. - Module nutzen
das Flipped Classroom Konzept, um den Studierenden einerseits den Zugang
zum wissenvermittelnden Material zu erleichtern und die andererseits die
gegebene Kontaktzeit besser zu nutzen - Module eines Semesters werden
sequentiell anstatt parallel durchgeführt. Dabei wird ein Modul in der
ersten Semesterhälfte durchgeführt und das andere in der zweiten
Semesterhälfte, wobei beide Module den Workload des anderen Moduls
nutzen, so dass der Workload für die Studierenden gleich groß bleibt.
Die Modulverantwortlichen haben damit Zugang zu deutliche mehr Räumen
und Ressourcen. Die Studierenden können sich besser auf ein Thema, bzw.
eine Domäne konzentieren, haben also weniger Kontexte gleichzeitig zu
bearbeiten. - stärkere Projektorientierung der anwendungsbezogenen
Module, wobei die Betreuung der Projektteams oftmals von
wissenschaftlichen Mitarbeitern erfolgt. - Zulassungsbeschränkung für
zum Wintersemester 2017/18 - Verlängerung über die Grenze des
Pensionsalters hinaus bei den Professoren Prof.~Dr.~Stenzel und
Prof.~Dr.~Jochum. Die Planstellen der Kollegen sind inzwischen trotzdem
neu besetzt, so dass durch die Überlappung von 2 bis 3 Jahren eine
größere Lehrkapazität zur Verfügung steht. Dieselbe Übergangsregelung
wird auch für die Kollegen/Innen Prof.~Dr.~Faekorn-Woyke,
Prof.~Dr.~Knittel und Prof.~Dr.~Klocke angestrebt.

Eine Übersicht über die verfügbaren Wissenschaftlichen Mitarbeiter und
Lehrbeauftragten\footnote{\href{https://th-koeln.github.io/mi-2017/anhaenge/stellungsnahme/mitarbeiter-und-module-mi-kern-2017.pdf}{Übersicht
  über alle Mitarbeiter und deren Einbindung in die Kernmodule des
  Medieninformatik Bachelor Studiengangs am Campus Gummersbach}}, sowie
deren Finanzierung findet sich im Anhang. Die Finanzierung erfolgt über
Hochschulpakt-Mittel, die zunächst bis zum Ende des Wintersemesters
2018/19 limitiert waren. \textbf{Inzwischen wurden die Mittel bis 2023
verlängt}, was in etwa dem Akkreditierungszeitraum entspricht. Die
wissenschaftlichen Mitarbeiter werden vor allem in projektorientieren
Modulen eingesetzt und übernehmen, neben organisatorischen Aufgaben, vor
allem die Mitbetreuung von Projektgruppen, sowie spezielle Schulungen in
Tools und Arbeitstechniken.

Eine Liste aller Dozenten und deren Beteiligung an Modulen in den
Informatik Studiengängen am Campus
Gummersbach(Lehrverflechtungsmatrix)\footnote{\href{https://th-koeln.github.io/mi-2017/anhaenge/stellungsnahme/dozenten-und-module-2017.pdf}{Übersicht
  über alle Dozenten und deren Module in den Informatik Studiengängen
  des Campus Gummersbach}} finden sich ebenfalls im Anhang.

\section{zu Kriterium 2.8: Fehlende Studien- und
Prüfungsordnungen\label{/mi-2017/selbstbericht/stellungnahme/0000-stellungnahme}}\label{zu-kriterium-2.8-fehlende-studien--und-pruxfcfungsordnungenpathlabelmi-2017selbstberichtstellungnahme0000-stellungnahme}

\subsection{Auszug aus dem Bericht der
Gutachter\label{/mi-2017/selbstbericht/stellungnahme/0000-stellungnahme}}\label{auszug-aus-dem-bericht-der-gutachterpathlabelmi-2017selbstberichtstellungnahme0000-stellungnahme-6}

\begin{siderules}
Nach Auskunft der Programmverantwortlichen muss die Studien- und
Prüfungsordnung des Bachelorstudiengangs nicht überarbeitet werden muss,
lediglich der Studienverlaufsplan, der Teil der Prüfungsordnung ist,
muss angepasst werden. Für den Masterstudiengang liegt lediglich der
Entwurf einer Studien- und Prüfungsordnung vor, der noch nicht offiziell
verabschiedet und veröffentlicht wurde. Dies muss für eine abschließende
Akkreditierung nachgeholt werden. Für beide Studiengänge liegen den
Gutachtern Diploma Supplements und Abschlusszeugnisse vor, die sich
jedoch noch auf die älteren Curricula beziehen. Die Gutachter erwarten
auch hierzu die Vorlage der überarbeiteten, angepassten Versionen.
\end{siderules}

\subsection{Stellungnahme der
Hochschule\label{/mi-2017/selbstbericht/stellungnahme/0000-stellungnahme}}\label{stellungnahme-der-hochschulepathlabelmi-2017selbstberichtstellungnahme0000-stellungnahme-5}

Die Prüfungsordnungen sind inzwischen von den entscheidungstragenden
Gremien verabschiedet worden und müssen lediglich noch veröffentlicht
werden. Überarbeitete Abschlusszeugnisse und Diploma Supplements sind
diesem Dokument angehängt.

\chapter{Nachlieferungen\label{/mi-2017/selbstbericht/stellungnahme/0100-nachlieferungen}}\label{nachlieferungenpathlabelmi-2017selbstberichtstellungnahme0100-nachlieferungen}

\section{Die folgenden Nachlieferungen wurden
angefragt\label{/mi-2017/selbstbericht/stellungnahme/0100-nachlieferungen}}\label{die-folgenden-nachlieferungen-wurden-angefragtpathlabelmi-2017selbstberichtstellungnahme0100-nachlieferungen}

\subsection{Lehrverflechtungsmatrix\label{/mi-2017/selbstbericht/stellungnahme/0100-nachlieferungen}}\label{lehrverflechtungsmatrixpathlabelmi-2017selbstberichtstellungnahme0100-nachlieferungen}

\begin{siderules}
Lehrverflechtungsmatrix für den Studiengang Medieninformatik inklusive
der in der Lehre tätigen Mitarbeiter (zusätzlich zu den Stunden, die in
verwandten Informatikstudiengängen absolviert werden).
\end{siderules}

Die Übersicht über die verfügbaren Wissenschaftlichen Mitarbeiter und
Lehrbeauftragten\footnote{\href{https://th-koeln.github.io/mi-2017/anhaenge/stellungsnahme/mitarbeiter-und-module-mi-kern-2017.pdf}{Übersicht
  über alle Mitarbeiter und deren Einbindung in die Kernmodule des
  Medieninformatik Bachelor Studiengangs am Campus Gummersbach}}, sowie
deren Finanzierung befindet sich im Anhang. Gleiches gilt für die
Übersicht aller Dozenten und deren Beteiligung an Modulen in den
Informatik Studiengängen am Campus
Gummersbach(Lehrverflechtungsmatrix)\footnote{\href{https://th-koeln.github.io/mi-2017/anhaenge/stellungsnahme/dozenten-und-module-2017.pdf}{Übersicht
  über alle Dozenten und deren Module in den Informatik Studiengängen
  des Campus Gummersbach}}.

\subsection{Zeugnismuster\label{/mi-2017/selbstbericht/stellungnahme/0100-nachlieferungen}}\label{zeugnismusterpathlabelmi-2017selbstberichtstellungnahme0100-nachlieferungen}

\begin{siderules}
Ggf. Zeugnismuster für die neue Studiengangsstruktur bzw. das neue
Curriculum mit Schwerpunkten. (Ich frage mich, ob nicht auch das DS
geändert werden müsste, um die Schwerpunkte sichtbar zu machen.)
\end{siderules}

Das Zeugnismuster für die neue Studiengangsstruktur befindet sich im
Anhang\footnote{\href{https://th-koeln.github.io/mi-2017/anhaenge/stellungsnahme/Urkunde_Zeugnis_Entwurf_black.pdf}{Zeugnismuster
  für die neue Studiengangsstruktur}}.
